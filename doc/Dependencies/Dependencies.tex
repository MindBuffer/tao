\documentclass[a4paper,twoside]{report}
\usepackage{hyperlatex}
\usepackage{epsfig}
\usepackage{a4}
\usepackage{caption}
\usepackage{pslatex}
\usepackage{float}
\usepackage{makeidx}

%% These commands are to set the overall format of the html output
%% produced by hyperlatex. See the hyperlatex user manual for details.

\setcounter{htmldepth}{5}
\setcounter{htmlautomenu}{1}

\newcommand{\toppanel}{
    \begin{rawhtml}<table width="500" border="0" align="left" cellspacing="2" cellpadding="2"><tr>\end{rawhtml}
    \begin{rawhtml}<td class="nav" valign="top"><!-- top panel -->\end{rawhtml}
    \EmptyP{\HlxUpUrl}
    {\xlink{\htmlimage[ALT="Up" ALIGN=BOTTOM BORDER=0]{up.gif}}{\HlxUpUrl}}
    { }
    \\\xlink{\HlxUpTitle}{\HlxUpUrl}\\
    \htmlimage[width="167" height="1"]{trans1x1.gif}
    \begin{rawhtml}</td>\end{rawhtml}
    \begin{rawhtml}<td class="nav" valign="top">\end{rawhtml}
    \EmptyP{\HlxBackUrl}
    {\xlink{\htmlimage[ALT="Back" ALIGN=BOTTOM BORDER=0]{back.gif}}{\HlxBackUrl}}
    { }
    \\\xlink{\HlxBackTitle}{\HlxBackUrl}\\
    \htmlimage[width="167" height="1"]{trans1x1.gif}
    \begin{rawhtml}</td>\end{rawhtml}
    \begin{rawhtml}<td class="nav" valign="top">\end{rawhtml}
    \EmptyP{\HlxForwUrl}
    {\xlink{\htmlimage[ALT="Forward" ALIGN=BOTTOM BORDER=0]{forward.gif}}{\HlxForwUrl}}
    { }
    \\\xlink{\HlxForwTitle}{\HlxForwUrl}\\
    \htmlimage[width="167" height="1"]{trans1x1.gif}
    \begin{rawhtml}</td></tr><!-- end top panel -->\end{rawhtml}
    \begin{rawhtml}<tr><td colspan="3" class="main"><!-- main text --><br><br>\end{rawhtml}
    }

\newcommand{\bottommatter}{
    \\
    \begin{rawhtml}</td></tr><!-- end main text -->\end{rawhtml}
    \begin{rawhtml}<tr>\end{rawhtml}
    \begin{rawhtml}<td class="nav" align="left" valign="top"><!-- bottom matter -->\end{rawhtml}
    \EmptyP{\HlxUpUrl}
    {\xlink{\htmlimage[ALT="Up" ALIGN=BOTTOM BORDER=0]{up.gif}}{\HlxUpUrl}}
    {}
    \\\xlink{\HlxUpTitle}{\HlxUpUrl}\\
    \htmlimage[width="167" height="1"]{trans1x1.gif}
    \begin{rawhtml}</td>\end{rawhtml}
    \begin{rawhtml}<td class="nav" align="left" valign="top">\end{rawhtml}
    \EmptyP{\HlxBackUrl}
    {\xlink{\htmlimage[ALT="Back" ALIGN=BOTTOM BORDER=0]{back.gif}}{\HlxBackUrl}}
    {}
    \\\xlink{\HlxBackTitle}{\HlxBackUrl}\\
    \htmlimage[width="167" height="1"]{trans1x1.gif}
    \begin{rawhtml}</td>\end{rawhtml}
    \begin{rawhtml}<td class="nav" align="left" valign="top"><!-- bottom matter -->\end{rawhtml}
    \EmptyP{\HlxForwUrl}
    {\xlink{\htmlimage[ALT="Forward" ALIGN=BOTTOM BORDER=0]{forward.gif}}{\HlxForwUrl}}
    {}
    \\\xlink{\HlxForwTitle}{\HlxForwUrl}
    \htmlimage[width="167" height="1"]{trans1x1.gif}
    \begin{rawhtml}</td></tr><!-- end bottom matter -->\end{rawhtml}
    }

\newcommand{\bottompanel}{
    \begin{rawhtml}<tr><td colspan="3" class="addr"><!-- bottom panel -->\end{rawhtml}
    \HlxBlk\EmptyP{\HlxAddress}
    {\html{ADDRESS}\HlxAddress\HlxBlk\html{/ADDRESS}\\}{}
    \begin{rawhtml}</td></tr><!-- end bottom panel -->\end{rawhtml}
    \begin{rawhtml}</table>\end{rawhtml}
    }

\htmltitle{Tao Dependencies}
\htmldirectory{html}
\htmladdress{\small\copyright 1999,2000 Mark Pearson
\xlink{m.pearson@ukonline.co.uk}{mailto:m.pearson@ukonline.co.uk} \today}

\htmlattributes{BODY}{BACKGROUND="bg.gif"}

\title{Tao Dependencies}
\author{Mark Pearson\\
m.pearson@ukonline.co.uk}
\date{\today}

\newcommand{\css}[1]
 {\renewcommand{\HlxMeta}
   {\begin{rawhtml}
    <link rel=stylesheet href="#1" type="text/css">
    \end{rawhtml}}}

\W\css{../../taomanual.css}
\renewcommand{\textsf}{}
\renewcommand{\samepage}{}

%% Tao logo typeset in bold font
\newcommand{\tao}{\textbf{Tao}}

%% TaoWebSite produces the following text in the printed document:
%% ... the Tao web site at http://web.ukonline.co.uk/taosynth ...
%% In the html document the phrase 'the Tao web site' is made into
%% a hyperlink.

\newcommand{\TaoWebSite}{\xlink{the Tao home page}[\begin{itemize}\item\Path{http://web.ukonline.co.uk/taosynth}\end{itemize}]{http://web.ukonline.co.uk/taosynth}}

% Image includes an eps image for TeX and either a jpg or gif for html
% Args:
%   #1 - base name of image file
%   #2 - extra instructions for epsfig
%   #3 - html image file extension [gif,jpg]

\newcommand{\Image}[3]{
    \texonly{\epsfig{file=#1.eps,#2}}
    \htmlonly{\htmlimage{#1.#3}}
    }

%% Commands for classifying index entries and typesetting
%% them in the main text accordingly

\newcommand{\hierindex}[1]{\texonly{\index{#1}}}
\newcommand{\Term}[1]{\emph{#1}\index{#1}}

\newcommand{\Index}[2]{#1\index{#1!#2}}

\newcommand{\Prog}[1]{\texttt{#1}}
\newcommand{\ProgIndex}[1]{\texttt{#1}\index{#1@\texttt{#1}}}

\newcommand{\Class}[1]{\texttt{\texttt{#1}}}
\newcommand{\ClassIndex}[1]{%
	\texttt{#1}%
	\index{classes!#1@\texttt{#1}}%
	\index{#1@\texttt{#1} class}}

\newcommand{\Device}[1]{\texttt{#1}}
\newcommand{\DeviceIndex}[1]{%
	\texttt{#1}%
	\texonly{\index{devices!#1@\texttt{#1}}}%
	\index{#1@\texttt{#1} device}}

\newcommand{\Instr}[1]{\texttt{#1}}
\newcommand{\InstrIndex}[1]{%
	\texttt{#1}%
	\texonly{\index{instrument!primitive types!#1@\texttt{#1}}}%
	\index{#1@\texttt{#1} instrument type}}

\newcommand{\Attr}[1]{\texttt{#1}}
\newcommand{\AttrIndex}[1]{%
	\texttt{#1}%
	\texonly{\index{attributes!#1@\textbf{\texttt{#1}}}}%
	\index{#1@\textbf{\texttt{#1}} attribute}}

\newcommand{\Operator}[1]{%
	\texonly{\index{operators!#1@\textbf{\texttt{#1}}}}%
	\index{#1@\textbf{\texttt{#1}}}}

\newcommand{\MathFunction}[1]{%
	\texonly{\index{math functions!#1@\textbf{\texttt{#1}}}}%
	\index{#1@\textbf{\texttt{#1}}}}

\newcommand{\Type}[1]{%
	\texttt{#1}%
	\texonly{\index{types!#1@\textbf{\texttt{#1}}}}%
	\index{#1@\textbf{\texttt{#1}} type}}

\newcommand{\Kwd}[1]{\texttt{#1}}
\newcommand{\KwdIndex}[1]{%
	\texttt{#1}%
	\texonly{\index{keywords!#1@\textbf{\texttt{#1}}}}%
	\index{#1@\textbf{\texttt{#1}} keyword}}

\newcommand{\Ctrl}[1]{\texttt{#1}}
\newcommand{\CtrlIndex}[1]{%
	\texttt{#1}%
	\texonly{\index{control structures!#1@\textbf{\texttt{#1}}}}%
	\index{#1@\textbf{\texttt{#1}} control structure}}

\newcommand{\Method}[1]{\texttt{#1}}
\newcommand{\MethodIndex}[1]{%
	\texttt{#1}%
	\texonly{\index{methods!#1@\textbf{\texttt{#1}}}}%
	\index{#1@\textbf{\texttt{#1}} method}}

\newcommand{\Var}[1]{\texttt{#1}}
\newcommand{\VarIndex}[1]{%
	\texttt{#1}%
	\texonly{\index{variables!#1@\textbf{\texttt{#1}}}}%
	\index{#1@\textbf{\texttt{#1}} variable}}

\newcommand{\Decl}[1]{#1}
\newcommand{\DeclIndex}[1]{%
	#1%
	\texonly{\index{declarations!#1@\emph{#1}}}%
	\index{#1@\emph{#1} declaration}}

\newcommand{\Filename}[1]{\texttt{#1}}


\newcommand{\Statement}[1]{#1}
\newcommand{\StatementIndex}[1]{%
	#1%
	\texonly{\index{statements!#1}}%
	\index{#1 statement}}

\newcommand{\EnvVar}[1]{\texttt{#1}}
\newcommand{\Path}[1]{\texttt{#1}}
\newcommand{\Lib}[1]{\texttt{#1}}
\newcommand{\Rpm}[1]{\texttt{#1}}

\newenvironment{CodeFragment}{\begin{verbatim}}{\end{verbatim}}



\makeindex

\title{Tao Dependencies}
\author{Mark Pearson}
\date{15 December 1999}

\renewcommand{\baselinestretch}{1.0}
\small
\normalsize

\begin{document}

\maketitle

\renewcommand{\baselinestretch}{1}
\small\normalsize

\T\tableofcontents

\chapter{Introduction}
This document describes external programs and libraries upon which
Tao is dependent. Most of these packages will either come as
standard with most Linux distributions or be available as easy to
install gzip'd tar archives or rpm files. There is no reason
why everything shouldn't work just as well on any UNIX system
but I have not had the chance to test the latest incarnation of
\tao\ on any other systems so I will concentrate on what I know,
i.e. Linux. 

The reason for writing this document is that I know that it can
sometimes be off-putting having to install several packages before
actually getting to the one you originally wanted to use. This
document attempts to lay out clearly what is needed, why it is
needed, what it actually provides that \tao\ uses, and where to find
it.

\chapter{Dependencies}
The main programs, libraries and header files you need to build \tao\ are
described below.

\section{Libraries and Headers}
\begin{itemize}
\item OpenGL or Mesa3D (3-D graphics library)
	\begin{itemize}
	\item libGL or libMesaGL and GL/gl.h
	\item libGLU or libMesaGLU and GL/glu.h
	\item libglut and glut.h
	\end{itemize}
\item lex or GNU flex (Lexical analyser generator)
\item yacc or GNU bison (Parser generator)
\item audiofile library (used to write .aiff format sound files)
	\begin{itemize}
	\item libaudiofile
	\item aupvlist.h
	\item audiofile.h
	\end{itemize}
\end{itemize}

\section{Tools used to build the programs and libraries}
\tao\ uses the GNU tools \verb|autoconf|, \verb|automake| and
\verb|libtool| for compilation and installation. You must have these
packages installed if you intend to build \tao\ from the sources.
\verb|autoconf| also relies upon the macro language GNU \verb|m4|
so you'll need that too! All of these packages are available from
\xlink{http://www.gnu.org}{http://www.gnu.org}.

\section{Tools used to build the documentation}
When writing documentation for a package there are so many options
and tools available that it can be difficult to decide which is
best. For \tao\ I chose \LaTeX\ and HTML as the main formats since
there are tools for converting from the former to the latter and
\LaTeX\ documents can be used to create postscript and thus PDF
format documents. I looked at \verb|texinfo| since this tool
is able to produce a wide range of formats from a single source
document but as yet I have never seen any \verb|texinfo|-derived
documentation containing images. I therefore opted for \LaTeX\ and
HTML since both can cope with images.

The following sections describe the tools which are required in order
to build Tao's documentation from source. You can find prebuilt
versions on \TaoWebSite

\subsection{Doxygen}
Doxygen is a tool for generating \LaTeX and HTML
documentation from C++ source files. It is able to generate a
complete reference including class members, file members, class
hierarchies, preprocessor symbols etc. By using specially formatted
comments in the sources it is possible to give detailed plain
English descriptions of all the elements in the package. It is
also very easy to install and is a superb tool.

Doxygen is available at the time of writing from:

\xlink{http://www.stack.nl/\~{}dimitri/doxygen}{http://www.stack.nl/\~{}dimitri/doxygen}

\subsection{Hyperlatex}
\verb|Hyperlatex| is a tool for producing high quality printed and
hypertext documentation from a single set of sources. Hyperlatex
uses a language which is a subset of \LaTeX. It requires
the inclusion of some \verb|usepackage| directives in the \LaTeX
sources and for best results some rewriting of existing \LaTeX
sources, but it is very flexible and configurable.

Hyperlatex is available at the time of writing from:

\xlink{ftp://ftp.cs.ust.hk/pub/ipe}{ftp://ftp.cs.ust.hk/pub/ipe}

or

\xlink{ftp://ftp.cs.uni-magdeburg.de/pub/ipe}{ftp://ftp.cs.uni-magdeburg.de/pub/ipe}

\end{document}





