\chapter{Introduction}
Welcome to the \tao\ user manual. This manual is for version
1.0-beta-04Mar2015 built on Wed Mar  4 16:16:25 AEDT 2015.

\tao\ is a software package for sound synthesis
using physical modelling techniques. It is written in C++ and provides
a kind of virtual material consisting of masses and springs, which can be
used to construct a wide variety of virtual musical instruments. New
instruments are described using a text-based synthesis language (contained
in what is referred to as a \tao\ \index{script} script). When a script is
invoked \tao\ carries out the synthesis described and automatically
produces 3-D animated visualisations of the instruments. The animations
show the actual acoustic waves propagating through the instruments
and play a central role in \tao's user interface. In particular they
make it much easier to think of the instruments as tangible physical
objects rather than abstract synthesis algorithms.

\tao\ will eventually come with full documentation for the C++ class
library API (application programming interface) so that it can also
be used by those who wish to write their own C++ applications
incorporating \tao's capabilities. This is not complete at the present
time.

\section{Where to find \tao}
\tao\ is available from the download section of \TaoWebSite

This user manual is also available for separate download either as a
tar'd gzip'd archive of HTML files or a gzip'd postscript file.

\section{Who was it designed for?}
\tao\ was conceived as a compositional tool for electroacoustic and
computer music. Therefore more emphasis has been placed on its ability
to produce a wide range of interesting and complex sounds than on
catering for traditional music based on scales, rhythms, harmony etc.
From a personal perspective I wanted to design a synthesis program
which would allow me to deal with tangible physical instruments
rather than abstract synthesis algorithms. I wanted to be able to
construct virtual musical instruments with very complex behaviours and
experiment with them in an intuitive physical manner.

\section{How \tao\ is structured}
\tao\ consists of a library of C++ classes for building synthesis
scenarios. The classes provide the means for creating
primitive acoustic building blocks, coupling these building blocks together
into more complex instruments, applying external excitations, and
generating sound output from the instruments. \tao's synthesis language
provides the same functionality as the C++ API but is easier to use. In
addition to providing a language for creating virtual instruments it
also provides an algorithmic score language for playing them. 

From the user's perspective the synthesis language appears to be interpreted
rather than compiled since a script may be invoked with a simple one-line
command. In reality though this process involves translating the \tao\
script into C++ first, linking it with the C++ \tao\ library
\Filename{libtao.so} and then invoking the executable produced. You
shouldn't have to worry about the details of this process when using the
synthesis language interface though.

\section{What this manual contains}
This user manual is divided into eight chapters as follows:

\begin{description}
\item[1. Introduction] This introductory chapter.
\item[2. Building and Installing \tao] Instructions on where
to obtain the \tao\ distribution and how to build and install it.
\item[3. Conceptual Overview] Introduces the main concepts which should
be understood before attempting to use \tao.
\item[4. Getting Started] Skips the details and goes straight to
a practical example session using \tao.
\item[5. Tao's User Interface] Self-explanatory.
\item[5. \tao's Synthesis Language in Detail] Gives a detailed description
of the synthesis language provided.
\item[6. Object Method Reference] A detailed reference for all of \tao's
classes which are available within the synthesis language. 
\item[7. Tutorial] Gives lots of example scripts describing what they
do and how.
\item[8. Closing Comments] General comments about the current state
of \tao\ and areas for future development. 
\end{description}

\section{Typographic conventions used}
This manual adopts a number of typographic conventions
\index{typographic conventions} which are described below.

The \texttt{monospace} font indicates shell window output,
filenames and names of environment variables; keywords, functions
and operators in \tao's synthesis language; and finally verbatim
examples which should be typed exactly as shown.

The \textbf{\texttt{monospace bold}} font is used in the index to highlight
entries for keywords, functions and operators in \tao's synthesis language.

The \emph{italic} font indicates important terminology being introduced for
the first time.

In addition to the above conventions sometimes it is necessary to indicate
where the user needs to supply some values in a \tao\ script. For example
in the following script fragment \verb|<pitch>| and \verb|<decay time>| would
need to be replaced with appropriate values by the user:

\begin{verbatim}
    String myString(<pitch>, <decay time>);
\end{verbatim}

Finally \tao\ refers to the entire software package whereas \verb|tao|
refers to the name of the shell command used for executing a \tao\ script.















