\chapter{Building and Installing \tao}
\label{section:installation}
\tao\ was initially developed on SGI Irix 5.3 but was ported in 1998 to
Red Hat Linux 5.0. Since then it has been developed primarily on Red
Hat 6.0. It has not been tested with other \index{Unix}Unix systems but
since it only uses widely available Unix tools and platform
independent libraries it is highly probable that it would work just
as well on other systems. 

\section{What else do you need to have installed?}
\label{section:whatdoyouneed}
\tao\ requires a handful of other programs and libraries to be
installed before it will work properly. The main packages which
you absolutely need are listed first in this section. Towards
the end of the section more details are given of tools which
are only required if you want to build the documentation from
the sources.

The essential packages to have installed are

\begin{itemize}
\item The GNU C++ compiler;
\item \Prog{OpenGL} (or \Prog{Mesa}) libraries and header files;
\item GL Utility Toolkit (\Prog{GLUT}) libraries and headers;
\item The \Prog{lex} and \Prog{yacc} compiler tools (or \Prog{flex}
and \Prog{bison} if you are using GNU versions).
\item Michael Pruett's port of the SGI audiofile library.
\end{itemize}

The GLUT libraries and headers come packaged with Mesa
in the most recent versions so you don't have to search for them
separately if you choose to use Mesa as your OpenGL replacement,
but otherwise you may have to download and install them separately.
Similarly the audiofile libraries and headers should be available
on SGI machines, and Red Hat 6.0 comes as standard with Michael
Pruetts implementation of this API (although one of the header
files had a syntax error which I have temporarily fixed by including
the corrected headers with this distribution). But if you can't
find any \Lib{libaudiofile.*} files on your system then you need
to download this package too.

Source packages for the above are available at the following URLs:

\begin{itemize}
\item\xlink{http://www.mesa3d.org}{http://www.mesa3d.org}
\item\xlink{http://reality.sgi.com/opengl/glut3}{http://reality.sgi.com/opengl/glut3}
\item\xlink{http://www.68k.org/~michael/audiofile/}{http://www.68k.org/~michael/audiofile/} 
\end{itemize}

Follow the installation instructions provided with each package. In practice
this should be quite a simple process.

NOTE: One thing to bear in mind is that if you choose instead to
download and install RPM packages for the above you must install
the associated development packages also. For example the audiofile RPMs
installed on my system include the following:

\begin{itemize}
\item\Rpm{audiofile-0.1.6-5}
\item\Rpm{audiofile-devel-0.1.6-5}
\end{itemize}

If you don't have the latter \emph{development} package then none of the
header files for the library will be installed, and in addition some
essential symbolic links will be missing in the library directory.

For Mesa and GLUT the RPMs you need are:

\begin{itemize}
\item\Rpm{Mesa-3.0.*}
\item\Rpm{Mesa-devel-3.0.*}
\item\Rpm{Mesa-glut-3.0.*}
\item\Rpm{Mesa-glut-devel-3.0.*}
\end{itemize}

Later versions should work just as well but if you have any problems
please let me know so that I can try to sort them out.

\section{Configuring, making and installing \tao}
As with most GNU-style software there are three easy steps to installing
\tao\ assuming everything goes to plan. First change to the directory
where you have unpacked the distribution and type the following
commands one by one, waiting for any intervening output from each
command to finish before typing the next.

\begin{verbatim}
    ./configure
    make
    make install
\end{verbatim}

The default path for installation of the binaries, library files and
shell scripts is \Path{/usr/local} so you will need root access in order
to use the default. If you do not have root access then change the
above to:

\begin{verbatim}
     ./configure --prefix=<your path>
\end{verbatim}

where \verb|<your path>| is the full path to wherever you want to
install Tao.

The configure script checks to see if you have the necessary programs
headers and libraries installed. If you do not the configuration will
abort with a message telling you what is missing. 

\subsection{Troubleshooting the configuration process}
If the configure script fails it should give you some feedback about
what it can't find on your system. One of the most common problems
is not being able to find library files.

Two common things to check for are: 

\begin{enumerate}
\item
Check the value of the \EnvVar{LD\_LIBRARY\_PATH} environment variable. This
is used to tell your system where to look for libraries which are
not installed in \Path{/usr/lib}. Quite often packages which you install
yourself will put library files in \Path{/usr/local/lib} by default. If 
\EnvVar{LD\_LIBRARY\_PATH} doesn't point to this directory (or wherever else
the library files are installed) then programs which depend on these
libraries at run time will not be able to find them.
 
To find out the value type:

\begin{verbatim}
   echo $LD_LIBRARY_PATH
\end{verbatim}

If the value is empty or doesn't contain \Path{/usr/local/lib} or any of
the paths where your libraries are located in its colon separated list of paths
then you must amend it so that it does. To do this first find out which shell you
use by typing:

\begin{verbatim}
   echo $SHELL
\end{verbatim}

If you're using the \Prog{bash} shell see section \ref{section:bash_shell}
below for details of how to amend the value. If you're using the \Prog{tcsh}
shell see section \ref{section:tcsh_shell}.

\item
If you install Mesa, GLUT or audiofile via RPM binary distributions
check that you have the appropriate \emph{development} packages installed
also. These include:

\begin{itemize}
\item\Rpm{audiofile-devel-0.1.6-*}
\item\Rpm{Mesa-devel-3.0.*}
\item\Rpm{Mesa-glut-devel-3.0.*}
\end{itemize}

These packages provide header files and symbolic links to the libraries
(e.g. \Lib{libaudiofile.so} linked to \Lib{libaudiofile.so.0.0}).
Without these packages the libraries themselves may be installed
but you still won't be able to compile and link programs with them.
\end{enumerate}

If, after reading this section you are still baffled then take a look
at the next section too, since there is a further tool you can use to
help diagnose problems.

\subsection{If you are still stuck with configuration problems}
\label{section:stillstuck}
After receiving a emails from some \tao\ users who had run into
problems with the configure script I decided to write a shell script
as and aid to testing for installed libraries, their locations,
and whether or not the executables dependent on those libraries
would be able to find them. This shell script is located in the top
level directory of the distribution and is called \Prog{diagnose-lib}.

Typing \verb|diagnose-lib| without any arguments prints out the following
usage message:

\begin{verbatim}
    Usage: diagnose-lib <libname>
    Diagnose problems in finding shared libraries during
    configuration of Tao

    <libname> can be one of the following:
    'gl', 'GL', 'glu', 'GLU', 'glut' or 'audiofile'. 
\end{verbatim}

So, for example, if the configure script claims that you don't have any
OpenGL libraries installed but you are convinced that you do, type:

\begin{verbatim}
    ./diagnose-lib gl
\end{verbatim}
to check for the GL library or
\begin{verbatim}
    ./diagnose-lib glu
\end{verbatim}
for the GLU library.

The \verb|diagnose-lib| script will respond with information about whether
it can find a library of the right name, where that library is installed
and whether the executables which depend on that library will be able
to find it. It does so by searching obvious locations such as
\verb|/usr/lib|, \verb|/usr/local/lib| first and then searches the
directory tree rooted at your home directory. If it fails to find
the library it will abort and let you know.

If on the other hand it does find a suitable candidate it then checks
 to see whether either the file \Filename{/etc/ld.so.conf} or the environment
variable \EnvVar{LD\_LIBRARY\_PATH} contain the appropriate path to
find this file. These are the two principal mechanisms by which your
system locates \emph{shared objects} or \emph{dynamic link libraries}
at run time.

If neither contain the path to this file a suitable message is printed out
and suggestions for solving the problem are given.

If you find that you are still having problems after following any advice
given by the \verb|diagnose-lib| script then please feel free to email
me at \xlink{m.pearson@ukonline.co.uk}{mailto:m.pearson@ukonline.co.uk}.
I will try to help out where I can.

\subsection{Continuing with the build process}
Assuming the configuration part worked you can continue with the
build process, i.e. \verb|make| and \verb|make install|. After
this you should have the following files installed (assuming that
\Path{prefix=/usr/local}):

\begin{verbatim}
    /usr/local/  
        lib/
            libtao.so*
            libtao.a
        bin/
            tao
            tao-config
            taosf
            taoparse
            tao2wav
\end{verbatim}

The install process leaves \tao's header files where they are but
provides a shell script \Prog{tao-config} which can be used to find out
where both these headers and the various libraries are installed.
This is particularly useful if you want to write your own C++ programs
and link them against the \tao\ library. It is used in the following
way:

\begin{verbatim}
    tao-config --prefix     =>  location for installed files
    tao-config --cflags     =>  command line flags for the compiler
                                to find Tao's header files
    tao-config --libs       =>  command line flags for the compiler
                                to find Tao's libraries
\end{verbatim}

The next step is important. In order for your system to locate
the binary executables, shell scripts and libraries you have to set
two environment variables: \EnvVar{PATH} and \EnvVar{LD\_LIBRARY\_PATH}.
This process is described step by step in the following sections. The
first thing you need to do though is find out which UNIX shell you use.
To do this type:

\begin{verbatim}
    echo $SHELL
\end{verbatim}

\subsubsection{Setting PATH and LD\_LIBRARY\_PATH for the bash shell}
\label{section:bash_shell}
Type the following to see if \Path{/usr/local/bin} is already in your path:

\begin{verbatim}
    echo $PATH
\end{verbatim}

If not then open the \Filename{.bash\_profile} file in your home directory and
add the following lines:

\begin{verbatim}
    PATH=$PATH:/usr/local/bin
    export PATH
\end{verbatim}

Then type the following to see if \Path{/usr/local/lib} is in your
library loading path:

\begin{verbatim}
    echo $LD_LIBRARY_PATH
\end{verbatim}

If not then add the following lines to the \Filename{.bash\_profile} file in
your home directory:

\begin{verbatim}
    LD_LIBRARY_PATH=$LD_LIBRARY_PATH:/usr/local/lib
    export LD_LIBRARY_PATH
\end{verbatim}

\subsubsection{Setting PATH and LD\_LIBRARY\_PATH for the tcsh shell}
\label{section:tcsh_shell}
Type the following to see if \Path{/usr/local/bin} is in your path:

\begin{verbatim}
    echo $PATH
\end{verbatim}

If not then add the following line to the \Filename{.tcshrc} file in
your home directory:

\begin{verbatim}
    setenv PATH $PATH:/usr/local/bin
\end{verbatim}

Then type the following to see if \Path{/usr/local/lib} is in your
library loading path:

\begin{verbatim}
    echo $LD_LIBRARY_PATH
\end{verbatim}

If not then add the following line to the \Filename{.tcshrc} file in
your home directory:

\begin{verbatim}
    setenv LD_LIBRARY_PATH $LD_LIBRARY_PATH:/usr/local/lib
\end{verbatim}

\section{What the distribution contains}
This distribution has the following directory structure:

\begin{verbatim}
    libtao/
    tao2wav/
    tao2aiff/
    taoparse/
    include/
    user-scripts/
    examples/
    doc/
        UserManual/
            html/
        ClassReference/
            latex/
            html/
        Dependencies/
            html/
\end{verbatim}

These directories contain the following:

\begin{description}
\item[libtao/] -- source code for the C++ library which \tao\
is built upon.
\item[tao2wav/] -- source code for the program which converts
\tao's raw floating point output files in \verb|.wav|\index{WAV output
files} format ready for playback.
\item[taoparse/] -- lex and yacc source code for
\tao's synthesis language parser.
\item[include/] -- header files for C++ classes.
\item[user-scripts/] -- a set of user shell scripts including
the following:
\begin{description}
\item[tao] -- shell script for compiling and executing a \tao\ script.
\item[tao-config] -- shell script which is useful when compiling
and linking C++ programs with the \tao\ library \Lib{libtao}. 
\item[taosf] -- shell script for converting \tao's output files to WAV
format.
\end{description}

\item[examples/] -- a set of examples illustrating the
main elements of \tao's synthesis language.
\item[doc/] -- documentation of various kinds.
This includes the following:
\begin{description}

\item[UserManual/] -- \Prog{Hyperlatex} sources for this manual which
can be used to produce DVI and PostScript versions using \LaTeX\ and 
\Prog{dvips}, and the HTML version using \Prog{hyperlatex}. The HTML formatted
version comes ready made with this distribution so you don't need
\Prog{hyperlatex}, although the \emph{Dependencies} document describes
where to get it.

\item[ClassReference/] -- \LaTeX\ and HTML documentation for the
C++ library API. Both versions of this document are produced automatically
from the C++ sources using a third party program called \Prog{Doxygen}.
See the \emph{Dependencies} document for details of where
to get \Prog{Doxygen}.

[Actually I haven't got around to this yet as I am concentrating on
making the distribution as robust as possible and finishing the user
manual, but it will happen eventually].

\item[Dependencies/] -- \Prog{hyperlatex} sources for the document
describing the external programs upon which \tao\ depends. Once
again these sources can be used to produce HTML, DVI and 
PostScript formatted versions and the HTML version comes ready built
with the distribution.

\end{description}
\end{description}

The installation step described in the previous section installs the
following files on your system (assuming \Path{prefix=/usr/local}):

\begin{verbatim}
    /usr/local/bin
        tao
        taosf
        tao2wav
        taoparse
    /usr/local/lib
        libtao.so.*
        libtao.a
\end{verbatim}

\section{Testing \tao}
To test that everything is working once the installation is complete open
a new shell window. Change to the top level directory in the \tao\
distribution copy the script \Path{examples/test.tao} to your home
directory. This test script contains the following text although you don't
need to understand how it works for the moment.

\begin{verbatim}
    Audio rate: 44100;

    Circle c(800 Hz, 20 secs);
    String strings[4]=
        {
        (800 Hz, 20 secs),
        (810 Hz, 20 secs),
        (820 Hz, 20 secs),
        (830 Hz, 20 secs)
        };

    Rectangle r(800 Hz, 900 Hz, 20 secs);
    Triangle t(800 Hz, 900 Hz, 20 secs);
    Connector conn1, conn2, conn3, conn4;
    Counter s;

    Init:
        For s = 0 to 3:
            strings[s].lockEnds();
            ...

        c.lockPerimeter();
        r.lockCorners();
        t.lockLeft().lockRight();

        strings[0](0.1) -- conn1 -- strings[1](0.1);
        strings[1](0.9) -- conn2 -- strings[2](0.9);
        strings[2](0.1) -- conn3 -- strings[3](0.1);

        r(0.6,0.2) -- conn4 -- 0.0;

        r.placeRightOf(c,20);
        t.placeAbove(r);
        ...

    Score 20 secs:
        At start for 0.1 msecs:
            strings[0](0.1).applyForce(1.0);
            strings[1](0.1).applyForce(1.0);
            strings[2](0.1).applyForce(1.0);
            strings[3](0.1).applyForce(1.0);

            c(0.1,0.5).applyForce(10.0);
            r(0.7,0.8).applyForce(10.0);
            t(0.8,0.6).applyForce(10.0);
            ...
        ...
\end{verbatim}

Change directory in the shell window to your home directory and type:
\begin{verbatim}
    tao test
\end{verbatim}

If everything is working OK \tao\ should respond firstly by printing
the following messages in the shell window:

\begin{verbatim}
    ========================================
    |     Tao (c) 1996-99 Mark Pearson     |
    | Sound Synthesis with Physical Models |
    ========================================

    Processing test.tao
    Making test.exe
    Executing test.exe

    Sample rate=44100 KHz
    Score duration=1 seconds                
\end{verbatim}

It should then open a window like the one shown in figure \ref{fig:instrvis}.
This is \tao's \emph{instrument visualisation window}\index{instrument!visualisation window},
which presents a 3-D animated representation of the instruments
described in the script \Filename{test.tao}. In the current implementation this
window is meant for visualisation only, it is not possible to edit the
instruments graphically.

\begin{figure}[htb]
  \begin{Label}{fig:instrvis}
    \begin{center}
    \Image{instrvis}{height=7cm}{gif}
    \end{center}
    \caption{\tao's instrument visualisation window}
  \end{Label}
\end{figure}

This is a good opportunity to try some of the key and mouse bindings which
affect the behaviour of the instrument visualisation window. These are
listed below but there is one thing you have to do first to set the synthesis
engine in motion.

\textbf{IMPORTANT:} When the visualisation window opens initially the
instrument animation is paused. This gives you time to move/resize/rotate
the image to get the view you want before setting everything in motion.
So the first thing you need to do is press the \textbf{right-arrow} key.
After doing this you should see the instruments spring to life, showing
the propagating waves. You can now try out some of the key and mouse bindings.

\subsection*{Key bindings}
\label{section:key_and_mouse_bindings}
\index{key bindings}
\begin{description}
\item[Down-arrow] Reduce the visible amplitude of the vibrations.
\item[Up-arrow] Increase the visible amplitude of the vibrations.
\item[Left-arrow] Decrease the number of ticks between displayed frames
i.e. make the animation slower but smoother.
\item[Right-arrow] Increase the number of ticks between displayed frames
i.e. make the animation faster but more jerky.
\item[Esc] Exit and close the graphics window.
\end{description}

\subsection*{Mouse bindings}
\index{mouse bindings}
In addition there are a number of mouse functions which work when the
appropriate button is held down and the mouse is moved:

\begin{description}
\item[Left-mouse] Translate the image.
\item[Middle-mouse] Zoom in/out.
\item[Right-mouse] Rotate the image.
\end{description}











