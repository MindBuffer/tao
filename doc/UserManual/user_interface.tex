\chapter{\tao's User Interface}
\index{user interface}

This section describes \tao's user interface in detail and includes
shell commands, and mouse and key bindings for the visualization
window.

\section{Shell commands}
A number of shell commands\index{shell commands}
are provided for executing scripts and converting
the output files to WAV format. These are described below:

\renewcommand{\descriptionlabel}[1]%
	{\hspace{\labelsep}\texttt{#1}}

\begin{description}
\item[tao <script>]
Takes the name of a script as its argument (minus the \verb|.tao|
extension) and executes it, producing a file called \verb|<script>.exe|.
This shell script calls upon the services of the binary executable
\Prog{taoparse}, which is described below.

\item[taoparse <script>.tao]
Binary executable which parses a \tao\ script, translates it into a
valid C++ program, and writes the results to standard output. This is
called by the \Prog{tao} shell script.

\item[taosf <outputfile>]
Takes the name of a file produced by an Output device, minus the
\verb|.dat| extension and normalises the audio samples, adding
a WAV header and writing the results to a file called \verb|<outputfile>.wav|.
This command actually calls the \verb|tao2wav| binary executable with
the arguments \verb|tao2wav <outputfile>.dat <outputfile>.wav|.

\item[tao2wav <outputfile>.dat <outputfile>.wav]
Binary executable used to convert a file produced by an Output device
(in raw floating point format) into a WAV file.
\end{description}

In addition to these commands, whenever a script is invoked with the
\verb|tao| command a corresponding \verb|.exe| file is produced. This
is the actual executable which carries out the synthesis described in
the script. For example the command \verb|tao <file>| invokes the script
\verb|<file>.tao| and as a by-product produces the executable file
\verb|<file>.exe|. Each \verb|.exe| file has one command line option,
\verb|[-g]|. This option causes the instrument visualisation window
to be opened. If omitted the synthesis proceeds without any visualisations.

If you want to replay a synthesis described in a script you can invoke
this executable file directly rather than having to re-compile the original
text file containing the script.

\section{Mouse bindings in the visualisation window}
\index{mouse bindings}
Moving the mouse whilst holding one of the mouse buttons down in the
instrument visualisation window causes the image to be zoomed, translated
or rotated. The mouse bindings are as follows:

\renewcommand{\descriptionlabel}[1]%
	{\hspace{\labelsep}\textbf{#1}}
\begin{description}
\item[Left button] -- translate the image.
\item[Middle button] -- zoom the image.
\item[Right button] -- rotate the image.
\end{description}

\section{Key bindings in the visualisation window}
\index{key bindings}
The visualisation window also has some associated key bindings. These are
as follows:

\begin{description}
\item[Left arrow key] -- increase the frequency with which the visualisation
is updated relative to ticks of the synthesis engine. This causes the
visual animation to become slower but smoother. If updating is already
occurring on every tick of the synthesis engine then an additional
press of this key pauses the whole synthesis engine until the right arrow key
is pressed.
\item[Right arrow key] -- decrease the frequency with which the visualisation
is updated relative to ticks of the synthesis engine. This causes the
visual animation to become faster but less smooth as more computational
resources are given over to the synthesis engine itself. This key also takes
the synthesis engine out of \emph{pause mode} if it has been paused.
\item[Up arrow key] -- increase the amplitude of the waves depicted in the
visualisation.
\item[Down arrow key] -- decrease the amplitude of the waves depicted in the
visualisation.
\item[I key] -- toggle display of instrument names. This can be useful if
the visualisation becomes too cluttered with information.
\item[D key] -- toggle display of device names.
\item[ESC key] -- exit the synthesis, closing the visualisation window.
\end{description}

