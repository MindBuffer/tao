\documentclass[a4paper,twoside]{report}
\usepackage{hyperlatex}
\usepackage{epsfig}
\usepackage{a4}
\usepackage{caption}
\usepackage{pslatex}
\usepackage{float}
\usepackage{makeidx}

%% These commands are to set the overall format of the html output
%% produced by hyperlatex. See the hyperlatex user manual for details.

\setcounter{htmldepth}{5}
\setcounter{htmlautomenu}{1}

\newcommand{\toppanel}{
    \begin{rawhtml}<table width="500" border="0" align="left" cellspacing="2" cellpadding="2"><tr>\end{rawhtml}
    \begin{rawhtml}<td class="nav" valign="top"><!-- top panel -->\end{rawhtml}
    \EmptyP{\HlxUpUrl}
    {\xlink{\htmlimage[ALT="Up" ALIGN=BOTTOM BORDER=0]{up.gif}}{\HlxUpUrl}}
    { }
    \\\xlink{\HlxUpTitle}{\HlxUpUrl}\\
    \htmlimage[width="167" height="1"]{trans1x1.gif}
    \begin{rawhtml}</td>\end{rawhtml}
    \begin{rawhtml}<td class="nav" valign="top">\end{rawhtml}
    \EmptyP{\HlxBackUrl}
    {\xlink{\htmlimage[ALT="Back" ALIGN=BOTTOM BORDER=0]{back.gif}}{\HlxBackUrl}}
    { }
    \\\xlink{\HlxBackTitle}{\HlxBackUrl}\\
    \htmlimage[width="167" height="1"]{trans1x1.gif}
    \begin{rawhtml}</td>\end{rawhtml}
    \begin{rawhtml}<td class="nav" valign="top">\end{rawhtml}
    \EmptyP{\HlxForwUrl}
    {\xlink{\htmlimage[ALT="Forward" ALIGN=BOTTOM BORDER=0]{forward.gif}}{\HlxForwUrl}}
    { }
    \\\xlink{\HlxForwTitle}{\HlxForwUrl}\\
    \htmlimage[width="167" height="1"]{trans1x1.gif}
    \begin{rawhtml}</td></tr><!-- end top panel -->\end{rawhtml}
    \begin{rawhtml}<tr><td colspan="3" class="main"><!-- main text --><br><br>\end{rawhtml}
    }

\newcommand{\bottommatter}{
    \\
    \begin{rawhtml}</td></tr><!-- end main text -->\end{rawhtml}
    \begin{rawhtml}<tr>\end{rawhtml}
    \begin{rawhtml}<td class="nav" align="left" valign="top"><!-- bottom matter -->\end{rawhtml}
    \EmptyP{\HlxUpUrl}
    {\xlink{\htmlimage[ALT="Up" ALIGN=BOTTOM BORDER=0]{up.gif}}{\HlxUpUrl}}
    {}
    \\\xlink{\HlxUpTitle}{\HlxUpUrl}\\
    \htmlimage[width="167" height="1"]{trans1x1.gif}
    \begin{rawhtml}</td>\end{rawhtml}
    \begin{rawhtml}<td class="nav" align="left" valign="top">\end{rawhtml}
    \EmptyP{\HlxBackUrl}
    {\xlink{\htmlimage[ALT="Back" ALIGN=BOTTOM BORDER=0]{back.gif}}{\HlxBackUrl}}
    {}
    \\\xlink{\HlxBackTitle}{\HlxBackUrl}\\
    \htmlimage[width="167" height="1"]{trans1x1.gif}
    \begin{rawhtml}</td>\end{rawhtml}
    \begin{rawhtml}<td class="nav" align="left" valign="top"><!-- bottom matter -->\end{rawhtml}
    \EmptyP{\HlxForwUrl}
    {\xlink{\htmlimage[ALT="Forward" ALIGN=BOTTOM BORDER=0]{forward.gif}}{\HlxForwUrl}}
    {}
    \\\xlink{\HlxForwTitle}{\HlxForwUrl}
    \htmlimage[width="167" height="1"]{trans1x1.gif}
    \begin{rawhtml}</td></tr><!-- end bottom matter -->\end{rawhtml}
    }

\newcommand{\bottompanel}{
    \begin{rawhtml}<tr><td colspan="3" class="addr"><!-- bottom panel -->\end{rawhtml}
    \HlxBlk\EmptyP{\HlxAddress}
    {\html{ADDRESS}\HlxAddress\HlxBlk\html{/ADDRESS}\\}{}
    \begin{rawhtml}</td></tr><!-- end bottom panel -->\end{rawhtml}
    \begin{rawhtml}</table>\end{rawhtml}
    }

\htmltitle{Tao User Manual}
\htmldirectory{html}
\htmladdress{\small\copyright 1999,2000 Mark Pearson
\xlink{m.pearson@ukonline.co.uk}{mailto:m.pearson@ukonline.co.uk} \today}

\htmlattributes{BODY}{BACKGROUND="bg.gif"}

\title{Tao User Manual}
\author{Mark Pearson\\
m.pearson@ukonline.co.uk}
\date{\today}

\newcommand{\css}[1]
 {\renewcommand{\HlxMeta}
   {\begin{rawhtml}
    <link rel=stylesheet href="#1" type="text/css">
    \end{rawhtml}}}

\W\css{../../taomanual.css}
\renewcommand{\textsf}{}
\renewcommand{\samepage}{}

%% Tao logo typeset in bold font
\newcommand{\tao}{\textbf{Tao}}

%% TaoWebSite produces the following text in the printed document:
%% ... the Tao web site at http://web.ukonline.co.uk/taosynth ...
%% In the html document the phrase 'the Tao web site' is made into
%% a hyperlink.

\newcommand{\TaoWebSite}{\xlink{the Tao home page}[\begin{itemize}\item\Path{http://web.ukonline.co.uk/taosynth}\end{itemize}]{http://web.ukonline.co.uk/taosynth}}

% Image includes an eps image for TeX and either a jpg or gif for html
% Args:
%   #1 - base name of image file
%   #2 - extra instructions for epsfig
%   #3 - html image file extension [gif,jpg]

\newcommand{\Image}[3]{
    \texonly{\epsfig{file=#1.eps,#2}}
    \htmlonly{\htmlimage{#1.#3}}
    }

%% Commands for classifying index entries and typesetting
%% them in the main text accordingly
\newcommand{\hierindex}[1]{\texonly{\index{#1}}}
\newcommand{\Term}[1]{\emph{#1}\index{#1}}

\newcommand{\Index}[2]{#1\index{#1!#2}}

\newcommand{\Prog}[1]{\texttt{#1}}
\newcommand{\ProgIndex}[1]{\texttt{#1}\index{#1@\texttt{#1}}}

\newcommand{\Class}[1]{\texttt{\texttt{#1}}}
\newcommand{\ClassIndex}[1]{%
	\texttt{#1}%
	\index{classes!#1@\texttt{#1}}%
	\index{#1@\texttt{#1} class}}

\newcommand{\Device}[1]{\texttt{#1}}
\newcommand{\DeviceIndex}[1]{%
	\texttt{#1}%
	\texonly{\index{devices!#1@\texttt{#1}}}%
	\index{#1@\texttt{#1} device}}

\newcommand{\Instr}[1]{\texttt{#1}}
\newcommand{\InstrIndex}[1]{%
	\texttt{#1}%
	\texonly{\index{instrument!primitive types!#1@\texttt{#1}}}%
	\index{#1@\texttt{#1} instrument type}}

\newcommand{\Attr}[1]{\texttt{#1}}
\newcommand{\AttrIndex}[1]{%
	\texttt{#1}%
	\texonly{\index{attributes!#1@\textbf{\texttt{#1}}}}%
	\index{#1@\textbf{\texttt{#1}} attribute}}

\newcommand{\Operator}[1]{%
	\texonly{\index{operators!#1@\textbf{\texttt{#1}}}}%
	\index{#1@\textbf{\texttt{#1}}}}

\newcommand{\MathFunction}[1]{%
	\texonly{\index{math functions!#1@\textbf{\texttt{#1}}}}%
	\index{#1@\textbf{\texttt{#1}}}}

\newcommand{\Type}[1]{%
	\texttt{#1}%
	\texonly{\index{types!#1@\textbf{\texttt{#1}}}}%
	\index{#1@\textbf{\texttt{#1}} type}}

\newcommand{\Kwd}[1]{\texttt{#1}}
\newcommand{\KwdIndex}[1]{%
	\texttt{#1}%
	\texonly{\index{keywords!#1@\textbf{\texttt{#1}}}}%
	\index{#1@\textbf{\texttt{#1}} keyword}}

\newcommand{\Ctrl}[1]{\texttt{#1}}
\newcommand{\CtrlIndex}[1]{%
	\texttt{#1}%
	\texonly{\index{control structures!#1@\textbf{\texttt{#1}}}}%
	\index{#1@\textbf{\texttt{#1}} control structure}}

\newcommand{\Method}[1]{\texttt{#1}}
\newcommand{\MethodIndex}[1]{%
	\texttt{#1}%
	\texonly{\index{methods!#1@\textbf{\texttt{#1}}}}%
	\index{#1@\textbf{\texttt{#1}} method}}

\newcommand{\Var}[1]{\texttt{#1}}
\newcommand{\VarIndex}[1]{%
	\texttt{#1}%
	\texonly{\index{variables!#1@\textbf{\texttt{#1}}}}%
	\index{#1@\textbf{\texttt{#1}} variable}}

\newcommand{\Decl}[1]{#1}
\newcommand{\DeclIndex}[1]{%
	#1%
	\texonly{\index{declarations!#1@\emph{#1}}}%
	\index{#1@\emph{#1} declaration}}

\newcommand{\Filename}[1]{\texttt{#1}}


\newcommand{\Statement}[1]{#1}
\newcommand{\StatementIndex}[1]{%
	#1%
	\texonly{\index{statements!#1}}%
	\index{#1 statement}}

\newcommand{\EnvVar}[1]{\texttt{#1}}
\newcommand{\Path}[1]{\texttt{#1}}
\newcommand{\Lib}[1]{\texttt{#1}}
\newcommand{\Rpm}[1]{\texttt{#1}}

\newenvironment{CodeFragment}{\begin{verbatim}}{\end{verbatim}}

%\setmarginsrb{35mm}{10mm}{30mm}{35mm}{20pt}{20pt}{20pt}{36pt}

\makeindex

\T\raggedbottom

\begin{document}

\maketitle

\renewcommand{\baselinestretch}{1}
\small\normalsize

\T\tableofcontents
\T\listoffigures

\chapter{Introduction}
Welcome to the \tao\ user manual. This manual is for version
1.0-beta-04Mar2015 built on Wed Mar  4 16:16:25 AEDT 2015.

\tao\ is a software package for sound synthesis
using physical modelling techniques. It is written in C++ and provides
a kind of virtual material consisting of masses and springs, which can be
used to construct a wide variety of virtual musical instruments. New
instruments are described using a text-based synthesis language (contained
in what is referred to as a \tao\ \index{script} script). When a script is
invoked \tao\ carries out the synthesis described and automatically
produces 3-D animated visualisations of the instruments. The animations
show the actual acoustic waves propagating through the instruments
and play a central role in \tao's user interface. In particular they
make it much easier to think of the instruments as tangible physical
objects rather than abstract synthesis algorithms.

\tao\ will eventually come with full documentation for the C++ class
library API (application programming interface) so that it can also
be used by those who wish to write their own C++ applications
incorporating \tao's capabilities. This is not complete at the present
time.

\section{Where to find \tao}
\tao\ is available from the download section of \TaoWebSite

This user manual is also available for separate download either as a
tar'd gzip'd archive of HTML files or a gzip'd postscript file.

\section{Who was it designed for?}
\tao\ was conceived as a compositional tool for electroacoustic and
computer music. Therefore more emphasis has been placed on its ability
to produce a wide range of interesting and complex sounds than on
catering for traditional music based on scales, rhythms, harmony etc.
From a personal perspective I wanted to design a synthesis program
which would allow me to deal with tangible physical instruments
rather than abstract synthesis algorithms. I wanted to be able to
construct virtual musical instruments with very complex behaviours and
experiment with them in an intuitive physical manner.

\section{How \tao\ is structured}
\tao\ consists of a library of C++ classes for building synthesis
scenarios. The classes provide the means for creating
primitive acoustic building blocks, coupling these building blocks together
into more complex instruments, applying external excitations, and
generating sound output from the instruments. \tao's synthesis language
provides the same functionality as the C++ API but is easier to use. In
addition to providing a language for creating virtual instruments it
also provides an algorithmic score language for playing them. 

From the user's perspective the synthesis language appears to be interpreted
rather than compiled since a script may be invoked with a simple one-line
command. In reality though this process involves translating the \tao\
script into C++ first, linking it with the C++ \tao\ library
\Filename{libtao.so} and then invoking the executable produced. You
shouldn't have to worry about the details of this process when using the
synthesis language interface though.

\section{What this manual contains}
This user manual is divided into eight chapters as follows:

\begin{description}
\item[1. Introduction] This introductory chapter.
\item[2. Building and Installing \tao] Instructions on where
to obtain the \tao\ distribution and how to build and install it.
\item[3. Conceptual Overview] Introduces the main concepts which should
be understood before attempting to use \tao.
\item[4. Getting Started] Skips the details and goes straight to
a practical example session using \tao.
\item[5. Tao's User Interface] Self-explanatory.
\item[5. \tao's Synthesis Language in Detail] Gives a detailed description
of the synthesis language provided.
\item[6. Object Method Reference] A detailed reference for all of \tao's
classes which are available within the synthesis language. 
\item[7. Tutorial] Gives lots of example scripts describing what they
do and how.
\item[8. Closing Comments] General comments about the current state
of \tao\ and areas for future development. 
\end{description}

\section{Typographic conventions used}
This manual adopts a number of typographic conventions
\index{typographic conventions} which are described below.

The \texttt{monospace} font indicates shell window output,
filenames and names of environment variables; keywords, functions
and operators in \tao's synthesis language; and finally verbatim
examples which should be typed exactly as shown.

The \textbf{\texttt{monospace bold}} font is used in the index to highlight
entries for keywords, functions and operators in \tao's synthesis language.

The \emph{italic} font indicates important terminology being introduced for
the first time.

In addition to the above conventions sometimes it is necessary to indicate
where the user needs to supply some values in a \tao\ script. For example
in the following script fragment \verb|<pitch>| and \verb|<decay time>| would
need to be replaced with appropriate values by the user:

\begin{verbatim}
    String myString(<pitch>, <decay time>);
\end{verbatim}

Finally \tao\ refers to the entire software package whereas \verb|tao|
refers to the name of the shell command used for executing a \tao\ script.
















\chapter{Building and Installing \tao}
\label{section:installation}
\tao\ was initially developed on SGI Irix 5.3 but was ported in 1998 to
Red Hat Linux 5.0. Since then it has been developed primarily on Red
Hat 6.0. It has not been tested with other \index{Unix}Unix systems but
since it only uses widely available Unix tools and platform
independent libraries it is highly probable that it would work just
as well on other systems. 

\section{What else do you need to have installed?}
\label{section:whatdoyouneed}
\tao\ requires a handful of other programs and libraries to be
installed before it will work properly. The main packages which
you absolutely need are listed first in this section. Towards
the end of the section more details are given of tools which
are only required if you want to build the documentation from
the sources.

The essential packages to have installed are

\begin{itemize}
\item The GNU C++ compiler;
\item \Prog{OpenGL} (or \Prog{Mesa}) libraries and header files;
\item GL Utility Toolkit (\Prog{GLUT}) libraries and headers;
\item The \Prog{lex} and \Prog{yacc} compiler tools (or \Prog{flex}
and \Prog{bison} if you are using GNU versions).
\item Michael Pruett's port of the SGI audiofile library.
\end{itemize}

The GLUT libraries and headers come packaged with Mesa
in the most recent versions so you don't have to search for them
separately if you choose to use Mesa as your OpenGL replacement,
but otherwise you may have to download and install them separately.
Similarly the audiofile libraries and headers should be available
on SGI machines, and Red Hat 6.0 comes as standard with Michael
Pruetts implementation of this API (although one of the header
files had a syntax error which I have temporarily fixed by including
the corrected headers with this distribution). But if you can't
find any \Lib{libaudiofile.*} files on your system then you need
to download this package too.

Source packages for the above are available at the following URLs:

\begin{itemize}
\item\xlink{http://www.mesa3d.org}{http://www.mesa3d.org}
\item\xlink{http://reality.sgi.com/opengl/glut3}{http://reality.sgi.com/opengl/glut3}
\item\xlink{http://www.68k.org/~michael/audiofile/}{http://www.68k.org/~michael/audiofile/} 
\end{itemize}

Follow the installation instructions provided with each package. In practice
this should be quite a simple process.

NOTE: One thing to bear in mind is that if you choose instead to
download and install RPM packages for the above you must install
the associated development packages also. For example the audiofile RPMs
installed on my system include the following:

\begin{itemize}
\item\Rpm{audiofile-0.1.6-5}
\item\Rpm{audiofile-devel-0.1.6-5}
\end{itemize}

If you don't have the latter \emph{development} package then none of the
header files for the library will be installed, and in addition some
essential symbolic links will be missing in the library directory.

For Mesa and GLUT the RPMs you need are:

\begin{itemize}
\item\Rpm{Mesa-3.0.*}
\item\Rpm{Mesa-devel-3.0.*}
\item\Rpm{Mesa-glut-3.0.*}
\item\Rpm{Mesa-glut-devel-3.0.*}
\end{itemize}

Later versions should work just as well but if you have any problems
please let me know so that I can try to sort them out.

\section{Configuring, making and installing \tao}
As with most GNU-style software there are three easy steps to installing
\tao\ assuming everything goes to plan. First change to the directory
where you have unpacked the distribution and type the following
commands one by one, waiting for any intervening output from each
command to finish before typing the next.

\begin{verbatim}
    ./configure
    make
    make install
\end{verbatim}

The default path for installation of the binaries, library files and
shell scripts is \Path{/usr/local} so you will need root access in order
to use the default. If you do not have root access then change the
above to:

\begin{verbatim}
     ./configure --prefix=<your path>
\end{verbatim}

where \verb|<your path>| is the full path to wherever you want to
install Tao.

The configure script checks to see if you have the necessary programs
headers and libraries installed. If you do not the configuration will
abort with a message telling you what is missing. 

\subsection{Troubleshooting the configuration process}
If the configure script fails it should give you some feedback about
what it can't find on your system. One of the most common problems
is not being able to find library files.

Two common things to check for are: 

\begin{enumerate}
\item
Check the value of the \EnvVar{LD\_LIBRARY\_PATH} environment variable. This
is used to tell your system where to look for libraries which are
not installed in \Path{/usr/lib}. Quite often packages which you install
yourself will put library files in \Path{/usr/local/lib} by default. If 
\EnvVar{LD\_LIBRARY\_PATH} doesn't point to this directory (or wherever else
the library files are installed) then programs which depend on these
libraries at run time will not be able to find them.
 
To find out the value type:

\begin{verbatim}
   echo $LD_LIBRARY_PATH
\end{verbatim}

If the value is empty or doesn't contain \Path{/usr/local/lib} or any of
the paths where your libraries are located in its colon separated list of paths
then you must amend it so that it does. To do this first find out which shell you
use by typing:

\begin{verbatim}
   echo $SHELL
\end{verbatim}

If you're using the \Prog{bash} shell see section \ref{section:bash_shell}
below for details of how to amend the value. If you're using the \Prog{tcsh}
shell see section \ref{section:tcsh_shell}.

\item
If you install Mesa, GLUT or audiofile via RPM binary distributions
check that you have the appropriate \emph{development} packages installed
also. These include:

\begin{itemize}
\item\Rpm{audiofile-devel-0.1.6-*}
\item\Rpm{Mesa-devel-3.0.*}
\item\Rpm{Mesa-glut-devel-3.0.*}
\end{itemize}

These packages provide header files and symbolic links to the libraries
(e.g. \Lib{libaudiofile.so} linked to \Lib{libaudiofile.so.0.0}).
Without these packages the libraries themselves may be installed
but you still won't be able to compile and link programs with them.
\end{enumerate}

If, after reading this section you are still baffled then take a look
at the next section too, since there is a further tool you can use to
help diagnose problems.

\subsection{If you are still stuck with configuration problems}
\label{section:stillstuck}
After receiving a emails from some \tao\ users who had run into
problems with the configure script I decided to write a shell script
as and aid to testing for installed libraries, their locations,
and whether or not the executables dependent on those libraries
would be able to find them. This shell script is located in the top
level directory of the distribution and is called \Prog{diagnose-lib}.

Typing \verb|diagnose-lib| without any arguments prints out the following
usage message:

\begin{verbatim}
    Usage: diagnose-lib <libname>
    Diagnose problems in finding shared libraries during
    configuration of Tao

    <libname> can be one of the following:
    'gl', 'GL', 'glu', 'GLU', 'glut' or 'audiofile'. 
\end{verbatim}

So, for example, if the configure script claims that you don't have any
OpenGL libraries installed but you are convinced that you do, type:

\begin{verbatim}
    ./diagnose-lib gl
\end{verbatim}
to check for the GL library or
\begin{verbatim}
    ./diagnose-lib glu
\end{verbatim}
for the GLU library.

The \verb|diagnose-lib| script will respond with information about whether
it can find a library of the right name, where that library is installed
and whether the executables which depend on that library will be able
to find it. It does so by searching obvious locations such as
\verb|/usr/lib|, \verb|/usr/local/lib| first and then searches the
directory tree rooted at your home directory. If it fails to find
the library it will abort and let you know.

If on the other hand it does find a suitable candidate it then checks
 to see whether either the file \Filename{/etc/ld.so.conf} or the environment
variable \EnvVar{LD\_LIBRARY\_PATH} contain the appropriate path to
find this file. These are the two principal mechanisms by which your
system locates \emph{shared objects} or \emph{dynamic link libraries}
at run time.

If neither contain the path to this file a suitable message is printed out
and suggestions for solving the problem are given.

If you find that you are still having problems after following any advice
given by the \verb|diagnose-lib| script then please feel free to email
me at \xlink{m.pearson@ukonline.co.uk}{mailto:m.pearson@ukonline.co.uk}.
I will try to help out where I can.

\subsection{Continuing with the build process}
Assuming the configuration part worked you can continue with the
build process, i.e. \verb|make| and \verb|make install|. After
this you should have the following files installed (assuming that
\Path{prefix=/usr/local}):

\begin{verbatim}
    /usr/local/  
        lib/
            libtao.so*
            libtao.a
        bin/
            tao
            tao-config
            taosf
            taoparse
            tao2wav
\end{verbatim}

The install process leaves \tao's header files where they are but
provides a shell script \Prog{tao-config} which can be used to find out
where both these headers and the various libraries are installed.
This is particularly useful if you want to write your own C++ programs
and link them against the \tao\ library. It is used in the following
way:

\begin{verbatim}
    tao-config --prefix     =>  location for installed files
    tao-config --cflags     =>  command line flags for the compiler
                                to find Tao's header files
    tao-config --libs       =>  command line flags for the compiler
                                to find Tao's libraries
\end{verbatim}

The next step is important. In order for your system to locate
the binary executables, shell scripts and libraries you have to set
two environment variables: \EnvVar{PATH} and \EnvVar{LD\_LIBRARY\_PATH}.
This process is described step by step in the following sections. The
first thing you need to do though is find out which UNIX shell you use.
To do this type:

\begin{verbatim}
    echo $SHELL
\end{verbatim}

\subsubsection{Setting PATH and LD\_LIBRARY\_PATH for the bash shell}
\label{section:bash_shell}
Type the following to see if \Path{/usr/local/bin} is already in your path:

\begin{verbatim}
    echo $PATH
\end{verbatim}

If not then open the \Filename{.bash\_profile} file in your home directory and
add the following lines:

\begin{verbatim}
    PATH=$PATH:/usr/local/bin
    export PATH
\end{verbatim}

Then type the following to see if \Path{/usr/local/lib} is in your
library loading path:

\begin{verbatim}
    echo $LD_LIBRARY_PATH
\end{verbatim}

If not then add the following lines to the \Filename{.bash\_profile} file in
your home directory:

\begin{verbatim}
    LD_LIBRARY_PATH=$LD_LIBRARY_PATH:/usr/local/lib
    export LD_LIBRARY_PATH
\end{verbatim}

\subsubsection{Setting PATH and LD\_LIBRARY\_PATH for the tcsh shell}
\label{section:tcsh_shell}
Type the following to see if \Path{/usr/local/bin} is in your path:

\begin{verbatim}
    echo $PATH
\end{verbatim}

If not then add the following line to the \Filename{.tcshrc} file in
your home directory:

\begin{verbatim}
    setenv PATH $PATH:/usr/local/bin
\end{verbatim}

Then type the following to see if \Path{/usr/local/lib} is in your
library loading path:

\begin{verbatim}
    echo $LD_LIBRARY_PATH
\end{verbatim}

If not then add the following line to the \Filename{.tcshrc} file in
your home directory:

\begin{verbatim}
    setenv LD_LIBRARY_PATH $LD_LIBRARY_PATH:/usr/local/lib
\end{verbatim}

\section{What the distribution contains}
This distribution has the following directory structure:

\begin{verbatim}
    libtao/
    tao2wav/
    tao2aiff/
    taoparse/
    include/
    user-scripts/
    examples/
    doc/
        UserManual/
            html/
        ClassReference/
            latex/
            html/
        Dependencies/
            html/
\end{verbatim}

These directories contain the following:

\begin{description}
\item[libtao/] -- source code for the C++ library which \tao\
is built upon.
\item[tao2wav/] -- source code for the program which converts
\tao's raw floating point output files in \verb|.wav|\index{WAV output
files} format ready for playback.
\item[taoparse/] -- lex and yacc source code for
\tao's synthesis language parser.
\item[include/] -- header files for C++ classes.
\item[user-scripts/] -- a set of user shell scripts including
the following:
\begin{description}
\item[tao] -- shell script for compiling and executing a \tao\ script.
\item[tao-config] -- shell script which is useful when compiling
and linking C++ programs with the \tao\ library \Lib{libtao}. 
\item[taosf] -- shell script for converting \tao's output files to WAV
format.
\end{description}

\item[examples/] -- a set of examples illustrating the
main elements of \tao's synthesis language.
\item[doc/] -- documentation of various kinds.
This includes the following:
\begin{description}

\item[UserManual/] -- \Prog{Hyperlatex} sources for this manual which
can be used to produce DVI and PostScript versions using \LaTeX\ and 
\Prog{dvips}, and the HTML version using \Prog{hyperlatex}. The HTML formatted
version comes ready made with this distribution so you don't need
\Prog{hyperlatex}, although the \emph{Dependencies} document describes
where to get it.

\item[ClassReference/] -- \LaTeX\ and HTML documentation for the
C++ library API. Both versions of this document are produced automatically
from the C++ sources using a third party program called \Prog{Doxygen}.
See the \emph{Dependencies} document for details of where
to get \Prog{Doxygen}.

[Actually I haven't got around to this yet as I am concentrating on
making the distribution as robust as possible and finishing the user
manual, but it will happen eventually].

\item[Dependencies/] -- \Prog{hyperlatex} sources for the document
describing the external programs upon which \tao\ depends. Once
again these sources can be used to produce HTML, DVI and 
PostScript formatted versions and the HTML version comes ready built
with the distribution.

\end{description}
\end{description}

The installation step described in the previous section installs the
following files on your system (assuming \Path{prefix=/usr/local}):

\begin{verbatim}
    /usr/local/bin
        tao
        taosf
        tao2wav
        taoparse
    /usr/local/lib
        libtao.so.*
        libtao.a
\end{verbatim}

\section{Testing \tao}
To test that everything is working once the installation is complete open
a new shell window. Change to the top level directory in the \tao\
distribution copy the script \Path{examples/test.tao} to your home
directory. This test script contains the following text although you don't
need to understand how it works for the moment.

\begin{verbatim}
    Audio rate: 44100;

    Circle c(800 Hz, 20 secs);
    String strings[4]=
        {
        (800 Hz, 20 secs),
        (810 Hz, 20 secs),
        (820 Hz, 20 secs),
        (830 Hz, 20 secs)
        };

    Rectangle r(800 Hz, 900 Hz, 20 secs);
    Triangle t(800 Hz, 900 Hz, 20 secs);
    Connector conn1, conn2, conn3, conn4;
    Counter s;

    Init:
        For s = 0 to 3:
            strings[s].lockEnds();
            ...

        c.lockPerimeter();
        r.lockCorners();
        t.lockLeft().lockRight();

        strings[0](0.1) -- conn1 -- strings[1](0.1);
        strings[1](0.9) -- conn2 -- strings[2](0.9);
        strings[2](0.1) -- conn3 -- strings[3](0.1);

        r(0.6,0.2) -- conn4 -- 0.0;

        r.placeRightOf(c,20);
        t.placeAbove(r);
        ...

    Score 20 secs:
        At start for 0.1 msecs:
            strings[0](0.1).applyForce(1.0);
            strings[1](0.1).applyForce(1.0);
            strings[2](0.1).applyForce(1.0);
            strings[3](0.1).applyForce(1.0);

            c(0.1,0.5).applyForce(10.0);
            r(0.7,0.8).applyForce(10.0);
            t(0.8,0.6).applyForce(10.0);
            ...
        ...
\end{verbatim}

Change directory in the shell window to your home directory and type:
\begin{verbatim}
    tao test
\end{verbatim}

If everything is working OK \tao\ should respond firstly by printing
the following messages in the shell window:

\begin{verbatim}
    ========================================
    |     Tao (c) 1996-99 Mark Pearson     |
    | Sound Synthesis with Physical Models |
    ========================================

    Processing test.tao
    Making test.exe
    Executing test.exe

    Sample rate=44100 KHz
    Score duration=1 seconds                
\end{verbatim}

It should then open a window like the one shown in figure \ref{fig:instrvis}.
This is \tao's \emph{instrument visualisation window}\index{instrument!visualisation window},
which presents a 3-D animated representation of the instruments
described in the script \Filename{test.tao}. In the current implementation this
window is meant for visualisation only, it is not possible to edit the
instruments graphically.

\begin{figure}[htb]
  \begin{Label}{fig:instrvis}
    \begin{center}
    \Image{instrvis}{height=7cm}{gif}
    \end{center}
    \caption{\tao's instrument visualisation window}
  \end{Label}
\end{figure}

This is a good opportunity to try some of the key and mouse bindings which
affect the behaviour of the instrument visualisation window. These are
listed below but there is one thing you have to do first to set the synthesis
engine in motion.

\textbf{IMPORTANT:} When the visualisation window opens initially the
instrument animation is paused. This gives you time to move/resize/rotate
the image to get the view you want before setting everything in motion.
So the first thing you need to do is press the \textbf{right-arrow} key.
After doing this you should see the instruments spring to life, showing
the propagating waves. You can now try out some of the key and mouse bindings.

\subsection*{Key bindings}
\label{section:key_and_mouse_bindings}
\index{key bindings}
\begin{description}
\item[Down-arrow] Reduce the visible amplitude of the vibrations.
\item[Up-arrow] Increase the visible amplitude of the vibrations.
\item[Left-arrow] Decrease the number of ticks between displayed frames
i.e. make the animation slower but smoother.
\item[Right-arrow] Increase the number of ticks between displayed frames
i.e. make the animation faster but more jerky.
\item[Esc] Exit and close the graphics window.
\end{description}

\subsection*{Mouse bindings}
\index{mouse bindings}
In addition there are a number of mouse functions which work when the
appropriate button is held down and the mouse is moved:

\begin{description}
\item[Left-mouse] Translate the image.
\item[Middle-mouse] Zoom in/out.
\item[Right-mouse] Rotate the image.
\end{description}












\chapter{Conceptual Overview}
\label{section:conceptual_overview}
This section introduces the main concepts which you will need to have
some grasp of in order to use \tao\ at anything but the simplest level.
Topics covered include \tao's \emph{cellular material}; \emph{instruments}; 
\emph{devices}; \emph{access points}; \emph{parameters}; and \emph{pitches}.

\section{\tao's Cellular Material}
\label{section:cellular_material}

\tao\ is based around the notion of building complex vibrating structures
from simpler acoustic building blocks. In order to realise this goal 
a general purpose adaptable acoustic material is provided. The material
consists of point masses arranged in a regular grid pattern and connected
together with springs. The overall structure of the material is shown in
figure \ref{fig:cells}. 

\begin{figure}[htb]
  \begin{Label}{fig:cells}
    \begin{center}
    \Image{cells}{height=5cm}{gif}
    \end{center}
    \caption{A small portion of \tao's cellular acoustic material}
  \end{Label}
\end{figure}

Each point mass or \Term{cell} maintains a set of state variables for
its position, velocity, mass, etc., and the overall state is updated in
discrete time steps or \Term{ticks} according to rules which take into
account a cell's own state and the states of its immediate neighbours.
More specifically a cell's spring connections to its neighbours exert
forces on the cell and from these forces Newton's laws of motion can
be used to calculate the acceleration and velocity of the cell.

Note that the cells are constrained to have one degree of freedom (in the
direction of the $z$ axis). This has two practical advantages:

\begin{itemize}
\item It makes the calculations involved in animating the material simpler.
\item It also makes it a simple matter to generate time varying waveforms
from the vibrations in the material since all cells will vibrate about a
fixed zero reference point at $z=0$ so long as at least one cell
is fixed at that position.
\end{itemize}

One question which has often been asked in relation to \tao's cellular
material is: `have you experimented with 3D blocks of material?'. The answer
to this question is no, for the following reasons. Firstly, the computational
expense of such instruments would be prohibitive, and secondly, although the
restriction of working with 2D instruments may at first seem like a limitation,
in practice it doesn't significantly affect \tao's ability to produce an
wide variety of interesting sounds.

\subsection{Cell Attributes}
\label{section:cell_attributes}
The most important \Term{attributes} maintained by each cell are its
\Attr{mass}, \Attr{position}, \Attr{velocity}, 
\Attr{force} and lastly \Attr{velocityMultiplier}. Not
surprisingly these are used to keep track of the cell's state of motion
and the forces acting upon it. The \Class{Cell} class is unusual in
that it is seldom dealt with directly. However we will see that many
of \tao's other classes have attributes which are often accessed
and set via \Term{methods}

The \Attr{velocityMultiplier} attribute is quite important. It holds
a value in the range [0..1] and the velocity of the cell is multiplied
by this value on each tick. This has the effect of dissipating energy (if the
value is $<$ 1) and thus damping the vibrations of the cell. Each cell
can have a different value for this attribute leading to non-uniform
damping of the material and this feature is a very important tool for
controlling the vibrational characteristics of the material as we will
see later on in this manual. 

For example with a simple string-like instrument consisting of a single row
of cells linked together with springs, damping small regions at either end
of the string more highly than the rest of the cells
causes the higher frequency vibrations to die away more quickly than the
lower ones. This leads to a more natural string-like spectral decay in the
sounds produced by the instrument, whereas a string with uniform damping
exhibits no significant change in the distribution of spectral energy as
the sound evolves. 

It is probably safe to say that all physically produced sounds
exhibit some kind of spectral evolution, and as a general rule
non-uniform damping always produces more interesting sounds from
\tao\ instruments than uniform damping.

The other main attributes that each cell maintains are pointers to its
neighbours \Attr{north}, \Attr{east}, \Attr{south},
\Attr{west}, \Attr{neast}, \Attr{nwest}, \Attr{seast}
and \Attr{swest}. Each pointer indicates to a particular cell that
it is connected to a neighbouring cell via a virtual spring. Similarly the
neighbouring cell will reciprocate by keeping a pointer to the first. You do
not need to deal with these pointers directly as an end-user \tao\ but it
is worth knowing that they are there. 

The final attribute is used to store other aspects of the cell's state,
such as whether it is locked or free to move. This attribute is called
\Attr{mode}.

\subsection{The Emergent Behaviour of the Material}
Having described the microscopic structure of the cellular material
it is now time to say something about its macroscopic behaviour.
One of the most appealing features of computer models in which many
simple elements interact on a local basis according to well-defined
rules is that they often exhibit interesting \Term{emergent behaviour}.
In \tao's case the material appears to behave (not surpisingly) like an
continuous elastic sheet when viewed from a distance.

Figure \ref{fig:circle_example} shows a typical piece of \tao's material,
in this case a circular sheet which has had a short impulse applied at a
single point. The image is a snapshot taken some short time interval later
and clearly shows wavefronts spreading out from the point of impact and
also reflecting off of the boundary of the object.

\begin{figure}[htb]
  \begin{Label}{fig:circle_example}
    \begin{center}
    \Image{circle_example}{height=6cm}{gif}
    \end{center}
    \caption{Screenshot of typical \tao\ instrument}
  \end{Label}
\end{figure}

In the graphical representation used the individual cells and springs
are not visible. Instead what we see is a wireframe representation in which
each line represents a single row of cells. Note how smoothly contoured
the waves are, giving the impression that the material is 
continuous and elastic in behaviour.

\subsection{Generating Sound Output from the Material}
In order to generate output waveforms the vibrations of the material
are `listened' to directly. In order to achieve this a `sensor' is placed on the
surface of the material at a chosen point and a time-varying trace of
the motion of the point with respect to the $z$ axis is written to an
output file. This numerical data can then be played back as digital
audio samples.

Output signals may be derived from mathematical expressions
involving more than one point on an instrument. In addition, the
positions of the points from which the signals are generated may be moved
around under algorithmic control. These are just two of the tools at
your disposal for creating interesting dynamically evolving sounds.

\section{Instruments and Devices}
\label{section:instruments_and_devices}
Although the cellular material is actually composed of many hundreds of
individual objects representing the cells and springs, from the user's
point of view a higher level of abstraction is provided for creating
and interacting with pieces of the material. This abstraction comes in
the form of \emph{instruments}\index{instrument} and
\emph{devices}\index{device}.

\tao\ provides a set of classes for creating primitive acoustic
building blocks. These are derived from a generic \ClassIndex{Instrument}
class and include \Instr{String}, \Instr{Rectangle},
\Instr{Circle}, \Instr{Ellipse} and \Instr{Triangle}
\texonly{\index{instrument!building blocks}}. The \Instr{String}
class creates a single line of cells and springs whilst the other
classes create 2-D sheets of material in a variety of shapes. 
Using these instrument classes you don't need to worry about
creating the individual cells or springs which link them together
as this is taken care of for you. 

It should be mentioned here that in the rest of this document the
generic term \Term{instrument} is used to refer both to these simple
building blocks and also more complex arrangements in which several pieces of
material are coupled together. In the latter case the term
\Term{compound instrument} is used. 

A number of other classes derived from a common \ClassIndex{Device} class
are also provided. Devices are objects which:

\begin{itemize}
\item
allow the primitive instruments listed above to be coupled together into more
interesting \Term{compound instruments};
\item 
provide the means for applying external excitations to the instruments;
\item
enable sound output to be generated by `listening' to points on instruments,
writing the resulting time varying waveforms to sound files.
\end{itemize}

Devices which are available in the current version of
\tao\ include bows, hammers, connectors, stops and outputs.
The purpose of each of each type of device is explained in the
following sections. But before moving on another fundamental
object class needs to be introduced -- the \Term{access point}.
Whenever a device interacts with an instrument in some way an
access point is involved. You don't need to worry too much
about how to use them at the moment but they are described
later on in section \ref{section:access_points}.

\subsection*{The Bow Device}
\label{section:bow_device}
\index{Bow device}
The bow device provides the user with a model of the interaction
between a bow and an instrument. It works by mathematically modelling the
static and dynamical frictional forces which occur between the bow and the
instrument. Each bow device has the following user accessible
attributes:

\begin{description}
\item[\Attr{velocity}] -- the current velocity of the bow.
\item[\Attr{force}] -- the downward force applied to the bow. 
\end{description}

The bowing model used in this device is based loosely upon a model
developed by Woodhouse.

\subsection*{The Hammer Device}
\label{section:hammer_device}
\index{Hammer device}
The Hammer device provides a generalised mechanism for producing
percussive sounds. A Hammer device has the following attributes:

\begin{description}
\item[\Attr{mass}] -- the mass of the hammer.
\item[\Attr{height}] -- the initial height from which the hammer is dropped.
\item[\Attr{position}] -- the current height of the hammer.
\item[\Attr{initVelocity}] -- the initial velocity of the hammer.
\item[\Attr{velocity}] -- the current velocity of the hammer.
\item[\Attr{maxImpacts}] -- the maximum number of the impacts with the
target instrument.
\item[\Attr{numImpacts}] -- the number of impacts with the target instrument
since the hammer was dropped.
\item[\Attr{gravity}] -- gravitational acceleration acting upon the hammer.
\item[\Attr{damping}] -- degree of damping applied to the hammer.
\item[\Attr{hardness}] -- how hard the impact surface of the hammer is.
\end{description}

All of these parameters may be altered under algorithmic control except the
number of impacts which is read only. 

\subsection*{The Connector Device}
\label{section:connector_device}
\index{Connector device}
The Connector device provides a flexible mechanism for coupling 
instrumental components together and coupling points on instrumental 
components to fixed \Term{anchor} points. It does so by installing 
springs between the
access points or anchors specified. Anchor points may be numerical constants
(usually 0.0) or arbitrary expressions, the latter being useful for driving
an instrument with an external signal (external to the instrument, not \tao\ 
itself).

The technique of connecting an access point to a fixed anchor point
is sometimes useful for restricting the amplitude of vibrations at certain
points on an instrument. For example a component might have too many low
frequency partials making the sound too bottom-heavy in which case various
points on the instrument can be connected to zero anchor points allowing only
the higher partials to continue vibrating. An extra attribute allows the
strength of the installed spring to be set. This takes a value in the range
[0..1] normally although higher values may work.

One of the most powerful features of \tao\ lies in the fact that the
coordinates specifying the position of an access points do not have to be
constant. They can be time-varying values derived from expressions in the
score. This leads to the ability to create instruments which morph
structurally as they are being played. The Connector device is
therefore one of the most important provided by \tao\ since it enables
complex, dynamically evolving instruments to be constructed.

\subsection*{The Output Device}
\label{output_device}
\index{Output device}
The Output device provides a general means for writing floating point
samples to output files. The samples are initially un-normalised but the 
resulting data files may be normalised and converted into WAV format sound
files with a separate program \Prog{tao2wav}. The minor inconvenience of
having to convert \tao\ output files into a more usable format as a separate
post-processing stage is imposed for good reason. Firstly it does have the
advantage that the user doesn't have to worry about sound samples going out
of range. Also, with a model which is heavily based around floating point
calculations and physically vibrating entities it is impossible to know the
amplitude of the vibrations in advance.

The present version of \tao\ only allows one and two channel output files but
this will be changed in future releases to allow for arbitrary numbers of
channels.

\subsection*{The Stop Device}
\label{stop_device}
\index{Stop device}
The Stop device provides a mechanism for producing specific pitches from a
string by stopping or fretting  it, i.e. temporarily changing its effective
vibrating length. The Stop device is not an entirely accurate model of
how a stopped string behaves and will probably be refined in future releases.
It currently works by installing a Connector device between
the specified access point and a fixed anchor position at 0.0. This restricts
the vibrations of the string at the access point. In addition there is
a `damping' attribute which causes a small region around the access point to
be more highly damped than the rest of the string. This causes the string to
almost stand still at the access point, which is the desired effect.


\subsection{The Information Needed to Create an Instrument}

When a new instrument is created three pieces of information are required
from the user in the case of rectangular and elliptical sheets and two in
the case of strings and circular sheets. These are:

\begin{itemize}
\item A frequency or pitch specifying the dimensions of the instrument
in the $x$ direction
\item A frequency or pitch specifying the dimensions of the instrument
in the $y$ direction (for rectangular and elliptical sheets only)
\item A decay time
\end{itemize}

The dimensions of a new instrument are always described in terms of
pitches or frequencies rather than physical units such as metres or
millimetres. For example when creating a new string the pitch specified
is used to determine the length of the string.
You may be thinking `what about the tension in the string?', but in
\tao's cellular material the tension is fixed at an optimum value which
gives the best frequency response, so the only factor affecting the
pitch of a string is its length and vice versa.

In the case of a rectangular sheet two pitches or frequencies are
required. The first relates to the time taken for a wavefront to make
the round trip from the left hand side of the sheet to the right and
back again (travelling in the $x$ direction). The second relates
to the time taken for a wavefront to make the round trip from the bottom
to the top and back again (travelling in the $y$ direction).

\hierindex{instrument!information needed to create}
\begin{figure}[hbt]
  \begin{Label}{fig:xandyfreqs}
    \begin{center}
    \Image{xandyfreqs}{height=7cm}{gif}
    \end{center}
    \caption{Relationship between x and y frequency and an instrument's size}
  \end{Label}
\end{figure}

The precise relationship between the frequencies and dimensions of the
various instrument types is illustrated by figure \ref{fig:xandyfreqs}.
The constant \verb|Hz2CellConst| is defined by \tao\ and is used to
translate a frequency into the appropriate number of cells required
to produce that frequency of vibration.

In order to explain this more clearly figure \ref{fig:circle_waves}
shows four snapshots of a circular sheet having had a short impulse
applied at its centre. The white arrows
indicate part of the wavefront traveling across the sheet in the 
$x$ direction, reflecting back off the boundary and eventually
ending up back at the starting point again. Once the wavefront has
traveled all the way to the other side of the sheet and then back
to the centre again it has made one round trip. The speed of wave
propagation is fixed for \tao's material so the \verb|Hz2CellConst|
constant can be used to calculate how many cells it takes to produce
the correct period $T$, and hence frequency $1/T$.

\begin{figure}[htb]
  \begin{Label}{fig:circle_waves}
    \begin{center}
    \Image{circle_waves}{height=7cm}{gif}
    \end{center}
    \caption{Round trip of a wavefront in a circular sheet}
  \end{Label}
\end{figure}


The final piece of information required by all new instruments is a
decay time, the time taken for vibrations to naturally die away
when the instrument is excited in some way and then left to vibrate
freely. The value given is used to calculate a default value
for the \verb|velocityMultiplier| attribute of each cell. This initially
gives the instrument uniform damping although this uniform behaviour
can be subsequently altered by damping local regions of the instrument.
More of this technique in section [TO DO: WRITE THIS SECTION].

\section{Access Points}
\label{section:access_points}
In order for \tao\ to provide an interface between instruments and devices
another key object is needed: the \Term{access point}. Access points
allow forces to be applied to the material and also for the physical
attributes of the material to be read off at \emph{any}\/ point, not just
at the discrete set of points where cells exist. This interpolation facility
is one of the most important as it overcomes some of the limitations associated
with the material being discrete in nature. An example might be trying to
simulate a string being stopped to a particular pitch. Without the ability
to interpolate the position at which the stop is applied it might not be
possible to achieve the desired pitch precisely.

\begin{figure}[htb]
  \begin{Label}{fig:instrument_coord_system}
    \begin{center}
    \Image{instrument_coord_system}{height=8cm}{gif}
    \end{center}
    \caption{The Instrument Coordinate System}
  \end{Label}
\end{figure}

In short, all interactions between \tao's cellular material and the outside
world take place via access points. This is probably a good place to 
introduce the notion of the
\emph{instrument coordinate system}\hierindex{instrument!coordinate system}.
This coordinate system allows points in the range $(0..1, 0..1)$ to be
specified.
All instruments, regardless of their shape and size use the same 
normalised coordinate system as illustrated in figure
\ref{fig:instrument_coord_system}. Of course for some instruments
such as the circular sheet depicted some points (such as point \textbf{a})
will be invalid. If a script attempts to use such an access point
nothing will happen. If the physical attributes are read off
the instrument at such a point, they will all return values of zero.

Note also that in the case of the string only the $x$ coordinate needs
to be specified.

\section{Instrument Visualisations}
\label{instrument_visualisations}
\hierindex{instrument!visualisation facility}
\tao\ is capable of producing visualisations of the instruments constructed
when a script is invoked. We have already seen one example of this is figure
\ref{fig:circle_example}, but in that example the instrument only showed a 
single circular sheet of material. Figure \ref{fig:compound_example} shows a
\Term{compound instrument} consisting of five strings attached to a
rectangular sheet. Access points are marked by small red points on the image
and in this case they show that the left hand sides of each string are
coupled to points on the rectangular sheet.

\begin{figure}[htb]
  \begin{Label}{fig:compound_example}
    \begin{center}
    \Image{compound_example}{height=8cm}{gif}
    \end{center}
    \caption{Screenshot of a compound instrument}
  \end{Label}
\end{figure}

It is possible to translate, zoom, and rotate the image by holding down the
left, middle and right mouse buttons respectively and moving the mouse.
\tao\ automatically labels instruments and devices with their names
(the names they are given when they are created in a script) and it is
possible to toggle both types of labels on and off in cases where the
graphics window becomes too cluttered with information. This is achieved
by pressing the \textbf{I} key to toggle instrument labels on and off,
and the \textbf{D} key to toggle device labels.

\section{Parameters}
\label{section:parameters}
The term \Term{parameter}\/ is used as a generic term to refer to all
numerical variables in \tao. There are floating point and integer parameters
and the latter are declared as either being of type \verb|Integer|,
\verb|Counter| or \verb|Flag|. These types are only used to make the
intended function of a particular integer variable clear to the human reader.
As far as the system is concerned they are functionally identical. 

\section{Pitches and Frequencies}
\label{section:pitches_and_frequencies}
The Pitch\index{Pitch} object provides a generalised mechanism for expressing and
converting between various pitch and frequency formats. The formats
supported are as follows:

\begin{itemize}
\item \texttt{\emph{value} Hz} (cycles per second, analogous to Csound's cps notation);
\item \texttt{\emph{octave}.\emph{semitone}} (analogous to Csound's pch notation);
\item \texttt{\emph{octave}.\emph{decimal}} (analogous to Csound's oct notation);
\item note name notation (the pitch is represented as a character string).
\end{itemize}

In the last case the pitch names \verb|C-G| can be used followed by an
optional \verb|b| for flat or \verb|#| for sharp. The basic pitch
name of the note is then followed by an octave number (whose value
has the same meaning as the integer parts of the \verb|pch| and
\verb|oct| notations).
Finally an optional microtonal adjustment may be added in the form of a
fraction \verb|+<x>/<y>| or \verb|-<x>/<y>|
which adds/subtracts a fraction of a semitone to/from the pitch given.
Some practical examples are given below:

\begin{verbatim}
    Pitch p1 = 110.5 Hz;
    Pitch p2 = C#5+1/2;
    Pitch p3 = 8.05 pch;
    Pitch p4 = 6.1764 oct;
\end{verbatim}

Pitch methods include:

\begin{description}
\item[asPitch()]
returns a number representing the pitch converted to pch notation;
\item[asOctave()]
returns a number representing the pitch converted to oct notation;
\item[asFrequency()]
returns a number representing the pitch converted to a frequency;
\item[asName()]
returns a character string representing the name of the pitch.
\end{description}

\chapter{Getting Started}
\label{section:getting_started}
In this section we look at a typical session with \tao\ from beginning to
end. I will assume that you have successfully installed the package and
that the test script worked OK.

\section{Writing a script}
First of all open a new text file in your favourite text editor and copy
the following text into it:

\begin{verbatim}
    Audio rate: 44100;

    String string(200 Hz, 20 secs);

    Output output(stereo);

    Init:
        string.lockEnds();
        ...

    Score 20 secs:
        At 0 secs for 1 msecs:
            string(0.2).applyForce(1.0);
            ...

        output.chL: string(0.1);
        output.chR: string(0.9);
        ...
\end{verbatim}

Now save the file as ``new.tao''.

In plain English this script does the following:

\begin{enumerate}
\item
declares the audio sampling rate for any output files to be 44.1 KHz
\item
declares a \emph{string} instrument called \verb|string|
\item
declares a two channel \emph{output} device called \verb|output|
\item
initialises the string by locking both ends (fixing them at their
initial position: $z=0$)
\item
states that the \emph{performance} will last for 20 seconds
\item
applies a fixed force of magnitude 1.0 to a point on the string
one fifth of the way along its length (from the left hand side)
for a short time interval (0 seconds to 0.001 seconds)
\item
writes the movements of the string out to a stereo output file
called ``new\_output.dat'' with the left and right channels following
the movements of a point one tenth of the way along its length and
another point nine tenths of the way along the string, respectively.
\end{enumerate}

In case you are wondering, the name of an output file is formed
by appending the name of the output device onto the name of the
script (minus the \verb|.tao| extension) and adding a \verb|.dat|
extension, indicating that the file contains raw floating point data,
i.e. ``new\_output.dat''. These files require further processing before
they can be played back. See section \ref{section:output_files}.

\section{Executing the script}
To execute this script simply type the following:

\begin{verbatim}
    tao new
\end{verbatim}

\tao\ should respond with the following messages in the shell window:

\begin{verbatim}
    ========================================
    |     Tao (c) 1996-99 Mark Pearson     |
    | Sound Synthesis with Physical Models |
    ========================================

    Processing new.tao
    Making new.exe
    Executing new.exe

    Sample rate=44100 KHz
    Score duration=20 seconds                
\end{verbatim}

You should then see the \emph{instrument visualisation window}
open. It should look something like the following:

\begin{figure}[h]
  \begin{Label}{fig:new}
    \begin{center}
    \Image{new}{height=6cm}{gif}
    \end{center}
    \caption{The instrument visualisation window generated by script new.tao}
  \end{Label}
\end{figure}    

When the visualisation window opens \tao\ is initially in \Term{pause mode}.
To get it out of this mode you should press the right cursor key once.
You can pause the whole system again by pressing the left cursor key at
any point in time. If you repeatedly press the right cursor key you will
see that the animation begins to move more and more rapidly although eventually
it becomes less smooth. If you now repeatedly press the left cursor
key, the animation slows down again and becomes smoother.

The reason for this is that it is possible to change the frequency with
which the visualisation window is updated. \tao\ has a \emph{synthesis engine}
which keeps track of all the instruments and devices created in a script, and
carries out all the calculations involved in bringing them to life. It also
has a \emph{graphics engine} which is responsible for displaying the
instruments and devices in the visualisation window. Pressing the left and
right cursor keys simply changes how frequently the graphics engine 
visualises what is going on inside the synthesis engine. Displaying the
instruments and devices on every tick of the synthesis engine leads to
smooth animations but at the expense of making the whole system slow
down. Conversely, displaying the instruments less frequently gives
a bigger slice of the processing power over to the synthesis engine at the
expense of producing more jerky animations.

Now supposing you want to run a performance at maximum efficiency, without
the overhead of the graphics window. When executing a script for the
first time you can simply hit the right cursor key to set the synthesis
engine in motion, and then mimimise the graphics window. This means
that all the computational resources available will be put into the
synthesis itself, until the graphics window is restored.

\section{The output of the `tao' command}
Whenever a \tao\ script is executed a \verb|.exe| file is generated as a
by-product. If you want to re-run the script at any point, instead of typing:

\begin{verbatim}
    tao new.tao
\end{verbatim}

you can simply type:

\begin{verbatim}
    ./new.exe [-g]
\end{verbatim}

The \verb|-g| option turns the visualisation window on. If you omit
this option then \tao\ proceeds with the synthesis without opening
the visualisation window. This is the most efficient way to execute
a script.

\section{Moving the image in the visualisation window}
The image displayed in the instrument visualisation window can be
rotated, translated and zoomed using the mouse. Try holding the
three mouse buttons down one at a time and moving the mouse to get a
better idea of how to manipulate the image. For more details on
keyboard and mouse bindings refer back to section
\ref{section:key_and_mouse_bindings}.

\section{Quitting the synthesis before the performance finishes}
If you want to quit the synthesis before the performance completes
either type [Ctrl-C] in the shell window you launched \tao\ from or
press the [Esc] key whilst the instrument visualisation window is the
active window.

\section{Post-processing the output files}
\label{section:output_files}
\tao's output files are written in raw-floating point format. This means
that they are not immediately playable, but require a further
post-processing step to convert them into a more usable format. This
is achieved with the \verb|taosf| command. The general syntax of this
command is:

\begin{verbatim}
    taosf <output_filename>
\end{verbatim}

Where \verb|output_filename| is the name of the output file minus the
\verb|.dat| extension. This reads the floating point sample data stored
in the output file, normalises it to fit the maximum sample range
available, and adds an appropriate WAV header to the file describing
the number of channels and sample data format.

The reason for \tao's output files being written in floating point
format is that it is difficult, if not impossible, to know in advance
the amplitudes of the waves in the instruments, especially in cases
where some sort of feedback is present in the system. It is therefore
easier to write the output samples in a format which presents no danger
of overflows (as often happens with CSound) and then normalise these
samples once the performance is finished.


\chapter{\tao's User Interface}
\index{user interface}

This section describes \tao's user interface in detail and includes
shell commands, and mouse and key bindings for the visualization
window.

\section{Shell commands}
A number of shell commands\index{shell commands}
are provided for executing scripts and converting
the output files to WAV format. These are described below:

\renewcommand{\descriptionlabel}[1]%
	{\hspace{\labelsep}\texttt{#1}}

\begin{description}
\item[tao <script>]
Takes the name of a script as its argument (minus the \verb|.tao|
extension) and executes it, producing a file called \verb|<script>.exe|.
This shell script calls upon the services of the binary executable
\Prog{taoparse}, which is described below.

\item[taoparse <script>.tao]
Binary executable which parses a \tao\ script, translates it into a
valid C++ program, and writes the results to standard output. This is
called by the \Prog{tao} shell script.

\item[taosf <outputfile>]
Takes the name of a file produced by an Output device, minus the
\verb|.dat| extension and normalises the audio samples, adding
a WAV header and writing the results to a file called \verb|<outputfile>.wav|.
This command actually calls the \verb|tao2wav| binary executable with
the arguments \verb|tao2wav <outputfile>.dat <outputfile>.wav|.

\item[tao2wav <outputfile>.dat <outputfile>.wav]
Binary executable used to convert a file produced by an Output device
(in raw floating point format) into a WAV file.
\end{description}

In addition to these commands, whenever a script is invoked with the
\verb|tao| command a corresponding \verb|.exe| file is produced. This
is the actual executable which carries out the synthesis described in
the script. For example the command \verb|tao <file>| invokes the script
\verb|<file>.tao| and as a by-product produces the executable file
\verb|<file>.exe|. Each \verb|.exe| file has one command line option,
\verb|[-g]|. This option causes the instrument visualisation window
to be opened. If omitted the synthesis proceeds without any visualisations.

If you want to replay a synthesis described in a script you can invoke
this executable file directly rather than having to re-compile the original
text file containing the script.

\section{Mouse bindings in the visualisation window}
\index{mouse bindings}
Moving the mouse whilst holding one of the mouse buttons down in the
instrument visualisation window causes the image to be zoomed, translated
or rotated. The mouse bindings are as follows:

\renewcommand{\descriptionlabel}[1]%
	{\hspace{\labelsep}\textbf{#1}}
\begin{description}
\item[Left button] -- translate the image.
\item[Middle button] -- zoom the image.
\item[Right button] -- rotate the image.
\end{description}

\section{Key bindings in the visualisation window}
\index{key bindings}
The visualisation window also has some associated key bindings. These are
as follows:

\begin{description}
\item[Left arrow key] -- increase the frequency with which the visualisation
is updated relative to ticks of the synthesis engine. This causes the
visual animation to become slower but smoother. If updating is already
occurring on every tick of the synthesis engine then an additional
press of this key pauses the whole synthesis engine until the right arrow key
is pressed.
\item[Right arrow key] -- decrease the frequency with which the visualisation
is updated relative to ticks of the synthesis engine. This causes the
visual animation to become faster but less smooth as more computational
resources are given over to the synthesis engine itself. This key also takes
the synthesis engine out of \emph{pause mode} if it has been paused.
\item[Up arrow key] -- increase the amplitude of the waves depicted in the
visualisation.
\item[Down arrow key] -- decrease the amplitude of the waves depicted in the
visualisation.
\item[I key] -- toggle display of instrument names. This can be useful if
the visualisation becomes too cluttered with information.
\item[D key] -- toggle display of device names.
\item[ESC key] -- exit the synthesis, closing the visualisation window.
\end{description}


\chapter{\tao's Synthesis Language in Detail}
\tao's main user interface elements are its \Term{synthesis language}
and \emph{instrument visualisation window}\hierindex{instrument!visualisation window}.
The synthesis language provides
the means for describing new instruments and `playing' them
and the instrument visualisation window provides graphical animations of
the instruments showing their behaviour as the acoustic waves propagate
through them. At present there is no GUI (Graphical User Interface) for
constructing instruments but this feature is planned for a future release.

This section focusses on the synthesis language in some detail. It begins by 
describing the main conceptual parts of a \tao\ script and then goes on to
cover each individual element of the language in more detail.

\section{Overview of a Script}
A \tao\ script, although contained within one text file, is conceptually
split into three main sections: the \Term{declarations} section;
the \Term{init} section; and the \Term{score} section. The
\hierindex{script!sections}
\emph{declarations} section contains instructions for creating instruments,
devices, pitches, arrays, parameters and access points etc. The
\emph{init} section contains instructions for initialising the instruments,
devices, parameters and other objects. It may optionally contain
instructions for applying devices to instruments, coupling instruments
together, and specifying the initial locations of any access points. Finally
the \emph{score} section contains instructions for playing the instruments
and generating output audio files during the \Term{performance}. The term
performance is used here to refer to the run-time execution of the synthesis
scenario described in the script.

The score provides a means for the user to apply time-varying excitations
to the instruments and control any instrument or device attributes via
the use of parameters. The score consists of nested
\Term{control structures} which allow events to be scheduled throughout
the performance. These control structures contain either further nested
control structures or \Term{statements}. Statements are the mechanism
by which parameters are assigned values, mathematical expressions are
evaluated, instrument and device attributes are controlled etc. For a
more detailed description of the statement types available see
section \ref{section:statements}. 
Statements are also used to specify the sound samples which are to
be written out to an Output device.
 
\section{The Declarations Section}
This section of the script may contain Instrument, Device, Parameter,
Access Point and Pitch declarations, each of which is described in
the following sections. However the very first declaration which
must appear in a script determines the audio sample rate of the output
files. This declaration takes the form:

\begin{verbatim}
    Audio rate: 44100;
\end{verbatim}

In the present release the sample rate must be set to 44.1kHz as above.

\subsection{Instrument Declarations}
\label{section:instrument_declarations}
\hierindex{declaration!instrument}
\tao\ provides a set of classes for creating pieces of the material described
in section \ref{section:cellular_material}. Each class deals with a creating
a piece of material of a particular geometrical shape, so for example the
user can create strings, circular sheets, rectangular sheets and
elliptical sheets. The way in which \tao's cellular material is actually
implementated provides for future support of irregularly shaped
components but in the present version the user is limited to these
geometrical primitives.

In practice though this is not a serious limitation since there are 
many other techniques available for designing interesting instruments.
These include damping and locking parts of an instrument and constructing
\Term{compound instruments} by coupling several pieces of material together
using Connector devices. All of these techniques provide ample
room for experimentation.

In order to create a \emph{primitive} instrument
\hierindex{instrument!creating} several pieces of information are required.
These include the instrument type, the name by which it will be referred
to in the script, its $x$ and $y$ frequencies, and its decay time.

The general form of an instrument declaration is illustrated by the
following string declaration:

\begin{verbatim}
    String string(<pitch>, <decay_time>);
\end{verbatim}

where \verb|String| is the name of the instrument class; \verb|string|
is the name of the particular instrument being created;
\verb|<pitch>| defines (indirectly) how long the string will be
(the tension in \tao's material cannot be altered so the length of a string
is related to its pitch or frequency alone and vice versa); and
\verb|<decay_time>| determines the amplitude decay time of the instrument.

In the next example the placeholders \verb|<pitch>| and \verb|<decay_time>|
are replaced with typical values which might occur in a script:

\begin{verbatim}
    String string(C#5+1/2, 4.5 secs);
\end{verbatim}

In this example the length of the string is set such that its pitch
is C sharp plus a quarter-tone (1/2 a semitone) in octave 5, and
its decay time is four and a half seconds. 
The \verb|<pitch>| argument can be specified in a number of different
formats, some of which are directly analogous to those provided in
Csound. The other formats are introduced throughout this section
by way of example. The format used in the example above is referred to
as \Term{note name format} format\hierindex{pitch formats!note name}.

The \verb|<decay_time>| argument consists of a numerical constant
followed by the units of time, i.e. \verb|sec|, \verb|secs|, \verb|min|,
\verb|mins| or \verb|msecs|, representing seconds, minutes and milliseconds
respectively.

A second practical example is given below, this time creating a
rectangular sheet called \verb|rect|:

\begin{verbatim}
    Rectangle rect(200 Hz, 500 Hz, 60 secs);
\end{verbatim}

In this example two pitch arguments are specified, and both are given
in \Term{frequency format}\hierindex{pitch formats!frequency}. The first
determines the size of the instrument in the $x$ direction and
the second, the size in the $y$ direction. It may seem slightly
unintuitive at first to be specifying the size of a rectangular
sheet in units of Hertz rather than physical dimensions such
as metres or millimetres, but this practice is adopted
for a number of good reasons:

\begin{enumerate}
\item It makes creating precisely pitched instruments a simpler matter;
\item \tao's material is not based upon any real-world material so
it would be meaningless to talk about a sheet of cellular material 5m by 3.5m;
\item The instrument is described in units which are of more perceptual
relevance to a musician than physical units of size (open to debate).
\end{enumerate}

Another advantage of specifying dimensions by pitch or frequency is that it
becomes a simple matter to construct an instrument which has an array of
similar components but with different pitches for each. For example the
following code fragment creates a set of rectangular components with uniform
$y$ dimension but pitches tuned to fractions of an octave for the
$x$ dimension. This kind of instrument might be the starting point
for some sort of pitched percussion instrument for example:

\begin{verbatim}
    Rectangle rect1(8.0 oct, 500 Hz, 60 secs);
    Rectangle rect2(8.2 oct, 500 Hz, 60 secs);
    Rectangle rect3(8.4 oct, 500 Hz, 60 secs);
    Rectangle rect4(8.6 oct, 500 Hz, 60 secs);
    Rectangle rect5(8.8 oct, 500 Hz, 60 secs);
    Rectangle rect6(9.0 oct, 500 Hz, 60 secs);
\end{verbatim}

This example shows yet another form of the pitch argument, i.e. 
\Term{octave/fraction format}\hierindex{pitch formats!octave/fraction}
or \Kwd{oct} format for short. In this format the integer part specifies
the octave and the fractional part after the decimal point specifies a
fraction of an octave. 

The previous example opens the way for describing another important
technique often used when creating instruments with arrays of similar components.
\tao\ provides an \Term{array} facility for grouping
together such components and giving them a common name. For example
the following script code has much the same effect as the previous example
but logically groups the six rectangular components together into an
array with a single name \verb|rect_array|:

\begin{verbatim}
Rectangle rect_array[6]=
    {
    (8.0 oct, 500 Hz, 60 secs),
    (8.2 oct, 500 Hz, 60 secs),
    (8.4 oct, 500 Hz, 60 secs),
    (8.6 oct, 500 Hz, 60 secs),
    (8.8 oct, 500 Hz, 60 secs),
    (9.0 oct, 500 Hz, 60 secs)
    };
\end{verbatim}

The individual components can be accessed using syntax which will be very
familiar to C and C++ programmers:

\begin{verbatim}
    rect_array[0], rect_array[1] .. rect_array[5]
\end{verbatim}

The declarations for circular, elliptical and triangular sheets of material
follow a similar format to the examples presented in this section, with
elliptical and triangular sheets requiring two pitch values and circular
sheets requiring only one (determining the diameter).

Examples are given below:

\begin{verbatim}
    Circle circle(5.03 pch, 20 secs)
    Ellipse ellipse(6.00 pch, 50 Hz, 1 min + 20 secs)
    Triangle triangle(100 Hz, 6.5 oct, 600 msecs)
\end{verbatim}

These declarations introduce the final pitch notation,
\Term{octave/semitone format}\hierindex{pitch formats!octave/semitone}
or \Kwd{pch} format for short. In this format the fractional part after
the decimal point is interpreted as semitones. For example \verb|5.03 pch|
means the third semitone above C in octave 5. Note that fractions of
semitones are also possible. For example \verb|5.035 pch| means 3.5
semitones above C in octave 5.

Note also that the different pitch formats can be used side by side
in an instrument declaration requiring more than one pitch.

\subsection{Device Declarations}
\hierindex{declarations!device}\index{device declaration}
Device declarations are similar to other declarations in that they
consist of a type name followed by a comma-separated list of either
individual device names or device array names. For example, all the
following are valid device declarations:

\begin{verbatim}
    Bow bow;
    Hammer hammer;
    Connector connector;
    Stop stop;
    Output output(1);

    Bow bow1, bow2, bowArray[10];
    Hammer hammerArray1[5], hammerArray2[5];
    Output out1(stereo), out2(mono), outArray(stereo)[10];
\end{verbatim}

Note that the syntax used for Output device declarations differs
from that used for the other devices. This is because an Output declaration
must specify the number of channels for the single output or array of
outputs being created. Also note the use of keywords \verb|mono| and
\verb|stereo| as alternatives to putting the numerical constants
1 and 2 respectively.

\subsection{Access Point Declarations}
\hierindex{declaration!access point}\index{access point declaration}
Some examples of access point declarations are given below:

\begin{verbatim}
    AccessPoint a1=instr1(0.1,0.5), a2=instr2(centre);
    AccessPoint a3=instr3(left,bottom), a4;
\end{verbatim}

If an access point is to be given an initial value then the familiar
access point notation of the instrument name followed by one or two
coordinates in parentheses is used. Note that an access point does
not need to be initialised, although if you try to apply a device via
an access point before it has been initialised nothing will happen.

\subsection{Pitch Declarations}
\label{section:pitch_declarations}
We have already come across the use of \Term{pitch literals} in
section \ref{section:instrument_declarations} but it is also possible
to declare pitches as seperate objects in a script. Pitch declarations
take the following form:

\begin{verbatim}
    Pitch p1=C#7+1/2, p2=8.3 oct, p3=8.01 pch, p4=50 Hz;
\end{verbatim}

This example illustrates the four pitch formats supported, i.e.
\emph{note name}, \emph{octave/fraction}, \emph{octave/semitone} and 
\emph{frequency}.

\subsection{Parameter Declarations}
\label{section:parameter_declarations}
The term \Term{parameter} is a blanket term which is applied to both
floating point and integer variables. There are four keywords available
for denoting different types of parameters: \Type{Param}, \Type{Integer},
\Type{Counter} and \Type{Flag}. The latter three all lead to the creation
of integer variables and the only reason for having three different keywords
for the same parameter type is so that the semantic function of a 
particular integer variable is made more clear in a script.

The following are examples of valid parameter declarations:

\begin{verbatim}
    Param p1, p2, p3=10.0;
    Integer i1=10, i2=30;
    Counter count=0;
    Flag flag1=false, flag2=true;
\end{verbatim}

As with instruments and devices it is also possible to create (one-dimensional)
arrays of parameters in the following ways:

\begin{verbatim}
    Param array1[10], array2[20], array3[]={0.1,0.2,0.3,0.4,0.5};
    Integer intArray[5]={1,3,5,7,9};
    Flag flagArray[]={true, false, true, true, false};
\end{verbatim}

Note that in the case of an initialised array (one with initial values
in curly brackets) giving the size of the array is optional. However
if the size is specified then it must match up with the number of
intial values given.

\section{The Init Section}
\label{section:init_section}
The declaration section of the script allows the user to create the basic
building blocks for the synthesis but there are often many other tasks
which must be performed just prior to the `performance' described in the
score. These include:

\begin{itemize}
\item locking parts of the instruments;
\item damping parts of the instruments;
\item initialising various parameter values;
\item initialising the devices;
\item specifying the points at which the various devices will interact
with the instruments.
\end{itemize}

This is the purpose of the Init section of a \tao\ script. A typical
Init section is shown below:

\begin{verbatim}
    Init:
        string1.lockEnds();
        rectangle1.lockCorners();
        param1=0;
        paramArray[0]=10;
        flag1=true;
        ...
\end{verbatim}

\section{The Score Section}
\label{section:score_section}
\tao's score language provides the means for controlling the instruments
and devices declared and initialised in the previous two sections of the
script. Unlike the score language used by Csound \tao's script
language is not a set of time-stamped numerical data to be fed into the
inputs of the instruments as the performance progresses. Instead it is
an algorithmic language which allows user specified pieces of code to be
executed under certain conditions or at specified times.

The score language is hierarchical in nature and consists of nested
\Term{control structures} and \Term{statements}. The available
control structures are described in the next section.

\section{Control Structures}
\label{section:control_structures}
\tao's score language provide a set of constructs for scheduling events
in a performance. These are referred to as \Term{control structures}
and include the following:

\begin{verbatim}
    At <t>: <body> ...
    At <t1> for <t2>: <body> ...
    From <t1> to <t2>: <body> ...
    Before <t>: <body> ...
    After <t>: <body> ...
    ControlRate <k>: <body> ...
    Every <t>: <body> ...
\end{verbatim}

In each case \verb|<body>| is the code which is to be executed at
the scheduled time. In the case of \CtrlIndex{At} the code is executed once
at the given time \verb|<t>|. For \CtrlIndex{At..for} and \CtrlIndex{From..to} the
\verb|<body>| is executed on every tick in the interval defined by
\verb|<t1>| and \verb|<t2>|.

\CtrlIndex{Before} and \CtrlIndex{After} also define time intervals during which the
\verb|<body>| is executed on every tick but the behaviour is slightly more
context dependent as we will see later in this section. For
\CtrlIndex{ControlRate} the code is executed on every \verb|<k>|'th tick and
finally, for \CtrlIndex{Every} the code is executed repeatedly once every
\verb|<t>| seconds (other units can be specified).

In each of the control structures introduced above there is a \Term{head}
part which determines when the \Term{body} will be executed. For most of
the control structures the head consists of some sort of test to see what
the value of the system variable \verb|Time| is compared to the
given values. This variable keeps track of the amount of sound synthesised
so far -- i.e. \Term{performance-time} -- not real-time.
If the particular condition specified is met then the body of the control
structure is executed.

What follows is a more detailed look at the syntax and behaviour of
the various control structures.

\subsection{At}
This control structure takes the following form:

\begin{verbatim}
    At <t>: <body> ...
\end{verbatim}

The \verb|<body>| is executed if the value of \verb|Time| is equal to
the value \verb|<t>|.
 
\subsection{At..for}
This control structure repeatedly executes the instructions contained
in the body on every tick from time \verb|<t1>| to \verb|<t1>+<t2>|
inclusively:

\begin{verbatim}
    At <t1> for <t2>: <body> ...
\end{verbatim}

\subsection{From..to}
This control structure repeatedly executes the body on every tick from
\verb|<t1>| to \verb|<t2>| inclusively:

\begin{verbatim}
    From <t1> for <t2>: <body> ...
\end{verbatim}

\subsection{Every}
This control structure repeatedly executes the body every \verb|<t>|
seconds starting at 0.0 seconds:

\begin{verbatim}
    Every <t>: <body> ...
\end{verbatim}

\subsection{ControlRate}
This control structure repeatedly executes the body once every
\verb|<k>| ticks starting at tick zero:

\begin{verbatim}
    ControlRate <k>: <body> ...
\end{verbatim}

\section{Conditional and looping control structures}
Looping and conditional control structures are also provided in the form
of \CtrlIndex{For..to}, \CtrlIndex{If}, \CtrlIndex{If..Else},
\CtrlIndex{If..ElseIf..Else}. These are described in more detail in the
next section.

\subsection{If, If..Else, If..ElseIf..Else}
Unlike the control structures introduced above, the next three are not
concerned with the \Var{Time} variable, but allow the user to specify
conditional execution by providing Boolean expressions. They do not require
much explanation really so all that is included here is the syntax:

\begin{verbatim}
    If <expr>:
        <body>
        ...
\end{verbatim}

\begin{verbatim}
    If <expr>
        <body1>
        ...
    Else
        <body2>
        ...
\end{verbatim}

\begin{verbatim}
    If <expr1>
        <body1>
        ...
    ElseIf <expr2>
        <body2>
        ...
    ElseIf <expr3>
        <body3>
        ...
    .
    .
    Else
        <default_body>
        ...
\end{verbatim}

Note that the block terminating symbol \verb|...| must appear after
the end of every \verb|<body>| section. Also note that conditional
expressions do not need to be surrounded by brackets as they do in C
and C++.

\subsection{For Loops}
The \verb|For| control structure provides a simple mechanism for iteration
with integer variables and takes the form:

\begin{verbatim}
    For <parameter>=<initial> to <final>:
        <body>
        ...
\end{verbatim}

The \verb|<parameter>| must be an integer parameter, i.e. declared
as one of the following:

\begin{verbatim}
    Integer <parameter>;
    Counter <parameter>;
    Flag <parameter>;
\end{verbatim}

There is no difference between these three integer parameter types
incidentally, they are only included to make it clear to a human
reader of a \tao\ script whether a particular integer variable is
a counter a Boolean flag etc. It therefore makes little stylistic sense
to use a \verb|Flag| as the loop variable in a \Ctrl{For} statement or a
\Type{Counter} in an \Ctrl{If} statement. 

\section{Statements}
\label{section:statements}
Whilst control structures allow various events to be scheduled throughout
a performance \Term{statements} provide the actual means by which individual
events are described. Examples events might include striking an instrument,
assigning a parameter a new value, changing a device's attributes etc. 
The individual statement types supported are descibed in the following
sections.

\subsection{Assignment Statement}
\label{section:assignment_statement}
Assignment statements are used to assign values to parameters much as
in any standard programming language such as C or C++. The operators
used are inherited from these languages (\verb|=, +=, -=, *=, /=|).
An assignment statement takes the following general form:

\begin{verbatim}
    <parameter> = <expression>
    <parameter> += <expression>
    <parameter> -= <expression>
    <parameter> *= <expression>
    <parameter> /= <expression>
\end{verbatim}

The last four assignment operators alter the parameter's current
value by adding/subtracting to/from or multiplying/dividing by the value
of the \verb|<expression>| on the right of the operator and then reassigning
the parameter with the result of the calculation. For more about expressions
see section \ref{section:expressions}.

\subsection{Print Statement}
\label{section:print_statement}
The \Statement{Print} statement is used to output text and parameter values to
the shell window from which \tao\ is invoked. This is useful for
getting ongoing feedback about how a score is proceeding. The following
script fragment illustrates its use:

\begin{verbatim}
    .
    .
    Parameter p1,p2,p3;
    Init:
        p1=10;
        p2=20;
        p3=30;
        ...
    
    Score 1 secs:
        At 0 secs:
            Print "p1=", p1, "p2=", p2, "p3=", p3, newline;
            ...
        Every 0.2 secs:
            Print "Elapsed Time=", Time, newline;
            ...
        ...
\end{verbatim}

In this example we see four different types of item being `printed'.
The first is a character string, no surprises there, the second is
a user defined parameter, the third is a system variable \verb|Time|,
and the fourth is the special item \verb|newline| which causes printing
to continue on the next line.

\subsection{For Statement}
\label{section:for_statement}
The For statement provides a rudimentary looping and iteration
facility. It is fairly basic as it only supports integer counting
from some initial value to some final value in steps of 1. An example
is given below:

\begin{verbatim}
    String stringArray[4]=
        {
        (50 Hz, 60 secs), 
        (100 Hz, 60 secs), 
        (150 Hz, 60 secs), 
        (200 Hz, 60 secs)
        };
     
    Counter c;
     Init:
        For c=0 to 3:
            stringArray[c].lockEnds();
            ...
        ... 
\end{verbatim}

\subsection{Label Statement}
\label{section:label_statement}
The Label statement allows the user to display text captions in the instrument
visualisation window. These \Term{labels} can be anchored to points on instruments
so that they move as the instrument moves. This is sometimes useful to clarify
visually precisely what is going on in the script. Label statements take
the following generic form:

\begin{verbatim}
    Label (<instrument>, <x>, <xOffset>, <yOffset>,
           <caption>, <red>, <green>, <blue>);
\end{verbatim}

This statement is rather ugly in its present form because of the large
number of arguments but it does the job on the few occasions the user
really needs to add to the automatic labeling produced by \tao.

The example below is a Label statement with real arguments:

\begin{verbatim}
    Label (s[4], 1.0, 0.0, 0.0, "String four", 1.0, 1.0, 1.0);
\end{verbatim}

\subsection{Method Statement}
\label{section:method_statement}
There are a number of object methods for each class which return
no value but are usually associated with setting the attributes of
an object such as an instrument or device. These methods are not
listed here since they are covered in detail in section
\ref{section:object_method_reference}.

Briefly though examples of the kind of things which we might do with
object methods include setting the amount of damping for a region of
an instrument; setting the height from which a hammer should be
dropped; and setting the velocity and downward force for a bow.
The example below shows how these tasks would be implemented in
practice in a script:

\begin{verbatim}
    Rectangle rect(100 Hz, 200 Hz, 20 secs);
    Bow bow;
    Hammer hammer;
    
    Init:
        // Damp the bottom left hand corner of the rectangular
        // sheet with a value of 0.5
        rect.setDamping(left, 0.1, bottom, 0.2, 0.5);

        // Set the drop height of the hammer
        hammer.setHeight(10.0).reset();

        // Set the initial velocity and downward force of the
        // bow
        bow.setVelocity(0.0).setForce(1.5);
        ...
\end{verbatim}

\subsection{Connection Statement}
\label{section:connection_statement}
The connection statement is the means by which the end points of a
Connector object are specified in a script. Each end of a 
Connector object can be assigned either an access point or 
a numerical value representing a fixed \Term{anchor point}. The following
script fragment illustrates the use of the connection statement in its
various forms:

\begin{verbatim}
    String string1(100 Hz, 20 secs);
    String string2(100 Hz, 20 secs);
    String string3(100 Hz, 20 secs);
    
    Connector conn1, conn2, conn3;
    
    Init:
        string1(0.5) -- conn1 -- string2(0.5);
        string2(0.1) -- conn2 -- 0.0;
        string2(right) -- conn3 -- string3(left) strength 0.5;
        ...
\end{verbatim}

There are several features about the connection statement to note. Firstly
it is possible for both ends of a Connector to be assigned
access points, for one end to be assigned an access point whilst the other 
is assigned an anchor point. However it is meaningless for both ends of a 
Connector to be assigned anchor points since this would have
no effect on any instruments and the Connector would thus be
rendered useless.

The second thing to note is that regardless of whether access or anchor
points are used the connection statement has an optional \verb|strength|
clause which allows the strength of the spring to be set. If the strength
value is specified in the range [0..1] then the model's behaviour is
guaranteed to remain stable. However some values higher than 1 may be
useful at times but can also make the whole instrument model unstable
to the point where it induces exponentially growing noisy vibrations.
Unfortunately this is an limitation inherent in the discrete model used
by \tao.

\subsection{Output Statement}
\label{section:output_statement}
The \Term{output statement} is used to feed floating point samples to
an Output device. The samples can be generated by arbitrary
mathematical expressions but are usually derived from expressions
involving access points.

\begin{verbatim}
    Output out1(stereo), out2(mono);
    .
    .
    Score 10 secs:
        out1.chL: string1(0.1);
        out1.chR: string1(0.9);
    
        out2.ch1: string1(0.5);
        ...
\end{verbatim}

As the previous example shows, the output statement consists of the
name of an Output object followed by one of the methods \verb|ch1|,
\verb|ch2|, \verb|chL|, \verb|chR|. This is then followed by
a colon and then the expression representing the floating point value
to be written out to the Output's associated file as an audio sample.

Note that when an access point expression appeears in an output
statement, e.g. \verb|string(0.1)|, it evaluates to a floating point
value representing the displacement of the instrument along the 
$z$ axis at that point. It is as if the expression
\verb|string(0.1).getPosition()| had been typed. This short-hand
notation makes it easier to read where the output samples are
coming from. Of course if you want to use the velocity of the
string at that point you could write instead:

\begin{verbatim}
    .
    .
        out1.chL: string(0.1).getVelocity();
        out1.chR: string(0.9).getVelocity();
\end{verbatim}

For a more detailed description of \tao's expression syntax see
section \ref{section:expressions}.

\subsection{Join Statement}
\label{section:join_statement}
The \Term{join statement} provides another means for coupling together
two components. It only works for rectangular components and does so
by lining up to adjacent edges and `stitching' them together with new
springs. After having been joined in this way, two rectangular components
will behave as a single, continuous piece of material. In figure
\ref{fig:joining} two rectabgular components are shown being joined 
together. Two Join statements are shown underneath the instruments.
Either one of the Join statements could be used in a script to lead to
the same end result.

\begin{figure}[h]
  \begin{Label}{fig:joining}
    \begin{center}
    \Image{joining}{height=8cm}{gif}
    \end{center}
    \caption{Joining two rectangular components together with the
	Join statement}
  \end{Label}
\end{figure}

\subsection{Apply Statement}
\label{section:apply_statement}
The \Term{apply statement} provides the means for the user to specify the
access point via which a device will interact with an instrument. It is
similar in syntax to the \Term{connection statement} (section
\ref{section:connection_statement}) in that it makes use of the 
\verb|--| operator as shown in the next example:

\begin{verbatim}
    String string(C#5+1/2, 55 secs);
    Bow bow;
    Parameter bowPosition=0.5;
    .
    .
    
    string(bowPosition) -- bow;
\end{verbatim}

If you wish to disengage the device from the specified access point you
use the device method \verb|remove()|. So for example:

\begin{verbatim}
    bow.remove();
\end{verbatim}

\section{Describing Musical Events}
Having introduced the various elements which comprise \tao's synthesis
language we now take a look at how to describe musical \Term{events}
using the score language provided. The term
\emph{event} needs some clarification before we start though. In \tao\
the term is used to signify \emph{anything}
which occurs during a performance, either at a particular instant in time
or over some time interval. Events come in all shapes and sizes from
low-level events such as setting the value of a parameter, to high-level
events such as playing a bowed note on a stringed instrument. 

Many events, especially the higher level musical events are hierarchical
in nature. For example in order to describe an event such as bowing a
note on a string the event will be broken down into sub-events such as
the following:

\begin{itemize}
\item
Apply the bow to the string;
\item
Increase the velocity of the bow over some short time interval to
create an attack, at the same time as controlling the downward
force of the bow;
\item
Hold the velocity steady for some time interval;
\item
Decrease the velocity steadily for some time interval;
\item
Remove the bow from the instrument.
\end{itemize}

Such high-level events are referred to as \Term{compound events}.
All compound events no matter how complex eventually reduce down to
low-level events, examples of which are given below:

\begin{itemize}
\item
Evaluating expressions and assigning values to parameters;
\item
Changing the attributes of an instrument;
\item
Changing the attributes of a device;
\item
Applying a device to an instrument or removing it again;
\item
Coupling instruments together;
\item
Displaying text output in the shell window to give feedback about how
a performance is progressing;
\item
Specifying output sources;
\item
Writing audio samples to output files.
\end{itemize}

The rest of this section takes a closer look at the various techniques
which are commonly used to implement compound events. 

\subsection{Nested control structures and the special variables start and end}
The way in which compound events are describedin a score is by nesting
control structures representing low-level events within higher level ones.
The following (trivial) script illustrates this technique, at the same time as
introducing two special variables called \verb|start| and \verb|end|, which
play a central role in describing compound events:

\begin{verbatim}
    Audio rate: 44100;
    
    Init:
        ...
    
    Score 2 secs:
        From 0 secs to 1 secs:
            At start:
                Print "For interval 0-1 seconds start=", Time, newline;
                ...
    
            At end:
                Print "For interval 0-1 seconds end=", Time, newline;
                ...
            ...
    
        From 1 secs to 2 secs:
            At start:
                Print "For interval 1-2 seconds start=", Time, newline;
                ...
    
            At end:
                Print "For interval 1-2 seconds end=", Time, newline;
                ...
            ...
        ...
\end{verbatim}

When invoked this script produces the following output:

\begin{verbatim}
    Sample rate=44100 Hz
    Score duration=2 seconds
    For interval 0-1 seconds start=0
    For interval 0-1 seconds end=1
    For interval 1-2 seconds start=1
    For interval 1-2 seconds end=2
\end{verbatim}

The four \Kwd{Print} statements in this example print out the values of
the \VarIndex{start} and \VarIndex{end} variables at various points during the
performance. Note that the values change depending on where the variables
are actually accessed. This is due to the concept of \Term{scope}. Each
control structure which defines a time interval during the performance
-- i.e. each instance of \Ctrl{At..for}, \Ctrl{From..to}, \Ctrl{Before}
or \Ctrl{After} -- has its own scope. Within that scope the values of
\Var{start} and \Var{end} are set to refer to the start and end times
of that particular time interval. This is useful for defining sub-events
in terms of the higher-level event in which they are enclosed.

Another example is given below to clarify this point:

\begin{verbatim}
    Audio rate: 44100;
    
    Init:
        ...
    
    Score 5 secs:
        At start:
            Print "For score, start=", Time, newline;
            ...
        At end:
            Print "For score, end=", Time, newline;
            ...

        From 1 secs to 4 secs:
            At start:
                Print "For interval 1-4 seconds, start=", Time, newline;
                ...
            At end:
                Print "For interval 1-4 seconds, end=", Time, newline;
                ...
            ...
        ...
\end{verbatim}

In this example the first pair of \Ctrl{At} structures are enclosed within the scope
of the top-level \Ctrl{Score} structure, whilst the second pair of \Ctrl{At}
structures are enclosed or nested within the \Ctrl{From..to} structure. As you can
see from the output from this script, the values of \Var{start} and \Var{end}
are altered accordingly depending on their scope:

\begin{verbatim}
    Sample rate=44100 KHz
    Score duration=5 seconds
    For score, start=0
    For interval 1-4 seconds, start=1
    For interval 1-4 seconds, end=4
    For score, end=5
\end{verbatim}

Note also that it doesn't matter in which textual order events are given in
a score, the only thing which matters is the instant in time, or time interval
defined by the values in the head of the control structure.

The ability to nest events and define the start and end times of a
sub-event in relative rather than absolute terms provides a rudimentary
mechanism for describing compound events \footnote{This scheme is far from
perfect since there is no \Term{encapsulation} facility as yet.
By encapsulation we refer to the ability of most general purpose
programming languages to break a problem down into manageable modules
(e.g. functions or procedures) which can be named and parameterised.
This feature would greatly enhance \tao's ability to deal with complex
musical events. See section \ref{section:script_deficiencies} for a more
in-depth discussion on this topic.}.   

\subsection{Streams of Events and Iteration}
Often it is necessary to repeat some simple event iteratively in order to
form a stream of similar events. One example of this might be repeatedly
striking an object at short (random) intervals in order to create a
dense granular texture. This section describes a common technique
for implementing such streams of events \footnote{Acknowledgements are due
to Prof. David Worrall of the Australian Centre for the Arts and
Technology for many fruitful discussions on the subject of events and
streams. At the time these were aimed at extending David's algorithmic
composition software \emph{Streamer}, but the discussions were of much
wider interest as events and streams are so fundamental to music.}.

In order to describe an iterated event we will use the technique of
nested control structures described in the previous section, but in
a particular way, which allows an
event to reschedule itself once its time is up. The following example
script schedules a series of events to occur at one second intervals.
Each individual event is trivial in nature, simply printing a message
to the shell window showing the time at which it occurs (performance-time,
not real-time).
 
\begin{verbatim}
    Audio rate: 44100;

    Param eventStart=0.0, eventDur=0.01, interval=1.0;

    Init:
        ...

    Score 10 secs:
        At eventStart for eventDur:
            At start:
                Print "Time=", Time, newline;
                ...
            At end:
                eventStart += interval;
                ...
            ...
        ...
\end{verbatim}

The first thing to note about this script is that it contains a
hierarchy of nested control structures. The outermost \verb|Score|
control structure contains a single \verb|At..for| structure, which in
turn contains two further \verb|At| structures. The rest of the script
is quite straightforward to understand. The parameters \verb|eventStart|
and \verb|eventDur| are used to define the start time and duration of
each event and the parameter \verb|interval| is used to define the
interval between successive events. The key element is the use of the
\verb|At end:| control structure. Every time the event occurs
the body of the \verb|At end:| structure is executed just once at the
very end of the event, and when it is a new start time is calculated
for the next event.

The script produces the following output:

\begin{verbatim}
    Sample rate=44100 KHz
    Score duration=10 seconds
    Time=0
    Time=1
    Time=2
    Time=3
    Time=4
    Time=5
    Time=6
    Time=7
    Time=8
    Time=9
    Time=10
\end{verbatim}

Of course the time interval between events does not have to be fixed.
The value by which the \verb|eventStart| parameter is incremented
can be derived from an arbitrary mathematical expression (see
section \ref{section:expressions} for details of expression syntax).
Since expressions can include numerical values derived from physical
attributes read off the various instruments and devices, this technique
opens the way for quite complex self-evolving events to be described.
This is one of \tao's strengths: any physical attribute, such as the
velocity of a point on an instrument or the current height of a hammer
device can potentially be used as input to an algorithm which is playing
the very same instruments and devices.

\subsection{Comparison with Csound}
You may ask why iteration has to be implemented by events rescheduling
themselves, rather than by being able to pre-compose a series of even
part of the answer is that it just evolved in this way. One specific
reason though relates to my own personal interest in describing complex
musical events which are self-evolving and depend upon lots of factors
including feedback from the various physical objects in the synthesis,
i.e. the instruments and devices.

If you want to have stricter control over precomposing events
then the best approach is to put all the start times, durations etc.
into arrays and then set up iterated events which step through the
arrays reading the appropriate values out for each successive event
in a stream. 

[TO DO: Write more on this subject and provide examples]

\section{Expressions}
\label{section:expressions}
This section describes \tao's expression syntax. If you have
experience of a programming language such as C or C++ then there
should be no surprises here.

\subsection{Operators}
\index{operators}
The following is a list of operators which are understood by \tao:

\begin{itemize}
\item
Arithmetic operators:
\begin{verbatim}
    +     addition
    -     subtraction
    *     multiplication
    /     division
    %     modulus
\end{verbatim}
\begin{iftex}
\Operator{\~}
\Operator{<<}
\Operator{>>}
\Operator{\&}
\Operator{\^}
\Operator{"|}
\end{iftex}

\item
Bitwise operators:
\begin{verbatim}
    ~     not
    <<    shift left
    >>    shift right
    &     bitwise AND
    ^     bitwise XOR
    |     bitwise OR
\end{verbatim}
\begin{iftex}
\Operator{\~}
\Operator{<<}
\Operator{>>}
\Operator{\&}
\Operator{\^}
\Operator{"|}
\end{iftex}

\item
Relational operators:
\begin{verbatim}
    ==    equal
    !=    not equal
    <     less than
    >     greater than
    <=    less than or equal to 
    >=    greater than or equal to
\end{verbatim}
\begin{iftex}
\Operator{==}
\Operator{!=}
\Operator{<}
\Operator{>}
\Operator{<=}
\Operator{>=}
\end{iftex}

\item 
Assignment operators:
\begin{verbatim}
    += -= *= %= <<= >>= &= ^=
\end{verbatim}
\begin{iftex}
\Operator{+=}
\Operator{-=}
\Operator{*=}
\Operator{\%=}
\Operator{<<=}
\Operator{>>=}
\Operator{\&=}
\Operator{"|=}
\end{iftex}

\item
Logical operators:
\begin{verbatim}
    and or not
\end{verbatim}
\Operator{and}
\Operator{or}
\Operator{not}

Note that the \verb|and| operator has higher precedence than the \verb|or|
operator. This means that the expression \verb|a and b or c and d|
evaluates to \verb|(a and b) or (c and d)|.
\end{itemize}

\subsection{Mathematical Functions}
\label{section:math_functions}
\index{math functions}
Mathematical functions available from within \tao's synthesis language
are inhereted directly from the gnu C++ math library. These include:
\MathFunction{acos}
\MathFunction{acosh}
\MathFunction{asin}
\MathFunction{atan}
\MathFunction{atanh}
\MathFunction{atan2}
\MathFunction{cbrt}
\MathFunction{cos}
\MathFunction{cosh}
\MathFunction{drem}
\MathFunction{exp}
\MathFunction{fabs}
\MathFunction{ceil}
\MathFunction{floor}
\MathFunction{hypot}
\MathFunction{log}
\MathFunction{log10}
\MathFunction{log1p}
\MathFunction{pow}
\MathFunction{rint}
\MathFunction{sin}
\MathFunction{sinh}
\MathFunction{sqrt}
\MathFunction{tan}
\MathFunction{tanh}

\begin{verbatim}
    acos(x)     arc cosine.
    acosh(x)    inverse hyperbolic cosine.
    asin(x)     arc sine.
    atan(x)     arc tangent.
    atanh(x)    inverse hyperbolic tangent.
    atan2(x,y)  arc tangent of two variables.
    cbrt(x)     cube root.
    cos(x)      cosine.
    cosh(x)     hyperbolic cosine.
    drem(x,y)   floating point remainder.
    exp(x)      exponential.
    fabs(x)     absolute value of floating point number.
    ceil(x)     smallest integral number not less than x.
    floor(x)    largest integral number not greater than x.
    hypot(x,y)  Euclidean distance function.
    log(x)      natural logarithm.
    log10(x)    base-10 logarithm.
    log1p(x)    logarithm of 1+x.
    pow(x,y)    value of x raised to the power of y.
    rint(x)     round to closest integer.
    sin(x)      sine.
    sinh(x)     hyperbolic sine.
    sqrt(x)     square root.
    tan(x)      tangent.
    tanh(x)     hyperbolic tangent.
\end{verbatim}

The set of functions which are available is currently governed by what
\tao's script parser has been told to expect. If \tao\ is ported
to an OS other than Linux in the future this strategy will have to
be rethought, since different math libraries often vary in the names
and availability of functions.

In addition to the standard math library functions two random number
functions are also provided. These are \verb|randomi(x,y)|
\MathFunction{randomi},
which returns a random integer in the range \verb|[x..y]|; and \verb|randomf(x,y)|
\MathFunction{randomf},
which returns a random floating point number in the range \verb|[x..y]| inclusive.

\subsection{The time-varying functions \emph{linear} and \emph{expon}}
\label{section:linear_and_expon}
Two simple time varying functions are provided for use in a score:

\begin{verbatim}
    linear(<initial>,<final>)
    expon(<initial>,<final>)
\end{verbatim}

The time interval over which they change is determined by the scope
in which they appear in the score. In other words they take their
start and end times from the \Var{start} and \Var{end} variables.
An example of their use is given in the following script:

\begin{verbatim}
    Audio rate: 44100;

    Init: ...

    Score 1 sec:
        Every 0.1 secs:
            Print "At time ", Time,
		" linear value=", linear(0,1),
		" expon value=", expon(0.001,1), newline;
            ...
        ...
\end{verbatim}

When invoked this script produces the following output:

\begin{verbatim}
    At time 0 linear value=0 expon value=0.001
    At time 0.1 linear value=0.1 expon value=0.00199526
    At time 0.2 linear value=0.2 expon value=0.00398107
    At time 0.3 linear value=0.3 expon value=0.00794328
    At time 0.4 linear value=0.4 expon value=0.0158489
    At time 0.5 linear value=0.5 expon value=0.0316228
    At time 0.6 linear value=0.6 expon value=0.0630957
    At time 0.7 linear value=0.7 expon value=0.125893
    At time 0.8 linear value=0.8 expon value=0.251189
    At time 0.9 linear value=0.9 expon value=0.501187
    At time 1 linear value=1 expon value=1
\end{verbatim}

It should be noted that in the current version of \tao\ the tools
provided for generating time varying functions are somewhat lacking when
compared to Csound and its plethora of function table generators and its
ability to create multi-segment linear or exponential curves. This
deficiency will be addressed in a future version, probably with the
introduction of a completely new set of table-based objects for use within
a script.

\section{Compiling and Executing a \tao\ Script}
A \tao\ script is executed using the \Prog{tao} command, which takes
as its only argument the name of the script with a \verb|.tao|
suffix. Although from the user's point of view the language seems
to be interpreted, since this one command interprets \emph{and} executes
the script, in reality a \tao\ script is compiled into an executable
file, which is then invoked automatically.  

The executable produced is stored in a file with the same name as the
original script but with a \verb|.exe| suffix. Once a script has been
compiled with the \Prog{tao} command, it can be executed several times
without having to recompiled, provided the script isn't altered in the
meantime. A \verb|.exe| executable has a number of command line options
which are described below:

\begin{description}
\item[\texttt{-g}] Enables the 
\emph{instrument visualisation window}
\hierindex{instrument!visualisation window}. if this option is
omitted the synthesis will proceed without any graphics at all. This is
useful for background batch processing of \tao\ scripts.
\end{description}







\chapter{Object Method Reference}
\label{section:object_method_reference}
\index{methods}
What follows in this section is a detailed description of all the
methods which are provided by the various object classes. The syntax
of each method is given, together with a description of its arguments,
purpose and function. The vast majority of methods occur with the
following generic syntax:

\begin{verbatim}
    object.method(<arg1>,<arg2> .. <argn>);
\end{verbatim}

where \verb|object| would be the name of an actual instrument, device or
other object and \verb|method| the name of an actual method.

However some of the most commonly accessed methods such as generating
an access point from an instrument name and a pair of coordinates have
a simpler syntax where the arguments are placed in parentheses immediately
after the object name, i.e. there is no method name as such. For example:

\begin{verbatim}
    instrument(<x>,<y>)
\end{verbatim}

In the following reference to all the object methods such methods
are described merely by the arguments which are placed in between
the brackets, e.g.

\begin{verbatim}
    (<x>,<y>)
\end{verbatim}

\section{Instrument Methods grouped by Function}
\hierindex{instrument!methods}
All instrument methods are listed here grouped together by function.

\subsection{Locking Parts of an Instrument}
\index{locking}
A number of instrument methods are available for locking parts of an
instrument. These include:

\begin{verbatim}
    lock(<x>,<y>)
    lockLeft()
    lockRight()
    lockTop()
    lockBottom()
    lockEnds()
    lockCorners()
    lockPerimeter()
\end{verbatim}

The \MethodIndex{lock} method locks the single cell which is nearest
to the \verb|(<x>,<y>)| position specified in instrument coordinates
(see section \ref{section:access_points}).

The \MethodIndex{lockLeft}, \MethodIndex{lockRight}, \MethodIndex{lockTop}
and \MethodIndex{lockBottom} methods each lock the cells at one extremity
of an instrument. For rectangular sheets the behaviour is obvious, but
for other instruments some clarification is needed. For strings, only the
\Method{lockLeft} and \Method{lockright} methods are appropriate. For
circular and elliptical sheets only a few cells at the edges of the sheet
will be locked by each method.

The \Method{lockEnds} method is equivalent to issuing a \Method{lockLeft}
and \Method{lockRight} together and is used mostly with strings.

Finally the \MethodIndex{lockCorners} is only appropriate for rectangular
sheets and the \MethodIndex{lockPerimeter} is appropriate for all
instruments except strings.

\subsection{Damping Parts of an Instrument}
\index{damping methods}
\hierindex{methods!damping}
A number of methods are provided for damping parts of an instrument.
These include:

\begin{verbatim}
    setDamping(<d>)
    setDamping(<x>,<d>)
    setDamping(<x>,<y>,<d>)
    setDamping(<x1>,<x2>,<y1>,<y2>,<d>)
    resetDamping()
    resetDamping(<x>)
    resetDamping(<x>,<y>)
    resetDamping(<x1>,<x2>,<y1>,<y2>)
\end{verbatim}

In each case the argument \verb|<d>| is a floating point value in
the range [0..1], where 0 represents no damping at all and 1 means that
the portion of material affected will be locked rigidly in a fixed
position. The progression from the former state to the latter as
\verb|<d>| changes from 0 to 1 is logarithmic rather than linear
for reasons which are explained below.

The damping value \verb|<d>| is converted via the following formula
into the appropriate \Attr{velocityMultiplier} attribute (see section
\ref{section:cell_attributes}):

\[ v_{m} = 1 - \frac{10000^{d}}{10000} \]

where $v_{m}$ and $d$ correspond to \Attr{velocityMultiplier} and
\verb|<d>| respectively.

You may remember that the \Attr{velocityMultiplier} value also lies in the
range [0..1] and on each tick of the synthesis engine the velocity of
each cell is multiplied by this value.

The \MethodIndex{resetDamping} family of methods set the damping back to what
it was when the instrument was created. This is useful for situations
where it is desirable to temporarily damp a region. An example application
might be playing a harmonic on a string. Any guitarist will know that
in order to do so a finger is placed momentarily in contact with one of
the strings, over a node, whilst (or after) the string is plucked. Once
the harmonic begins to clearly ring out the players finger is removed
again leaving the string to continue vibrating in its modified pattern.

\subsection{Graphically Placing Instruments}
\label{graphical_placement}
\hierindex{instrument!graphical placement}
\hierindex{methods!graphical placement of instruments}
These methods allow the user to override \tao's default graphical placement
scheme. The default scheme is not very intelligent in the current version
and simply places each new instrument above the previous one (`above'
meaning in the +ve $y$ direction). The \MethodIndex{placeAt} method
allows the position of the bottom left hand corner of the bounding
box surrounding the instrument to be set explicitly. The 
\MethodIndex{placeAbove}, \MethodIndex{placeBelow}, \MethodIndex{placeRightOf}
and \MethodIndex{placeLeftOf} methods allow an instrument to be 
placed relative to another instrument.

In the case of the methods which expect an additional argument
\verb|<offset>|, this argument specifies an additional offset
measured in world coordinates from the reference instrument (the default
is to seperate each instrument by 5 units in world coordinates).
This is somtimes necessary to prevent the instrument visualisation
window from becoming too cluttered.

\begin{verbatim}
    placeAt(<x, y>)
    placeAbove(<instrument>)
    placeBelow(<instrument>)
    placeRightOf(<instrument>)
    placeLeftOf(<instrument>)
    placeAbove(<instrument>, <offset>)
    placeBelow(<instrument>, <offset>)
    placeRightOf(<instrument>, <offset>)
    placeLeftOf(<instrument>, <offset>)
\end{verbatim}

\subsection{Accessing the Internal Attributes of an Instrument}
\hierindex{instrument!attributes!accessing}
\hierindex{methods!instrument attribute}
This set of methods allow the internal attributes of an instrument
to be inspected. Please note that some of these methods are only listed
for completeness. In practice they are not of much use in the
average \tao\ script.

\begin{verbatim}
    getName()
    getMagnification()
    getWorldX()
    getWorldY()
    getXMax()
    getYMax()
    getXFrequency()
    getYFrequency()
\end{verbatim}

The \MethodIndex{getName} method returns a string containing the name
of the instrument. The \MethodIndex{getMagnification} method returns
the current factor by which the visual amplitude of the waves in the
instrument are being magnified. The two methods \MethodIndex{getWorldX}
and \MethodIndex{getWorldY} return the world coordinates of the bottom
left hand corner of the bounding box around the instrument (i.e. the
position in the $xy$ plane). The \MethodIndex{getXMax} and
\MethodIndex{getXMax} methods return the $N-1$ where $N$ is the
width or height of the instrument in cells respectively.

The only two methods which should be of any use in the average \tao\
script are \MethodIndex{getXFrequency} and \MethodIndex{getYFrequency}.
These return the pitch values which were passed in when the instrument
was created, but converted to Hertz, regardless of the initial pitch
format used.

\subsection{Setting the Internal Attributes of an Instrument}
\hierindex{instrument!attributes!setting}
The only instrument attribute which can be set by a user is the
factor by which the amplitude of vibrations in the component are
exaggerated in the visualisation window. This is useful
for evening out differences between components within an instrument,
for the purposes of visualisation only. This attribute has no effect
on sound output.

\begin{verbatim}
    setMagnification(<m>)
\end{verbatim}

\subsection{Automatically Generating Access Points}
\hierindex{access points!generating}
\hierindex{methods!access point generation}
Each of the following set of methods generates a single access point
on an instrument given $x$ and $y$ coordinates. The \MethodIndex{point(...)}
methods only differ from the other two in that they do not lead to the
automatic generation of graphical markers in the instrument
visualisation window.

\begin{verbatim}
    (<x>, <y>)
    (<x>)
    point(<x>, <y>)
    point(<x>)

    e.g.

    rect(left,0.5)
    string(0.7)
    ellipse.point(0.1,centre)
\end{verbatim}

\subsection{Accessing Individual Cells}
\hierindex{methods!accessing cells}
The following method allows access to the nearest cell to the
position on the instrument specified. Once the cell has been
selected its internal attributes can be examined. You should
never really need to use this method as it was really designed
for \tao's internal use. Also access points provide a much
more flexible mechanism for interacting with instruments.

\begin{verbatim}
    at(<x>, <y>)
\end{verbatim}

\section{Device Methods}
\hierindex{Device methods}
\hierindex{methods!Device}
This section describes all the methods available to the different
devices. It begins by listing generic methods which are applicable
to any device.

\subsection{Generic Device Methods}
The following methods are available with any device.

\begin{verbatim}
    getName()
    getX()
    getY()
    apply(<accessPoint>)
    remove()
\end{verbatim}

The \MethodIndex{getName} method returns a string containing the
name of the device. The \MethodIndex{getX} and \MethodIndex{getY}
methods return the current position in instrument coordinates of
the device if it has actually been applied to an instrument.
Otherwise they return zero. The return values are also zero
if the device has been applied to an instrument and removed
again.

\subsection{Bow Methods}
\index{Bow methods}
\hierindex{methods!Bow}
The main attributes of a Bow device are the force with which
it is applied to the instrument (which has a marked effect on the bow's
ability to sustain the frictional forces needed to move the instrument),
and its velocity. The following methods are available with a bow.

\begin{verbatim}
    setForce(<force>)
    setVelocity(<velocity>)
    getForce()
    getVelocity()
    (<accessPoint>)
    (<instr>, <x>)
    (<instr>, <x>, <y>)
\end{verbatim}

The last three methods provide three different ways to apply
a bow to an instrument. The first specifies an access point, the second
an instrument and a single $x$ coordinate (for a string), and 
the third an instrument and both $x$ \& $y$ coordinates.

Although these methods are available in a script an alternative
syntax is usually used for applying a bow to an instrument at a specific
point. This consists of the access point specification (the instrument
name followed by the instrument coordinates enclosed in parentheses)
followed by the \emph{apply} operator: \verb|--|, followed by the
name of the bow. For example in the following code fragments the left
and right hand sides are exactly equivalent:

\begin{verbatim}
    Bow bow;
    
    Init:
        bow(string(0.1));    <==>    string(0.1) -- bow;
        bow(rect(0.5,0.7));  <==>    rect(0.5,0.7) -- bow;
        ...
\end{verbatim}

\subsection{Hammer Methods}
\index{Hammer methods}
\hierindex{methods!Hammer}
For a description of the Hammer device see section \ref{section:hammer_device}.
The following methods are available with a hammer.

\begin{verbatim}
    reset()
    drop()
    (<accessPoint>)
    (<instr>, <x>)
    (<instr>, <x>, <y>)
    setMass(<m>)
    setPosition(<p>)
    setInitVelocity(<v>)
    setGravity(<g>)
    setHeight(<h>)
    setDamping(<d>)
    setHardness(<h>)
    setMaxImpacts(<maxImpacts>)
    getMass()
    getPosition()
    getVelocity()
    getInitVelocity()
    getGravity()
    getHeight()
    getDamping()
    getHardness()
    numberOfImpacts()
    getMaxImpacts()
\end{verbatim}

The \MethodIndex{reset} method resets the hammer to its initial height and
causes it to wait for a subsequent call to the \MethodIndex{drop} method
before the hammer will start falling and interacting with the instrument.
As with the bow device there are three unnamed methods for specifying
the access point with which the hammer will interact. The first expects
an access point, the second an instrument name and a single $x$ coordinate
and the third an instrument name followed by an $x$ and $y$ coordinate.

The \verb|set...| family of methods are used to set the various attributes
of the hammer. Note that the \verb|setHeight| method sets the height from
which the hammer is dropped whereas the \verb|setPosition| method sets
the instantaneous height of the hammer. The \verb|setInitVelocity| method
sets the initial velocity of the hammer immediately after the \verb|drop|
method has been called. The \verb|setHardness| method sets the strength
of the spring which is used to simulate the elastic connection with the
instrument. This is usually a value in the range [0..1] where 0 means
the spring has no effect and 1 means that the spring has the same strength
as the springs used in \tao's material. Values greater than 1 can also
be used although this can lead to the model becoming unstable, due to
the inherent limitations in modeling a continuous physical system using
discrete time steps or ticks. 

\subsection{Connector Methods}
\index{Connector methods}
\hierindex{methods!Connector}
The Connector methods listed below allow any combination of access
and anchor points to be coupled together. To recap, an \Term{anchor point}
is a fixed numerical value (usually 0.0) and a spring is connected
between the access point specified and this anchor point, effectively
restricting the vibrations of the instrument at that point.

\begin{verbatim}
    (<access point 1>, <access point 2>)
    (<access point 1>, <access point 2>, <strength>)
    (<access point>, <anchor>)
    (<access point>, <anchor>, <strength>)
    (<anchor>, <access point>)
    (<anchor>, <access point>, <strength>)
\end{verbatim}

The \verb|<strength>|\index{spring strength} argument expected by
some of the methods sets the strength of the spring used to connect
the two points. It is usually a value in the range [0..1] but higher
values may sometimes work. You should be aware though that if you
use a value higher than 1 \tao's cellular model may become unstable,
leading to exponentially increasing noisy vibrations. This is limitation
inherent the kind of discrete time step modelling used by \tao.

\subsection{Stop Methods}
\index{Stop methods}
\hierindex{methods!Stop}
The following stop device methods are available:

\begin{verbatim}
    dampModeOn()
    dampModeOff()
    setAmount(<amount>)
    setDamping(<damping>)
    (<access point>)
    (<instr>, <x>)
    (<instr>, <x>, <y>)
    (<string>, <pitch>)
\end{verbatim}

To briefly recap, the Stop device provides a rudimentary
mechanism for stopping strings in order to obtain specific pitches
from them. The \verb|<amount>| attribute is a value in the
range [0..1], with 0 meaning that the string is not stopped at all
and 1 meaning that it is fully stopped. The \verb|<damping>|
attribute determines how highly damped the left hand side of the string
will be (the right hand side is given the appropriate length to achieve
the specified pitch).

The unnamed \verb|(...)| methods are used to apply the stop to an
instrument. Much like the bow and hammer devices there are three standard
versions available, expecting either an access point, an instrument and
single $x$ coordinate, or an instrument and both $x$ and $y$ coordinates.
However for the Stop device there is a fourth method
\verb|(<string>, <pitch>)| available. This method, given a string
instrument and a pitch as arguments will automatically calculate at what
point the Stop device should be applied to the string in order to produce
the desired pitch.

Note that it is always the portion of the string to the right of the
applied Stop device which has the correct pitch. This should be borne
in mind if connecting the string to other components. If you build your
instrument with the left hand sides of each string attached to some sort
of resonator you will get all the wrong pitches when you start to play
the instrument!

Note also that as with the Bow and Hammer devices the preferred
syntax to use in a script when applying the device to an instrument
is as follows:

\begin{verbatim}
    string(0.1) -- stop;
\end{verbatim}

which is exactly equivalent to:

\begin{verbatim}
    stop(string(0.1));
\end{verbatim}

The first format is more commonly used since it is more clearly
legible when quickly scanning a script to see what it does. Wherever
the \verb|--| operator appears in a script, you know that there is
some kind of interfacing between access points and devices taking
place.

\subsection{Output Methods}
\index{Output methods}
\hierindex{methods!Output}
The Output device provides methods for writing sound
samples out to the various channels of its associated output file.
The  \MethodIndex{ch1} and \MethodIndex{chL} methods are equivalent, as are the
\MethodIndex{ch2} and \MethodIndex{chR} methods. Obviously \Method{chL}
and \Method{chR} are designed for use with two channel stereo
output. In each case \verb|<value>| is an arbitrary mathematical
expression yielding a floating point value. There is no need to
ensure that the output samples stay within a predefined range, since
all \tao\ output files are normalised as a post-processing stage 
before conversion into a more conventional integer-based format
such as WAV. This is achieved with the \Prog{taosf}
shell command (see section \ref{section:output_files}).

\begin{verbatim}
    ch1(<value>)
    ch2(<value>)
    ch3(<value>)
    ch4(<value>)
    chL(<value>)
    chR(<value>)
\end{verbatim}

\section{Access Point Methods}
These methods are some of the most important since they are the
ones which allow simulated physical interaction with \tao's
cellular material.

\begin{verbatim}
    getPosition()
    getVelocity()
    getForce()
    getInstrument()
    applyForce(<force>)
    clear()   
\end{verbatim}

The \MethodIndex{getPosition}, \MethodIndex{getForce} and \MethodIndex{getVelocity}
methods return the physical attributes of the material at the access point.
All three values returned are with respect to the $z$ axis of the material.

The \MethodIndex{getInstrument} method returns the instrument on which the
access point is operating. 

The \MethodIndex{applyForce} method applies the given force at the position
of the access point, not surprisingly! Note that in the same way that an
access point can be used to read the physical attributes of the material at
any point, forces can be applied at any point too. This means that you
could for example generate a moving access point whose instrument coordinates
were governed by some time varying value generated by an arbitrary expression,
and apply a time varying force at that moving point. 

The following example illustrates the use of the access point methods 
(although I don't know whether it would produce interesting sounds or not).

\begin{verbatim}
    Audio rate: 44100;
    
    String s(200 Hz, 20 secs);
    AccessPoint p1, p2;
    Param x1, x2;
    
    Init:
        s.lockEnds();
        ...
    
    Score 10 secs:
        x1=0.5+0.5*cos(Time*1000.0);
        x2=0.5+0.5*cos(Time*1100.0);
        p1=s(x1);
        p2=s(x2);
        
        Every 0.01 secs:
            Print "At time ", Time, ", position=", p1.getPosition(), 
                  " velocity=", p1.getVelocity(),  newline;
            ...
    
        If p1.getForce() < 1:
            p2.applyForce(1.0);
            ...
        If p2.getVelocity() > -1:
            p1.applyForce(-1.0);
            ...
        ...
\end{verbatim}

Finally, the \MethodIndex{clear} method resets the access point to
be null. Attempting to read any physical attributes off of the
access point will result in zero being returned. In addition,
attempting to apply a force to a null access point has no effect.

\section{Pitch Methods}
















\chapter{Tutorial}
\label{section:tutorial}
This section introduces the main concepts involved in creating
interesting virtual instruments using \tao. It does so in a 
`hands on' manner with the aid of practical script examples.
All of the examples presented in this section are available in the
\verb|examples| directory of the distribution. The examples start off with
the most basic instrument possible -- a single string -- and work up to more
and more complex instruments.

The examples presented here are divided into two categories. The first set
are designed to illustrate one by one the main techniques involved in 
contructing \tao\ instruments. Each script covers one major technique. The
second set are concerned with designing synthesis scenarios which produce
interesting sounds. So don't be too surprised if the first
set of scripts produce either no sound at all or rather uninteresting
sounds!

By the end of this tutorial you should have at your disposal a useful
armoury of techniques which will hopefully whet your appetite
for creating your own instruments.

\section{Learning the Basics}
\subsection{A simple string instrument}
The simplest possible instrument consists of a single string. The
following script illustrates how to create such an instrument,
`pluck' it and generate an output file. 

\begin{verbatim}
    Audio rate: 44100;

    String string(100 Hz, 20 secs);
    Output output(stereo);

    Init:
        string.lockEnds();
        ...

    Score 20 secs:
        At 0 secs for 1 msecs:
            string(0.1).applyForce(1.0);
            ...
        output.chL: string(0.1);
        output.chR: string(0.9);

	Every 0.1 secs: Print Time, newline; ...
        ...
\end{verbatim}

The first line of this script declares the audio sampling rate of
any output files generated (which in the current version has to be
44100). A single string with a fundamental frequency of 100Hz is
created, followed by a single two-channel output device which will
be used to write the movements of the string to a soundfile.

The statement \verb|string.lockEnds()| is contained within the \emph{Init}
block of the script. This block of statements is executed once just prior
to the performance and is delimited (as with any block of statements in
\tao\ with a \verb|:| just after the keyword \Kwd|Init| and a \verb|...|
just after the last statement.

The score is fairly self-explanatory. It has a duration of 20 seconds
and for a short time interval at the beginning of the performance a
fixed force of magnitude 1 is applied to a point on the string one tenth
of the way along its length. The movements of two points on the string
are traced throughout the performance and

\subsection{Damping the ends of the string}
Damping local regions of an instrument more highly than the rest of
the instrument leads to the vibrations in that part of the instrument
dying away more quickly. This is one of \tao's most important
features as it is probably the single most significant factor in
determining the character of an instrument after its basic shape
and structure.

For strings damping can be used to produce a more natural spectral
decay by choosing to damp small regions near the ends of the string.
By \emph{small regions} I mean in the order of $1/10$ to $1/20$ of
the length of the string.

The exact value chosen for the amount of damping depends on how
quickly you want the harmonics to die away. For example:

\begin{verbatim}
    Audio rate: 44100;

    String string(100 Hz, 20 secs);
    Output output(stereo);

    Init:
        string.lockEnds();
	string.setDamping(left, 0.1, 0.1);
	string.setDamping(right, 0.9, 0.1);
        ...

    Score 20 secs:
        At 0 secs for 1 msecs:
            string(0.1).applyForce(1.0);
            ...
        output.chL: string(0.1);
        output.chR: string(0.9);

	Every 0.1 secs: Print Time, newline; ...
        ...
\end{verbatim}


\subsection{Producing harmonics from the string}
\subsection{Coupling two strings together}
\subsection{An instrument with an array of pitched strings}
\subsection{Coupling the strings together}
\subsection{A rectangular sheet with locked corners and local damping}
\subsection{Bowed and stopped strings}
\subsection{Moving access points}
\subsection{Using the Connector device}
\subsection{Using the Hammer device}
\subsection{Using the Output device}
\subsection{Output Expressions}
\subsection{Controlling the graphical layout of instruments}

\section{Combining the techniques to make interesting sounds}
\subsection{Tips for bowing strings}
\subsection{Effective uses of damping}
\subsection{Using Connector devices and anchor points}
\subsection{Rules of thumb for effective instrument design}
\begin{itemize}
\item Try to make everything dynamically evolve. It will not
cost you any more computationally but will sound more 
interesting.
\item Always damp parts of an instrument so that its spectral
profile will change as the sound decays away. Instruments
with uniform damping often sound the most clinical and
synthetic.
\item Generate lots of output file from each performance. 
Whether or not you generate any output, \tao\ will still
have to churn its way through the intensive calculations
needed to realise a performance. It will not cost you any more
to see how ten different points on an instrument sound, rather
than just one.
\item Use lots of Connector devices. The more highly coupled
an instrument is the more complex the resulting vibrations will
be. Once again using lots of Connectors will not make the 
performance significantly more computationally expensive than
using none, but it will make the resulting sounds more
interesting.
\item Experiment with small parts of a large instrument in isolation
before coupling them together. If one of your strings goes ``clunk''
instead or ringing beautifully when you pluck it, there is not much
sense in connecting 100 such strings to a resonator and coming
back later only to find that the whole instrument goes ``clunk''!
\end{itemize}



\section{Using access points}
\subsection{Example 1 - accesspoint1.tao}
This example illustrates how to use access points\index{access points}
to generate output from an instrument and send the resulting data via an
Output device to a output file.

\begin{verbatim}
    Audio rate: 44100;
        
    String s1(100 Hz, 20 secs);    
    AccessPoint l=s1(0.1), r=s1(0.9);
    Output out(2);
        
    Init:
        s1.lockEnds();
        ...
        
    Score 5.0 secs:
        At 0 secs for 0.1 msecs:
            s1(0.05).applyForce(1.0);
            ...
        
        Every 0.1 secs:
            Print Time, " ", l, " ", r,  newline;
            ...
        
        out.chL: l;
        out.chR: r;
    ...    
\end{verbatim}

\subsection{Example 2 - accesspoint2.tao}
This example illustrates the use of access points\index{access points}
to connect together two components to form a more interesting compound
instrument. It does so by creating two strings and applying a short
impulse to one of them. A Connector device is used to connect together
two access points, one on each string. The access points are not fixed
but move along their respective strings as the performance progresses. 

\begin{verbatim}
    // accesspoint2.tao
    //
    // Create two strings and implement a connection between them using
    // two moving access points.
        
    Audio rate: 44100;
        
    String string1(200 Hz, 20 secs);
    String string2(200 Hz, 20 secs);
        
    AccessPoint point1, point2;
        
    Connector connector;
        
    Init:
        string1.lockEnds();
        string2.lockEnds();
        ...
        
    Score 5.0 secs:
        At 0 secs for 0.1 msecs:
            string1(0.05).applyForce(1.0);
            ...
        
        ControlRate 100:
            point1=string1(linear(0,1));
            point2=string2(linear(1,0));
            ...
            
        point1 -- connector -- point2;
        ...
\end{verbatim}

\section{An instrument with a single string}
This script creates an instrument with a single string and then applies
a force to one end of it for a short time interval.

\begin{verbatim}
    // string.tao
    //
    // Create a single string and apply a very short impulse to one
    // end. 
    
    Audio rate: 44100;
    
    String string(200 Hz, 20 secs);
    
    Init:
        string.lockEnds();
        ...
    
    Score 5 secs:
        At 0 secs for 1 msecs:
            string(0.1).applyForce(linear(1,0));
            ...
        ...    
\end{verbatim}
 
\section{Using the Bow device - bow.tao}
\begin{verbatim}
\end{verbatim}
     
\section{Creating a circular sheet}
\subsection{Example 1 - circle.tao}
This script creates a single circular sheet, locks its entire
perimeter, and then applies a linearly changing force at a point
$(x=0.3,y=0.3)$ for an interval of 1 millisecond.

\begin{verbatim}
    Audio rate: 44100;
        
    Circle circle(300 Hz, 20 secs);
        
    Init:
        circle.lockPerimeter();
        ...
        
    Score 5 secs:
        At 0 secs for 1 msecs:
            circle(0.3,0.3).applyForce(linear(40,0));
            ...
        ...
\end{verbatim}
 
\subsection{Example 2 - circle2.tao}
This script is similar to the previous one except that the circle has
its left and right edges locked instead of the whole perimeter.

\begin{verbatim}
    Audio rate: 44100;
        
    Circle circle(300 Hz, 20 secs);
        
    Init:
        circle.lockLeft().lockRight();
        ...
        
    Score 5 secs:
        At 0 secs for 1 msecs:
            circle(0.3,0.3).applyForce(linear(30,0));
            ...
        ...        
\end{verbatim}
 
\section{Using the Connector device - connector.tao}
This script creates two strings and a Connector device. The
Connector device is used to couple the two strings together. The
access points representing the end points of the connector move during
the performance. One migrates from one end of one of the strings to the
other, whilst the second migrates in the opposite direction on the other
string. This script also involves the \Type{Param} keyword.

\begin{verbatim}
    Audio rate: 44100;
        
    String string1(200 Hz, 30 secs);
    String string2(200 Hz, 30 secs);
        
    Connector conn;
        
    Param x1,x2;
        
    Init:
        string1.lockEnds();
        string2.lockEnds();
        ...
        
    Score 1 secs:
        At start for 1 msecs:
            string1(0.1).applyForce(1.0);
            ...
        
        x1 = linear(0,1);
        x2 = linear(1,0);
        
        string1(x1) -- conn -- string2(x2);
        ...    
\end{verbatim}
 
\section{Damping parts of an instrument - damp.tao}
This script creates a set of strings and damps the same region at the
left hand end of each string but to different degrees. The strings
are then all plucked in unison and the subsequent wave patterns give
a side by side comparison of the effects of different damping values.

\begin{verbatim}
    Audio rate: 44100;
        
    String string1(300 Hz, 20 secs);
    String string2(300 Hz, 20 secs);
    String string3(300 Hz, 20 secs);
    String string4(300 Hz, 20 secs);
    String string5(300 Hz, 20 secs);
    String string6(300 Hz, 20 secs);
    String string7(300 Hz, 20 secs);
    String string8(300 Hz, 20 secs);
    String string9(300 Hz, 20 secs);
    String string10(300 Hz, 20 secs);
        
    Init:
        string1.lockEnds().setDamping(left,1/20,0.0);
        string2.lockEnds().setDamping(left,1/20,0.1);
        string3.lockEnds().setDamping(left,1/20,0.2);
        string4.lockEnds().setDamping(left,1/20,0.3);
        string5.lockEnds().setDamping(left,1/20,0.4);
        string6.lockEnds().setDamping(left,1/20,0.5);
        string7.lockEnds().setDamping(left,1/20,0.6);
        string8.lockEnds().setDamping(left,1/20,0.7);
        string9.lockEnds().setDamping(left,1/20,0.8);
        string10.lockEnds().setDamping(left,1/20,0.8);
        ...
        
    Score 5 secs:
        At 0 secs for 0.1 msecs:
            string1(0.1).applyForce(10);
            string2(0.1).applyForce(10);
            string3(0.1).applyForce(10);
            string4(0.1).applyForce(10);
            string5(0.1).applyForce(10);
            string6(0.1).applyForce(10);
            string7(0.1).applyForce(10);
            string8(0.1).applyForce(10);
            string9(0.1).applyForce(10);
            string10(0.1).applyForce(10);
            ...
        Every 0.1 secs: Print Time, newline; ...
        ...
\end{verbatim}
 
\section{Emergent behaviour (diffraction) - diffraction.tao}
This script creates two rectangular sheets and joins them together using
the \Statement{Join} statement. The boundary between the two sheets is then
locked in several places leaving a few `slots' where the waves can get through.
A short impulse is applied to one of the sheets named verb|source|
and the resulting wave fronts interfere after having passed through the
slots to form diffraction patterns.

\begin{verbatim}
    Audio rate: 44100;
        
    Rectangle source(150 Hz, 300 Hz, 20 secs);
    Rectangle dest(150 Hz, 300 Hz, 20 secs);
        
    Init:
        source.lockCorners();
        dest.lockCorners();
        
        source.lock(0.000000, 0.050000, top, top);
        source.lock(0.070000, 0.120000, top, top);
        source.lock(0.140000, 0.190000, top, top);
        source.lock(0.210000, 0.260000, top, top);
        source.lock(0.280000, 0.330000, top, top);
        source.lock(0.350000, 0.400000, top, top);
        source.lock(0.420000, 0.470000, top, top);
        source.lock(0.490000, 0.540000, top, top);
        source.lock(0.560000, 0.610000, top, top);
        source.lock(0.630000, 0.680000, top, top);
        source.lock(0.700000, 0.750000, top, top);
        source.lock(0.770000, 0.820000, top, top);
        source.lock(0.840000, 0.890000, top, top);
        source.lock(0.910000, 0.960000, top, top);
        source.lock(0.980000, 1.000000, top, top);
        
        Join source(centre, top) to dest(centre, bottom);
        
        dest.setMagnification(5.0);
        ...
        
    Score 5 secs:
        At 0 secs for 0.1 msecs:
            source(0.5,bottom).applyForce(50);
            ...
        ...    
\end{verbatim}

\section{Creating an elliptical sheet}
\subsection{Example 1 - ellipse.tao}
This script creates an elliptical sheet, locks the whole perimeter
and then applies a short impulse to the sheet at a point $(x=0.15,y=0.5)$.

\begin{verbatim}
    Audio rate: 44100;
        
    Ellipse ellipse(200 Hz, 400 Hz, 20 secs);
        
    Init:
        ellipse.lockPerimeter();
        ...
        
    Score 5 secs:
        At 0 secs for 0.5 msecs:
            ellipse(0.15,0.5).applyForce(linear(30,0));
            ...
        ...
\end{verbatim}
 
\subsection{Example 2 - ellipse2.tao}
This script is similar to the previous one except that instead of
locking the whole perimeter, a thin strip of the instrument is locked.

\begin{verbatim}
    Audio rate: 44100;
        
    Ellipse ellipse2(200 Hz, 400 Hz, 20 secs);
        
    Init:
        ellipse2.lock(left, 0.8, centre, centre);
        ...
        
    Score 5 secs:
        At 0 secs for 0.5 msecs:
            ellipse2(0.3,0.2).applyForce(linear(30,0));
            ...
        ...
\end{verbatim}
 
\section{Using the Hammer device - hammer.tao}
This script illustrates the use of the Hammer device. It creates
a single string and a single hammer. It then locks the ends of the string,
applies the hammer to a point $x=0.7$ on the string, and sets some
hammer attributes.

Finally it drops the hammer at the beginning of the performance, leaving
it to bounce naturally on the string.

\begin{verbatim}
    Audio rate: 44100;
        
    String string(200 Hz, 30 secs);
        
    Hammer hammer;
        
    Init:
        string.lockEnds();
        string(0.7) -- hammer;
        hammer.setGravity(0.0001).setMass(200.0);
        ...
        
    Score 10 secs:
        At 0 secs:
            hammer.drop();
            ...
        ...
\end{verbatim}
 
\section{Locking parts of an instrument - lock.tao}
This script illustrates the use of the various \Term{locking}
instrument methods. These include \Method{lockCorners},
\Method{lockLeft}, \Method{lockRight}, \Method{lockTop},
\Method{lockPerimeter}, and \Method{lock}. It creates six
rectangular instruments and locks each one in a slightly different
way. It then applies short impulse to each instrument so as to allow
a side by side comparison of the effects on each.

\begin{verbatim}
    Audio rate: 44100;
        
    Rectangle rect1(500 Hz, 600 Hz, 20 secs);
    Rectangle rect2(500 Hz, 600 Hz, 20 secs);
    Rectangle rect3(500 Hz, 600 Hz, 20 secs);
    Rectangle rect4(500 Hz, 600 Hz, 20 secs);
    Rectangle rect5(500 Hz, 600 Hz, 20 secs);
    Rectangle rect6(500 Hz, 600 Hz, 20 secs);
        
    Init:
        rect1.lockCorners();
        rect2.lockLeft().lockRight();
        rect3.lockTop().lockBottom();
        rect4.lockPerimeter();
        rect5.lock(0.2,0.4);
        rect6.lock(0.7, right, 0.7, top);
        ...
        
    Score 5 secs:
        At 0 secs for 1 msecs:
            rect1(0.1,0.1).applyForce(linear(30,0));
            rect2(0.1,0.1).applyForce(linear(50,0));
            rect3(0.1,0.1).applyForce(linear(50,0));
            rect4(0.1,0.1).applyForce(linear(50,0));
            rect5(0.1,0.1).applyForce(linear(20,0));
            rect6(0.1,0.1).applyForce(linear(30,0));
            ...
        ...
\end{verbatim}
 
\section{Arrays of Output devices - outputarray.tao}
This script creates a string and an array of Output devices. It then
applies a short impulse to the string and writes output to each Output
device in the array from different positions on the string.

\begin{verbatim}
    Audio rate: 44100;
        
    String string(200 Hz, 30 secs);
        
    Output outputs(mono)[5];
        
    Init:
        string.lockEnds();
        ...
        
    Score 30 secs:
        At start for 0.1 secs:
            string(0.1).applyForce(1.0);
            ...
        
        outputs[0].ch1: string(0.1);
        outputs[1].ch1: string(0.3);
        outputs[2].ch1: string(0.5);
        outputs[3].ch1: string(0.7);
        outputs[4].ch1: string(0.9);
        ...
\end{verbatim}
 
\section{Using the Output device - outputs.tao}
This script illustrates \Term{stereo} and \Term{mono} Output
devices. It creates one of each and a single string and then writes output
to each channel of each device from different positions on the string,
after a short impulse has been applied to the string.

\begin{verbatim}
    Audio rate: 44100;
        
    String string(100 Hz, 30 secs);
        
    Output out1(stereo), out2(mono);
        
    Init:
        string.lockEnds();
        ...
        
    Score 30 secs:
        At start for 0.1 msecs:
            string(0.9).applyForce(1.0);
            ...
            
        out1.chL: string(0.1);
        out1.chR: string(0.9);
        
        out2.ch1: string(0.5);
        
        Every 0.1 secs: Print Time, newline; ...
        ...
\end{verbatim}
     
\section{Using pitches - pitches.tao}
This script illustrates the various pitch formats which are supported
by \tao. These include \Term{oct}, \Term{cps}, \Term{Hz} and
\Term{note name} formats (see sections \ref{section:pitches_and_frequencies}
and \ref{section:pitch_declarations}).

\begin{verbatim}
    Audio rate: 44100;
        
    String array1[]=
        {
        (200 Hz, 20 secs),
        (220 Hz, 20 secs),
        (240 Hz, 20 secs),
        (260 Hz, 20 secs)
        };
        
    String array2[]=
        {
        (8.00 pch, 20 secs),
        (8.04 pch, 20 secs),
        (8.06 pch, 20 secs),
        (8.08 pch, 20 secs)
        };
        
    String array3[]=
        {
        (8.0 oct, 20 secs),
        (8.2 oct, 20 secs),
        (8.4 oct, 20 secs),
        (8.6 oct, 20 secs)
        };
        
    Counter n;
        
    Init:
        For n = 0 to 3:
            array1[n].lockEnds();
            array2[n].lockEnds();
            array3[n].lockEnds();
            ...
        ...
        
    Score 5 secs:
        Label(array1[0], 1.0, 0.0, 0.0, "LABEL", 0);
        
        At start for 0.1 msecs:
            For n = 0 to 3:
                array1[n](0.1).applyForce(1.0);
                array2[n](0.1).applyForce(1.0);
                array3[n](0.1).applyForce(1.0);
                ...
            ...
        ...
\end{verbatim}
 
\section{Creating a rectangular sheet}
\subsection{Example 1 - rectangle.tao} 
This script creates a rectangular sheet, locks all four corners and
then applies a short impulse at a point ($x$=0.1,$y$=0.1). This impulse
consists of a force linearly changing from a value of 30 to 0 over a
1ms interval.

\begin{verbatim}
    Audio rate: 44100;
        
    Rectangle rectangle(300 Hz, 400 Hz, 20 secs);
        
    Init:
        rectangle.lockCorners();
        ...
        
    Score 5 secs:
        At 0 secs for 1 msecs:
            rectangle(0.1,0.1).applyForce(linear(30,0));
            ...
        ...
\end{verbatim}
 
\subsection{Example 2 - rectangle2.tao}
This script is similar to the previous one except that the left and
right sides of the rectangle are locked instead of all four corners.

\begin{verbatim}
    Audio rate: 44100;
        
    Rectangle rectangle2(150 Hz, 800 Hz, 20 secs);
        
    Init:
        rectangle2.lockLeft().lockRight();
        ...
        
    Score 5 secs:
        At 0 secs for 1 msecs:
            rectangle2(0.1,0.1).applyForce(linear(30,0));
            ...
        ...
\end{verbatim}
 
\section{Using the Stop device - stop.tao}
This script illustrates the use of the \Device{Stop} device.

\begin{verbatim}
    Audio rate: 44100;
        
    String string1(200 Hz, 40 secs);
    Stop stop;    
    Param position, amount=0.0;
        
    Init:
        string1.lockEnds();
        ...
            
    Score 0.3 secs:
        At 0 secs for 0.1 msecs:
            string1(0.9).applyForce(10.0);
            ...
        
        position=linear(0.1, 0.9);
        	
        string1(position) -- stop;
        
        From 0.05 secs to 0.1 secs:
            amount=linear(0,1);
            stop.setAmount(linear(0,1));
            ...
        
        From 0.20 to 0.25 secs:
            amount=linear(1,0);
            stop.setAmount(linear(1,0));
            ...
        
        Every 0.005 secs:
            Print Time, " ", amount, newline;
            ...
        ...
\end{verbatim}
 
\section{Creating an array of strings - stringarray.tao}

\begin{verbatim}
    Audio rate: 44100;
        
    String string[]=
        {
        (8.00 pch, 20 secs),
        (8.01 pch, 20 secs),
        (8.02 pch, 20 secs),
        (8.03 pch, 20 secs),    
        (8.04 pch, 20 secs),
        (8.05 pch, 20 secs),
        (8.06 pch, 20 secs),
        (8.07 pch, 20 secs),    
        (8.08 pch, 20 secs),
        (8.09 pch, 20 secs),
        (8.10 pch, 20 secs),
        (8.11 pch, 20 secs)
        };
        
    Counter n=0;
    Param startPluck=0.0, pluckDuration=0.001;
        
    Init:
        For n = 0 to 11:
            string[n].lockEnds();
            ...
        ...
        
    Score 5 secs:
        At start:
            n=0;
            ...
        
        At startPluck for pluckDuration:
            At start:
                Print "Plucking string ", n, " at ", startPluck, " seconds", newline;
                ...
        
            string[n](0.1).applyForce(linear(1,0));
        
            At end:
                n+=1;
                If n <= 11:
                    startPluck+=0.1;
                    // pluck the next string in 0.1 seconds time
    
                    Print "Pluck string ", n, " at ", startPluck, " seconds", newline;
                    ...
                Else:
                    startPluck=-1.0;
                    // prevent any more plucks from occurring
                    // by setting startPluck to a negative value.
                    Print "No more strings to be plucked", newline;
                    ...
                ...
            ...
        ...
\end{verbatim}











\chapter{Closing Comments}
This closing section of the User Manual is basically a place for
anything else I thought was important enough to include, but couldn't
find a suitable place for elsewhere.

\section{Background to \tao's design and implementation}
During my Music Technology masters degree at the University of York I was
impressed by freeware audio and synthesis tools such as \Prog{Csound}.
One of the things that impressed me about Csound in particular was that
it was a synthesis language and allowed a finite set of primitive
building blocks to be assembled in an infinite number of ways. 

However, my experience as a musician playing a variety of acoustic
instruments told me that whilst Csound's unit generator approach was
quite powerful, it also had several shortcomings. For example, when
playing an acoustic instrument such as a guitar or even just experimenting
with `found sounds' there is something very direct and intuitive about
the mode of experimentation. If you want something to make a louder
and brighter sound you just hit it harder!

Whilst Csound's unit
generator approach does allow you to design instruments with input
parameters and feed different values into these inputs in a score,
it doens't allow you to think directly in terms of forces, velocities,
spatial positioning of excitations etc. So I wanted to set out to
design a synthesis program which would carry on the tradition of
programs like Csound, in providing an open-ended synthesis language,
but would be capable of creating much more tangibly physical
instruments. Another item which was high on the agenda was to be able
to visualise the instruments.

\tao\ arose out of my interests in a number of different areas
including musical performance, electroacoustic music, computer
modeling of complex dynamical systems, cellular automata, and
computer graphics.

I have thought for a long time that whilst GUIs (graphical
user interfaces) are invaluable tools in certain situations there are
still many things which are more elegantly expressed in text/language
form. Take general computer programming languages for example, or
the rapidly growing number of `mark-up languages' such as HTML, XML,
VRML. Text is a very powerful tool for communicating structured
ideas and has the advantage that it can serve as the basis for
more user friendly GUI-based tools to sit on top.

It is for this reason that I concentrated my efforts on designing a
text based interface to \tao\ which would be simple to use, clear to read
and above all accessible to musicians with some degree of technical
competency (having used tools such as Csound for example). Having
stated this however it is likely that the format of the language and
the lack of a GUI will be addressed in the future, along with several
other aspects of the user interface.

\tao\ was originally developed as part of my DPhil at the
University of York, England and then subsequently during a one year
visiting research fellowship to the Australian Centre for the Arts
and Technology at the Australian National University in Canberra.

My DPhil addressed the question of precisely why it is that digitally
synthesised sounds often lack the \emph{warmth}, \emph{life}, and
\emph{organic} qualities of acoustically produced sounds, whether
musical in the traditional sense or not. What came out of this work,
apart from a thesis of course, was \tao.

As mentioned earlier on I wanted a system which would be capable of producing
\emph{organic sounds}. The term \emph{organic}
is quite difficult to define precisely but my thesis \emph{Synthesis
of Organic Sounds for Electroacoustic Music: Cellular Models and the
TAO Computer Music Program} does a better job of
addressing the issues than I have scope to do here. Very briefly though
the term \emph{organic} is used to refer to sounds which are:

\begin{itemize}
\item complex
\item fluid
\item dynamic
\item coherent
\item lively
\item suggestive of physical and energetic causality
\end{itemize}

The term \Term{coherence} is used to refer to the fact that in sounds
produced by physical means, the transient behaviour, the perception
of the sound having been produced by some physical mechanism and
the overall character of the sound hang together very well. This cannot
be said of many digitally synthesised sounds, even those produced by
some physical modeling techniques. This problem of synthesising
\emph{complex}, \emph{coherent} and \emph{organic} sounds was the main
focus of the whole project.

The original design goals which have been adhered to throughout \tao's
development were to produce a system which would have the following
features:

\begin{itemize}
\item capable of synthesising a wide variety of acoustic and
instrumental-like sounds with organic qualities;
\item relatively straightforward to use, making physical modeling accessible
to those without a strong maths or physics background without compromising
the power or flexibility of the tool;
\item based around a flexible and open-ended synthesis language following
in the tradition of other synthesis languages such as Csound (this objective
is ongoing, the current synthesis language used by \tao\ is only one possible
language for controlling it).
\end{itemize}

\section{The computational expense of Tao's synthesis engine}
\tao\ takes what has been referred to as the `brute force' approach to physical
modelling and as such is not as efficient as some of the digital waveguide
models developed by Julius Smith et al. However at the time of designing \tao\
I made a conscious decision to steer away from the obsession with real-time
performance and look at what would be possible if I just concentrated on the
conceptual structure of the system. First and foremost I wanted to design
a system which made instruments which were tangible objects. As a musician
I am accustomed to being able to experiment with sound in a direct,
physical and intuitive manner and having a system which was capable
of visualising the instruments was quite high on the list of priorities.
At least if I couldn't actually get my hands on the instruments I could
see them, which would in turn fuel my imagination for things to try out.

Real instruments such as stringed or percussion instruments have a wonderful
property that their 'user interface' is spatially distributed, and more
importantly, doing things to the instrument at different spatial locations
leads to markedly different timbral results.

In my DPhil thesis I argued that details which make a huge difference to the
aesthetic appeal of the sounds produced from a physical model are often simply
missed out in the name of real-time performance. I do not make any great
claims about my model being so much more mathematically accurate than any
others simply because it needs more processing power, but one
thing I would say is that in the majority of cases musicians and composers
who have heard the sounds which \tao\ is capable of producing have commented
on their \emph{organic} and inherent musical qualities, which cannot be a bad
thing. Besides, with the exponential growth in computing power, the number
crunching needed by \tao\ simply may not be an issue in the near future.

The calculations employed to animate the model are described in detail in
my thesis. As they stand I am sure that improvements could be made in the
name of efficiency and optimisation although they have already been optimised
to an extent. There may be ways in which the efficiency could be radically
improved without compromising the quality of the sounds produced and indeed
I would be very interested to hear any ideas from individuals more mathematically
skilled than myself. \tao\ is based upon mathematical skills I picked up at
school level, so there should be room for improvement!

\section{Deficiencies in Tao's synthesis language}
\label{section:script_deficiencies}
\tao's synthesis language was developed primarily as a test-bed for the synthesis
engine. It has evolved into a usable language but lacks several features for
larger scale compositional work including:
\begin{itemize}
\item
Encapsulation of events
\item
Encapsulation of instrument components
\item
Table generating and reading functions
\end{itemize}

What is meant by `encapsulation of events' is the ability to describe a
complex algorithm for producing some kind of high-level event and then
place this algorithm inside a black box with input parameters. The
algorithm would then be invoked whenever it was needed
(much like a C or C++ function) using just its name and arguments.
The addition of this feature would make it a much simpler matter to
produce complex multi-layered textures of sound.

Encapsulation as applied to instruments refers to the ability to create
compound instruments where the components of the instrument are arranged
hierarchically with parent-child relationships. For example an instrument
named `guitar' might have child components named `string1', `string2'
etc. In order to achieve this the syntax of \tao's synthesis language
would need to be modified to allow components to be created within the
scope of other components, much as local variables may be declared
within C or C++ functions.

It would also be possible to describe template instruments which would
act as templates for whole families of instruments with similar
characteristics. For example the description of a `guitar' template
instrument would allow the construction of multiple instances each
having their own body sizes and string tunings, but with aspects of
the physical structure common to each instance.

Table reading functions are an essential part of any synthesis program
and the only way they can be implemented at the moment is by setting
up arrays of values by hand and accessing (and interpolating them)
yourself with parameters declared in the script. Of course all the 
math library functions are available in the synthesis language so it
is feasible to write table initialisation code in the Init part of
the score, but it is still cumbersome compared with \Prog{Csound}'s
provision of table generating features.

\section{New Devices}
As it stands there are only a handful of devices available for use. A
long term goal is to expand this set of available devices in order to
make \tao a more powerful and enticing environment for sound design.

\section{Parallel Processing}
I did carry out some initial investigations into how \tao could be 
parallelised whilst at the Australian Centre for the Arts and Technology
during 1997-8 and came up with the idea of using Posix threads (pthreads)
to split the number crunching performed by the synthesis engine up into
separate threads. Unfortunately the work went no further than that but
will be resumed one day if I have access to a multi-processor machine.

\section{The User Interface}
As with the parallelisation of \tao's synthesis engine I did some evaluation
of the various options available both on the SGI and Linux platforms
for building a GUI for \tao\ whilst at ACAT. However at the time there were
too many other things to be done to improve the basic synthesis engine
and I never atually implemented anything concrete. Since then
the main GUI toolkits for Linux (Gtk and Qt) have become more widely
used and documented and as such are obvious choices. Of course the use
of Qt would also make it very easy to create a port for other platforms
such as MS Windows.

I have had some more general thoughts on the subject of GUIs though and
having had some brief experience of 3-D animation packages such as Side Effects'
Houdini and more recently Blender I think that many of the concepts 
used in their interfaces would be applicable to \tao\. To be more specific
the ability of such packages to control and animate the values of any
parameters with the use of various spline, bezier and NURBS curves
is directly applicable. The other obvious area in which there is overlap
is that \tao\ is basically a modeling tool and many of the concepts used
in the user interfaces of animation packages such as object heirarchies;
different views, layers of objects etc. would also be applicable.

Both Houdini and Blender allow the evolution of objects to be described
in a procedural manner through the use of use of scripting. Blender in
particular uses the Python OO scripting language. \tao\ could also benefit
from the dual GUI/scripting approach. 

To summarise, the following features would be desirable in a new
interface for \tao:

\begin{itemize}
\item
a curve editor with editable curves representing arbitrary parameters;
\item
a graphical instrument editor with a suitable toolbox;
\item
a 3D visualisation window with rotate/zoom/translate capabilities;
\item
some kind of data block structure showing hierarchies of components,
devices and other objects and their relationships;
\item
some kind of integrated graphical and text based score system
where changes in the graphical representation would immediately
lead to changes in the text version and vice versa.
\end{itemize}

\section{Contributing to Tao's development}
Finally a word on how you can contribute to \tao's development.

\begin{itemize}
\item
Email me if you find any glaring errors or mistakes in this manual or
the software itself.
\item
Tell your friends about \tao. The more interest there is in \tao\ the
more likely I am to continue development.
\item
If anyone would like to set up a CVS server or mailing list I would be
happy to cooperate. I may eventually do both myself as and when I get time.
\item
I am happy to discuss ways in which any aspect of \tao\ can be improved
or developed. 
\end{itemize}

To conclude -- my vision when I started designing \tao\ was to create a
powerful and intuitive sound synthesis tool which would allow composers
to create their own `acoustic' instruments to order. There is still a
long way to go to realise this vision but \tao\ is already fun to play with
in its present state. As machines become faster and faster it will become
possible to experiment with increasingly complex (and realistic) instruments,
eventually in real-time. The provision of a sophisticated GUI and maybe even
haptic feedback interfaces will eventually bring \tao\ closer to this goal. 

In the meantime I hope you find \tao\ useful and enjoy using it!

Mark Pearson.








\printindex
\end{document}












