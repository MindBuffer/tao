\chapter{Closing Comments}
This closing section of the User Manual is basically a place for
anything else I thought was important enough to include, but couldn't
find a suitable place for elsewhere.

\section{Background to \tao's design and implementation}
During my Music Technology masters degree at the University of York I was
impressed by freeware audio and synthesis tools such as \Prog{Csound}.
One of the things that impressed me about Csound in particular was that
it was a synthesis language and allowed a finite set of primitive
building blocks to be assembled in an infinite number of ways. 

However, my experience as a musician playing a variety of acoustic
instruments told me that whilst Csound's unit generator approach was
quite powerful, it also had several shortcomings. For example, when
playing an acoustic instrument such as a guitar or even just experimenting
with `found sounds' there is something very direct and intuitive about
the mode of experimentation. If you want something to make a louder
and brighter sound you just hit it harder!

Whilst Csound's unit
generator approach does allow you to design instruments with input
parameters and feed different values into these inputs in a score,
it doens't allow you to think directly in terms of forces, velocities,
spatial positioning of excitations etc. So I wanted to set out to
design a synthesis program which would carry on the tradition of
programs like Csound, in providing an open-ended synthesis language,
but would be capable of creating much more tangibly physical
instruments. Another item which was high on the agenda was to be able
to visualise the instruments.

\tao\ arose out of my interests in a number of different areas
including musical performance, electroacoustic music, computer
modeling of complex dynamical systems, cellular automata, and
computer graphics.

I have thought for a long time that whilst GUIs (graphical
user interfaces) are invaluable tools in certain situations there are
still many things which are more elegantly expressed in text/language
form. Take general computer programming languages for example, or
the rapidly growing number of `mark-up languages' such as HTML, XML,
VRML. Text is a very powerful tool for communicating structured
ideas and has the advantage that it can serve as the basis for
more user friendly GUI-based tools to sit on top.

It is for this reason that I concentrated my efforts on designing a
text based interface to \tao\ which would be simple to use, clear to read
and above all accessible to musicians with some degree of technical
competency (having used tools such as Csound for example). Having
stated this however it is likely that the format of the language and
the lack of a GUI will be addressed in the future, along with several
other aspects of the user interface.

\tao\ was originally developed as part of my DPhil at the
University of York, England and then subsequently during a one year
visiting research fellowship to the Australian Centre for the Arts
and Technology at the Australian National University in Canberra.

My DPhil addressed the question of precisely why it is that digitally
synthesised sounds often lack the \emph{warmth}, \emph{life}, and
\emph{organic} qualities of acoustically produced sounds, whether
musical in the traditional sense or not. What came out of this work,
apart from a thesis of course, was \tao.

As mentioned earlier on I wanted a system which would be capable of producing
\emph{organic sounds}. The term \emph{organic}
is quite difficult to define precisely but my thesis \emph{Synthesis
of Organic Sounds for Electroacoustic Music: Cellular Models and the
TAO Computer Music Program} does a better job of
addressing the issues than I have scope to do here. Very briefly though
the term \emph{organic} is used to refer to sounds which are:

\begin{itemize}
\item complex
\item fluid
\item dynamic
\item coherent
\item lively
\item suggestive of physical and energetic causality
\end{itemize}

The term \Term{coherence} is used to refer to the fact that in sounds
produced by physical means, the transient behaviour, the perception
of the sound having been produced by some physical mechanism and
the overall character of the sound hang together very well. This cannot
be said of many digitally synthesised sounds, even those produced by
some physical modeling techniques. This problem of synthesising
\emph{complex}, \emph{coherent} and \emph{organic} sounds was the main
focus of the whole project.

The original design goals which have been adhered to throughout \tao's
development were to produce a system which would have the following
features:

\begin{itemize}
\item capable of synthesising a wide variety of acoustic and
instrumental-like sounds with organic qualities;
\item relatively straightforward to use, making physical modeling accessible
to those without a strong maths or physics background without compromising
the power or flexibility of the tool;
\item based around a flexible and open-ended synthesis language following
in the tradition of other synthesis languages such as Csound (this objective
is ongoing, the current synthesis language used by \tao\ is only one possible
language for controlling it).
\end{itemize}

\section{The computational expense of Tao's synthesis engine}
\tao\ takes what has been referred to as the `brute force' approach to physical
modelling and as such is not as efficient as some of the digital waveguide
models developed by Julius Smith et al. However at the time of designing \tao\
I made a conscious decision to steer away from the obsession with real-time
performance and look at what would be possible if I just concentrated on the
conceptual structure of the system. First and foremost I wanted to design
a system which made instruments which were tangible objects. As a musician
I am accustomed to being able to experiment with sound in a direct,
physical and intuitive manner and having a system which was capable
of visualising the instruments was quite high on the list of priorities.
At least if I couldn't actually get my hands on the instruments I could
see them, which would in turn fuel my imagination for things to try out.

Real instruments such as stringed or percussion instruments have a wonderful
property that their 'user interface' is spatially distributed, and more
importantly, doing things to the instrument at different spatial locations
leads to markedly different timbral results.

In my DPhil thesis I argued that details which make a huge difference to the
aesthetic appeal of the sounds produced from a physical model are often simply
missed out in the name of real-time performance. I do not make any great
claims about my model being so much more mathematically accurate than any
others simply because it needs more processing power, but one
thing I would say is that in the majority of cases musicians and composers
who have heard the sounds which \tao\ is capable of producing have commented
on their \emph{organic} and inherent musical qualities, which cannot be a bad
thing. Besides, with the exponential growth in computing power, the number
crunching needed by \tao\ simply may not be an issue in the near future.

The calculations employed to animate the model are described in detail in
my thesis. As they stand I am sure that improvements could be made in the
name of efficiency and optimisation although they have already been optimised
to an extent. There may be ways in which the efficiency could be radically
improved without compromising the quality of the sounds produced and indeed
I would be very interested to hear any ideas from individuals more mathematically
skilled than myself. \tao\ is based upon mathematical skills I picked up at
school level, so there should be room for improvement!

\section{Deficiencies in Tao's synthesis language}
\label{section:script_deficiencies}
\tao's synthesis language was developed primarily as a test-bed for the synthesis
engine. It has evolved into a usable language but lacks several features for
larger scale compositional work including:
\begin{itemize}
\item
Encapsulation of events
\item
Encapsulation of instrument components
\item
Table generating and reading functions
\end{itemize}

What is meant by `encapsulation of events' is the ability to describe a
complex algorithm for producing some kind of high-level event and then
place this algorithm inside a black box with input parameters. The
algorithm would then be invoked whenever it was needed
(much like a C or C++ function) using just its name and arguments.
The addition of this feature would make it a much simpler matter to
produce complex multi-layered textures of sound.

Encapsulation as applied to instruments refers to the ability to create
compound instruments where the components of the instrument are arranged
hierarchically with parent-child relationships. For example an instrument
named `guitar' might have child components named `string1', `string2'
etc. In order to achieve this the syntax of \tao's synthesis language
would need to be modified to allow components to be created within the
scope of other components, much as local variables may be declared
within C or C++ functions.

It would also be possible to describe template instruments which would
act as templates for whole families of instruments with similar
characteristics. For example the description of a `guitar' template
instrument would allow the construction of multiple instances each
having their own body sizes and string tunings, but with aspects of
the physical structure common to each instance.

Table reading functions are an essential part of any synthesis program
and the only way they can be implemented at the moment is by setting
up arrays of values by hand and accessing (and interpolating them)
yourself with parameters declared in the script. Of course all the 
math library functions are available in the synthesis language so it
is feasible to write table initialisation code in the Init part of
the score, but it is still cumbersome compared with \Prog{Csound}'s
provision of table generating features.

\section{New Devices}
As it stands there are only a handful of devices available for use. A
long term goal is to expand this set of available devices in order to
make \tao a more powerful and enticing environment for sound design.

\section{Parallel Processing}
I did carry out some initial investigations into how \tao could be 
parallelised whilst at the Australian Centre for the Arts and Technology
during 1997-8 and came up with the idea of using Posix threads (pthreads)
to split the number crunching performed by the synthesis engine up into
separate threads. Unfortunately the work went no further than that but
will be resumed one day if I have access to a multi-processor machine.

\section{The User Interface}
As with the parallelisation of \tao's synthesis engine I did some evaluation
of the various options available both on the SGI and Linux platforms
for building a GUI for \tao\ whilst at ACAT. However at the time there were
too many other things to be done to improve the basic synthesis engine
and I never atually implemented anything concrete. Since then
the main GUI toolkits for Linux (Gtk and Qt) have become more widely
used and documented and as such are obvious choices. Of course the use
of Qt would also make it very easy to create a port for other platforms
such as MS Windows.

I have had some more general thoughts on the subject of GUIs though and
having had some brief experience of 3-D animation packages such as Side Effects'
Houdini and more recently Blender I think that many of the concepts 
used in their interfaces would be applicable to \tao\. To be more specific
the ability of such packages to control and animate the values of any
parameters with the use of various spline, bezier and NURBS curves
is directly applicable. The other obvious area in which there is overlap
is that \tao\ is basically a modeling tool and many of the concepts used
in the user interfaces of animation packages such as object heirarchies;
different views, layers of objects etc. would also be applicable.

Both Houdini and Blender allow the evolution of objects to be described
in a procedural manner through the use of use of scripting. Blender in
particular uses the Python OO scripting language. \tao\ could also benefit
from the dual GUI/scripting approach. 

To summarise, the following features would be desirable in a new
interface for \tao:

\begin{itemize}
\item
a curve editor with editable curves representing arbitrary parameters;
\item
a graphical instrument editor with a suitable toolbox;
\item
a 3D visualisation window with rotate/zoom/translate capabilities;
\item
some kind of data block structure showing hierarchies of components,
devices and other objects and their relationships;
\item
some kind of integrated graphical and text based score system
where changes in the graphical representation would immediately
lead to changes in the text version and vice versa.
\end{itemize}

\section{Contributing to Tao's development}
Finally a word on how you can contribute to \tao's development.

\begin{itemize}
\item
Email me if you find any glaring errors or mistakes in this manual or
the software itself.
\item
Tell your friends about \tao. The more interest there is in \tao\ the
more likely I am to continue development.
\item
If anyone would like to set up a CVS server or mailing list I would be
happy to cooperate. I may eventually do both myself as and when I get time.
\item
I am happy to discuss ways in which any aspect of \tao\ can be improved
or developed. 
\end{itemize}

To conclude -- my vision when I started designing \tao\ was to create a
powerful and intuitive sound synthesis tool which would allow composers
to create their own `acoustic' instruments to order. There is still a
long way to go to realise this vision but \tao\ is already fun to play with
in its present state. As machines become faster and faster it will become
possible to experiment with increasingly complex (and realistic) instruments,
eventually in real-time. The provision of a sophisticated GUI and maybe even
haptic feedback interfaces will eventually bring \tao\ closer to this goal. 

In the meantime I hope you find \tao\ useful and enjoy using it!

Mark Pearson.







