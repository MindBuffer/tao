\chapter{Object Method Reference}
\label{section:object_method_reference}
\index{methods}
What follows in this section is a detailed description of all the
methods which are provided by the various object classes. The syntax
of each method is given, together with a description of its arguments,
purpose and function. The vast majority of methods occur with the
following generic syntax:

\begin{verbatim}
    object.method(<arg1>,<arg2> .. <argn>);
\end{verbatim}

where \verb|object| would be the name of an actual instrument, device or
other object and \verb|method| the name of an actual method.

However some of the most commonly accessed methods such as generating
an access point from an instrument name and a pair of coordinates have
a simpler syntax where the arguments are placed in parentheses immediately
after the object name, i.e. there is no method name as such. For example:

\begin{verbatim}
    instrument(<x>,<y>)
\end{verbatim}

In the following reference to all the object methods such methods
are described merely by the arguments which are placed in between
the brackets, e.g.

\begin{verbatim}
    (<x>,<y>)
\end{verbatim}

\section{Instrument Methods grouped by Function}
\hierindex{instrument!methods}
All instrument methods are listed here grouped together by function.

\subsection{Locking Parts of an Instrument}
\index{locking}
A number of instrument methods are available for locking parts of an
instrument. These include:

\begin{verbatim}
    lock(<x>,<y>)
    lockLeft()
    lockRight()
    lockTop()
    lockBottom()
    lockEnds()
    lockCorners()
    lockPerimeter()
\end{verbatim}

The \MethodIndex{lock} method locks the single cell which is nearest
to the \verb|(<x>,<y>)| position specified in instrument coordinates
(see section \ref{section:access_points}).

The \MethodIndex{lockLeft}, \MethodIndex{lockRight}, \MethodIndex{lockTop}
and \MethodIndex{lockBottom} methods each lock the cells at one extremity
of an instrument. For rectangular sheets the behaviour is obvious, but
for other instruments some clarification is needed. For strings, only the
\Method{lockLeft} and \Method{lockright} methods are appropriate. For
circular and elliptical sheets only a few cells at the edges of the sheet
will be locked by each method.

The \Method{lockEnds} method is equivalent to issuing a \Method{lockLeft}
and \Method{lockRight} together and is used mostly with strings.

Finally the \MethodIndex{lockCorners} is only appropriate for rectangular
sheets and the \MethodIndex{lockPerimeter} is appropriate for all
instruments except strings.

\subsection{Damping Parts of an Instrument}
\index{damping methods}
\hierindex{methods!damping}
A number of methods are provided for damping parts of an instrument.
These include:

\begin{verbatim}
    setDamping(<d>)
    setDamping(<x>,<d>)
    setDamping(<x>,<y>,<d>)
    setDamping(<x1>,<x2>,<y1>,<y2>,<d>)
    resetDamping()
    resetDamping(<x>)
    resetDamping(<x>,<y>)
    resetDamping(<x1>,<x2>,<y1>,<y2>)
\end{verbatim}

In each case the argument \verb|<d>| is a floating point value in
the range [0..1], where 0 represents no damping at all and 1 means that
the portion of material affected will be locked rigidly in a fixed
position. The progression from the former state to the latter as
\verb|<d>| changes from 0 to 1 is logarithmic rather than linear
for reasons which are explained below.

The damping value \verb|<d>| is converted via the following formula
into the appropriate \Attr{velocityMultiplier} attribute (see section
\ref{section:cell_attributes}):

\[ v_{m} = 1 - \frac{10000^{d}}{10000} \]

where $v_{m}$ and $d$ correspond to \Attr{velocityMultiplier} and
\verb|<d>| respectively.

You may remember that the \Attr{velocityMultiplier} value also lies in the
range [0..1] and on each tick of the synthesis engine the velocity of
each cell is multiplied by this value.

The \MethodIndex{resetDamping} family of methods set the damping back to what
it was when the instrument was created. This is useful for situations
where it is desirable to temporarily damp a region. An example application
might be playing a harmonic on a string. Any guitarist will know that
in order to do so a finger is placed momentarily in contact with one of
the strings, over a node, whilst (or after) the string is plucked. Once
the harmonic begins to clearly ring out the players finger is removed
again leaving the string to continue vibrating in its modified pattern.

\subsection{Graphically Placing Instruments}
\label{graphical_placement}
\hierindex{instrument!graphical placement}
\hierindex{methods!graphical placement of instruments}
These methods allow the user to override \tao's default graphical placement
scheme. The default scheme is not very intelligent in the current version
and simply places each new instrument above the previous one (`above'
meaning in the +ve $y$ direction). The \MethodIndex{placeAt} method
allows the position of the bottom left hand corner of the bounding
box surrounding the instrument to be set explicitly. The 
\MethodIndex{placeAbove}, \MethodIndex{placeBelow}, \MethodIndex{placeRightOf}
and \MethodIndex{placeLeftOf} methods allow an instrument to be 
placed relative to another instrument.

In the case of the methods which expect an additional argument
\verb|<offset>|, this argument specifies an additional offset
measured in world coordinates from the reference instrument (the default
is to seperate each instrument by 5 units in world coordinates).
This is somtimes necessary to prevent the instrument visualisation
window from becoming too cluttered.

\begin{verbatim}
    placeAt(<x, y>)
    placeAbove(<instrument>)
    placeBelow(<instrument>)
    placeRightOf(<instrument>)
    placeLeftOf(<instrument>)
    placeAbove(<instrument>, <offset>)
    placeBelow(<instrument>, <offset>)
    placeRightOf(<instrument>, <offset>)
    placeLeftOf(<instrument>, <offset>)
\end{verbatim}

\subsection{Accessing the Internal Attributes of an Instrument}
\hierindex{instrument!attributes!accessing}
\hierindex{methods!instrument attribute}
This set of methods allow the internal attributes of an instrument
to be inspected. Please note that some of these methods are only listed
for completeness. In practice they are not of much use in the
average \tao\ script.

\begin{verbatim}
    getName()
    getMagnification()
    getWorldX()
    getWorldY()
    getXMax()
    getYMax()
    getXFrequency()
    getYFrequency()
\end{verbatim}

The \MethodIndex{getName} method returns a string containing the name
of the instrument. The \MethodIndex{getMagnification} method returns
the current factor by which the visual amplitude of the waves in the
instrument are being magnified. The two methods \MethodIndex{getWorldX}
and \MethodIndex{getWorldY} return the world coordinates of the bottom
left hand corner of the bounding box around the instrument (i.e. the
position in the $xy$ plane). The \MethodIndex{getXMax} and
\MethodIndex{getXMax} methods return the $N-1$ where $N$ is the
width or height of the instrument in cells respectively.

The only two methods which should be of any use in the average \tao\
script are \MethodIndex{getXFrequency} and \MethodIndex{getYFrequency}.
These return the pitch values which were passed in when the instrument
was created, but converted to Hertz, regardless of the initial pitch
format used.

\subsection{Setting the Internal Attributes of an Instrument}
\hierindex{instrument!attributes!setting}
The only instrument attribute which can be set by a user is the
factor by which the amplitude of vibrations in the component are
exaggerated in the visualisation window. This is useful
for evening out differences between components within an instrument,
for the purposes of visualisation only. This attribute has no effect
on sound output.

\begin{verbatim}
    setMagnification(<m>)
\end{verbatim}

\subsection{Automatically Generating Access Points}
\hierindex{access points!generating}
\hierindex{methods!access point generation}
Each of the following set of methods generates a single access point
on an instrument given $x$ and $y$ coordinates. The \MethodIndex{point(...)}
methods only differ from the other two in that they do not lead to the
automatic generation of graphical markers in the instrument
visualisation window.

\begin{verbatim}
    (<x>, <y>)
    (<x>)
    point(<x>, <y>)
    point(<x>)

    e.g.

    rect(left,0.5)
    string(0.7)
    ellipse.point(0.1,centre)
\end{verbatim}

\subsection{Accessing Individual Cells}
\hierindex{methods!accessing cells}
The following method allows access to the nearest cell to the
position on the instrument specified. Once the cell has been
selected its internal attributes can be examined. You should
never really need to use this method as it was really designed
for \tao's internal use. Also access points provide a much
more flexible mechanism for interacting with instruments.

\begin{verbatim}
    at(<x>, <y>)
\end{verbatim}

\section{Device Methods}
\hierindex{Device methods}
\hierindex{methods!Device}
This section describes all the methods available to the different
devices. It begins by listing generic methods which are applicable
to any device.

\subsection{Generic Device Methods}
The following methods are available with any device.

\begin{verbatim}
    getName()
    getX()
    getY()
    apply(<accessPoint>)
    remove()
\end{verbatim}

The \MethodIndex{getName} method returns a string containing the
name of the device. The \MethodIndex{getX} and \MethodIndex{getY}
methods return the current position in instrument coordinates of
the device if it has actually been applied to an instrument.
Otherwise they return zero. The return values are also zero
if the device has been applied to an instrument and removed
again.

\subsection{Bow Methods}
\index{Bow methods}
\hierindex{methods!Bow}
The main attributes of a Bow device are the force with which
it is applied to the instrument (which has a marked effect on the bow's
ability to sustain the frictional forces needed to move the instrument),
and its velocity. The following methods are available with a bow.

\begin{verbatim}
    setForce(<force>)
    setVelocity(<velocity>)
    getForce()
    getVelocity()
    (<accessPoint>)
    (<instr>, <x>)
    (<instr>, <x>, <y>)
\end{verbatim}

The last three methods provide three different ways to apply
a bow to an instrument. The first specifies an access point, the second
an instrument and a single $x$ coordinate (for a string), and 
the third an instrument and both $x$ \& $y$ coordinates.

Although these methods are available in a script an alternative
syntax is usually used for applying a bow to an instrument at a specific
point. This consists of the access point specification (the instrument
name followed by the instrument coordinates enclosed in parentheses)
followed by the \emph{apply} operator: \verb|--|, followed by the
name of the bow. For example in the following code fragments the left
and right hand sides are exactly equivalent:

\begin{verbatim}
    Bow bow;
    
    Init:
        bow(string(0.1));    <==>    string(0.1) -- bow;
        bow(rect(0.5,0.7));  <==>    rect(0.5,0.7) -- bow;
        ...
\end{verbatim}

\subsection{Hammer Methods}
\index{Hammer methods}
\hierindex{methods!Hammer}
For a description of the Hammer device see section \ref{section:hammer_device}.
The following methods are available with a hammer.

\begin{verbatim}
    reset()
    drop()
    (<accessPoint>)
    (<instr>, <x>)
    (<instr>, <x>, <y>)
    setMass(<m>)
    setPosition(<p>)
    setInitVelocity(<v>)
    setGravity(<g>)
    setHeight(<h>)
    setDamping(<d>)
    setHardness(<h>)
    setMaxImpacts(<maxImpacts>)
    getMass()
    getPosition()
    getVelocity()
    getInitVelocity()
    getGravity()
    getHeight()
    getDamping()
    getHardness()
    numberOfImpacts()
    getMaxImpacts()
\end{verbatim}

The \MethodIndex{reset} method resets the hammer to its initial height and
causes it to wait for a subsequent call to the \MethodIndex{drop} method
before the hammer will start falling and interacting with the instrument.
As with the bow device there are three unnamed methods for specifying
the access point with which the hammer will interact. The first expects
an access point, the second an instrument name and a single $x$ coordinate
and the third an instrument name followed by an $x$ and $y$ coordinate.

The \verb|set...| family of methods are used to set the various attributes
of the hammer. Note that the \verb|setHeight| method sets the height from
which the hammer is dropped whereas the \verb|setPosition| method sets
the instantaneous height of the hammer. The \verb|setInitVelocity| method
sets the initial velocity of the hammer immediately after the \verb|drop|
method has been called. The \verb|setHardness| method sets the strength
of the spring which is used to simulate the elastic connection with the
instrument. This is usually a value in the range [0..1] where 0 means
the spring has no effect and 1 means that the spring has the same strength
as the springs used in \tao's material. Values greater than 1 can also
be used although this can lead to the model becoming unstable, due to
the inherent limitations in modeling a continuous physical system using
discrete time steps or ticks. 

\subsection{Connector Methods}
\index{Connector methods}
\hierindex{methods!Connector}
The Connector methods listed below allow any combination of access
and anchor points to be coupled together. To recap, an \Term{anchor point}
is a fixed numerical value (usually 0.0) and a spring is connected
between the access point specified and this anchor point, effectively
restricting the vibrations of the instrument at that point.

\begin{verbatim}
    (<access point 1>, <access point 2>)
    (<access point 1>, <access point 2>, <strength>)
    (<access point>, <anchor>)
    (<access point>, <anchor>, <strength>)
    (<anchor>, <access point>)
    (<anchor>, <access point>, <strength>)
\end{verbatim}

The \verb|<strength>|\index{spring strength} argument expected by
some of the methods sets the strength of the spring used to connect
the two points. It is usually a value in the range [0..1] but higher
values may sometimes work. You should be aware though that if you
use a value higher than 1 \tao's cellular model may become unstable,
leading to exponentially increasing noisy vibrations. This is limitation
inherent the kind of discrete time step modelling used by \tao.

\subsection{Stop Methods}
\index{Stop methods}
\hierindex{methods!Stop}
The following stop device methods are available:

\begin{verbatim}
    dampModeOn()
    dampModeOff()
    setAmount(<amount>)
    setDamping(<damping>)
    (<access point>)
    (<instr>, <x>)
    (<instr>, <x>, <y>)
    (<string>, <pitch>)
\end{verbatim}

To briefly recap, the Stop device provides a rudimentary
mechanism for stopping strings in order to obtain specific pitches
from them. The \verb|<amount>| attribute is a value in the
range [0..1], with 0 meaning that the string is not stopped at all
and 1 meaning that it is fully stopped. The \verb|<damping>|
attribute determines how highly damped the left hand side of the string
will be (the right hand side is given the appropriate length to achieve
the specified pitch).

The unnamed \verb|(...)| methods are used to apply the stop to an
instrument. Much like the bow and hammer devices there are three standard
versions available, expecting either an access point, an instrument and
single $x$ coordinate, or an instrument and both $x$ and $y$ coordinates.
However for the Stop device there is a fourth method
\verb|(<string>, <pitch>)| available. This method, given a string
instrument and a pitch as arguments will automatically calculate at what
point the Stop device should be applied to the string in order to produce
the desired pitch.

Note that it is always the portion of the string to the right of the
applied Stop device which has the correct pitch. This should be borne
in mind if connecting the string to other components. If you build your
instrument with the left hand sides of each string attached to some sort
of resonator you will get all the wrong pitches when you start to play
the instrument!

Note also that as with the Bow and Hammer devices the preferred
syntax to use in a script when applying the device to an instrument
is as follows:

\begin{verbatim}
    string(0.1) -- stop;
\end{verbatim}

which is exactly equivalent to:

\begin{verbatim}
    stop(string(0.1));
\end{verbatim}

The first format is more commonly used since it is more clearly
legible when quickly scanning a script to see what it does. Wherever
the \verb|--| operator appears in a script, you know that there is
some kind of interfacing between access points and devices taking
place.

\subsection{Output Methods}
\index{Output methods}
\hierindex{methods!Output}
The Output device provides methods for writing sound
samples out to the various channels of its associated output file.
The  \MethodIndex{ch1} and \MethodIndex{chL} methods are equivalent, as are the
\MethodIndex{ch2} and \MethodIndex{chR} methods. Obviously \Method{chL}
and \Method{chR} are designed for use with two channel stereo
output. In each case \verb|<value>| is an arbitrary mathematical
expression yielding a floating point value. There is no need to
ensure that the output samples stay within a predefined range, since
all \tao\ output files are normalised as a post-processing stage 
before conversion into a more conventional integer-based format
such as WAV. This is achieved with the \Prog{taosf}
shell command (see section \ref{section:output_files}).

\begin{verbatim}
    ch1(<value>)
    ch2(<value>)
    ch3(<value>)
    ch4(<value>)
    chL(<value>)
    chR(<value>)
\end{verbatim}

\section{Access Point Methods}
These methods are some of the most important since they are the
ones which allow simulated physical interaction with \tao's
cellular material.

\begin{verbatim}
    getPosition()
    getVelocity()
    getForce()
    getInstrument()
    applyForce(<force>)
    clear()   
\end{verbatim}

The \MethodIndex{getPosition}, \MethodIndex{getForce} and \MethodIndex{getVelocity}
methods return the physical attributes of the material at the access point.
All three values returned are with respect to the $z$ axis of the material.

The \MethodIndex{getInstrument} method returns the instrument on which the
access point is operating. 

The \MethodIndex{applyForce} method applies the given force at the position
of the access point, not surprisingly! Note that in the same way that an
access point can be used to read the physical attributes of the material at
any point, forces can be applied at any point too. This means that you
could for example generate a moving access point whose instrument coordinates
were governed by some time varying value generated by an arbitrary expression,
and apply a time varying force at that moving point. 

The following example illustrates the use of the access point methods 
(although I don't know whether it would produce interesting sounds or not).

\begin{verbatim}
    Audio rate: 44100;
    
    String s(200 Hz, 20 secs);
    AccessPoint p1, p2;
    Param x1, x2;
    
    Init:
        s.lockEnds();
        ...
    
    Score 10 secs:
        x1=0.5+0.5*cos(Time*1000.0);
        x2=0.5+0.5*cos(Time*1100.0);
        p1=s(x1);
        p2=s(x2);
        
        Every 0.01 secs:
            Print "At time ", Time, ", position=", p1.getPosition(), 
                  " velocity=", p1.getVelocity(),  newline;
            ...
    
        If p1.getForce() < 1:
            p2.applyForce(1.0);
            ...
        If p2.getVelocity() > -1:
            p1.applyForce(-1.0);
            ...
        ...
\end{verbatim}

Finally, the \MethodIndex{clear} method resets the access point to
be null. Attempting to read any physical attributes off of the
access point will result in zero being returned. In addition,
attempting to apply a force to a null access point has no effect.

\section{Pitch Methods}















